% !TeX root = ../ThesisMain.tex
% !TeX program = XeLaTeX
% !TeX encoding = UTF-8
% !TeX spellcheck = en_GB

\documentclass[../ThesisMain]{subfiles}
\ifSubfilesClassLoaded{}{}%

\begin{document}
\doublespacing%

\chapter{Code and Data Availability}
\label{appendix:code}

\section*{Repository Information}

All source code, training data, trained models, and analysis scripts developed for this thesis are publicly available in a GitHub repository:

\begin{center}
\fbox{\parbox{0.85\textwidth}{
\centering
\textbf{GitHub Repository}\\[6pt]
\texttt{https://github.com/JianouJiang/flow2Dtube}\\[6pt]
\small Licensed under MIT License
}}
\end{center}

\noindent The repository is organised into the following primary directories:

\begin{table}[H]
\centering
\caption*{\textbf{Repository Structure}}
\begin{tabular}{ll}
\toprule
\textbf{Directory} & \textbf{Contents} \\
\midrule
\texttt{BENCHMARKS/} & Experimental benchmark data processing \\
\texttt{TRAINING\_DATA/} & 244 CFD simulation cases (25,485 samples) \\
\texttt{FEATURE\_VARIABLES\_AS\_INPUTS/} & Chapter 5: Physics-based feature library \\
\texttt{FEATURE\_VARIABLES\_AS\_NEURONS/} & Chapter 6: Neuron correlation analysis \\
\texttt{FEATURE\_VARIABLES\_AS\_PINN/} & Chapter 7: Physics-constrained learning \\
\texttt{IDENTIFY\_FLOW\_SEPARATION/} & Chapter 8: Separation detection classifiers \\
\texttt{OPENFOAM\_INTEGRATION/} & Chapter 9: C++ boundary conditions \\
\texttt{ADDITIONAL\_STUDIES/} & Supplementary experiments \\
\bottomrule
\end{tabular}
\end{table}

\section*{Chapter-by-Chapter Code Mapping}

The following tables provide a comprehensive mapping between thesis figures, tables, and results and their corresponding source code locations.

\subsection*{Cross-Folder Dependencies}

Several chapters use code from multiple folders. The following summary highlights these dependencies:

\begin{table}[H]
\centering
\small
\begin{tabular}{lll}
\toprule
\textbf{Chapter} & \textbf{Primary Folder} & \textbf{Additional Folders Used} \\
\midrule
Chapter 3 & \texttt{TRAINING\_DATA/} & \texttt{BENCHMARKS/} (validation figures) \\
Chapter 4 & \texttt{FEATURE\_VARIABLES\_AS\_INPUTS/} & \texttt{TRAINING\_DATA/} (data analysis) \\
Chapter 5 & \texttt{FEATURE\_VARIABLES\_AS\_INPUTS/} & \texttt{TRAINING\_DATA/} (feature extraction) \\
Chapter 7 & \texttt{FEATURE\_VARIABLES\_AS\_PINN/} & \texttt{FEATURE\_VARIABLES\_AS\_NEURONS/} (L1-PINN) \\
Chapter 9 & \texttt{OPENFOAM\_INTEGRATION/} & \texttt{BENCHMARKS/} (validation figures) \\
Chapter 10 & \texttt{ADDITIONAL\_STUDIES/} & All folders (comparative analysis) \\
\bottomrule
\end{tabular}
\end{table}

%==============================================================================
\subsection*{Chapter 3: Methodology and Structured Data Generation}
%==============================================================================

\noindent\textit{Primary code locations: \texttt{TRAINING\_DATA/} and \texttt{BENCHMARKS/}}

\begin{table}[H]
\centering
\small
\caption*{\textbf{Chapter 3 -- Figures and Tables}}
\begin{tabular}{p{3.5cm}p{8cm}}
\toprule
\textbf{Item} & \textbf{Code/Data Location} \\
\midrule
\multicolumn{2}{l}{\textit{Main figure generation script:}} \\
All Chapter 3 figures & \texttt{TRAINING\_DATA/scripts/thesis\_figures/generate\_chapter3\_figures.py} \\
\midrule
Figure 3.1 (Data pipeline) & \texttt{TRAINING\_DATA/scripts/generate\_cases.py}, \texttt{mesh\_utils.py} \\
Figure 3.2 (Geometry family) & \texttt{TRAINING\_DATA/scripts/geometry\_sampling.py} \\
Figure 3.3 (Parameter space) & \texttt{TRAINING\_DATA/scripts/visualize\_geometries.py} \\
Figure 3.4 (Mesh comparison) & \texttt{TRAINING\_DATA/scripts/mesh\_sizing.py} \\
Figure 3.5 (Grid independence) & \texttt{TRAINING\_DATA/scripts/validate\_fine\_mesh.py} \\
Figure 3.6 (Boundary conditions) & OpenFOAM case files in \texttt{TRAINING\_DATA/data/cases\_WF/} \\
Figure 3.7 (Residual convergence) & \texttt{TRAINING\_DATA/scripts/visualize\_results.py} \\
Figure 3.8 (Velocity profiles) & \texttt{BENCHMARKS/turbulent/channel\_flow/benchmark/visualize\_channel\_flow.py} \\
Figure 3.9 (Diffuser validation) & \texttt{BENCHMARKS/turbulent/diffuser/benchmark/visualize\_diffuser.py} \\
Figure 3.10 (Heat transfer) & \texttt{BENCHMARKS/turbulent/channel\_flow\_thermal/benchmark/visualize\_channel\_flow\_thermal.py} \\
Figure 3.11 (BFS geometry) & \texttt{BENCHMARKS/turbulent/backward\_facing\_step/benchmark/visualize\_backward\_facing\_step.py} \\
Figure 3.12 (BFS $C_f$) & \texttt{BENCHMARKS/turbulent/backward\_facing\_step/openfoam/validate\_bfs.py} \\
Figure 3.13 (Wall hump) & \texttt{BENCHMARKS/turbulent/wall\_mounted\_hump/benchmark/visualize\_wall\_hump.py} \\
Figure 3.14 (Periodic hills) & \texttt{BENCHMARKS/turbulent/periodic\_hills/benchmark/visualize\_periodic\_hills.py} \\
Figure 3.15 (Dataset statistics) & \texttt{TRAINING\_DATA/scripts/visualization.py} \\
\midrule
Table 3.1 (Mesh quality) & \texttt{TRAINING\_DATA/scripts/mesh\_utils.py} \\
Table 3.2 (Schemes) & OpenFOAM \texttt{fvSchemes} in \texttt{TRAINING\_DATA/data/} \\
Table 3.3 (Relaxation) & OpenFOAM \texttt{fvSolution} in \texttt{TRAINING\_DATA/data/} \\
Table 3.4 (DNS databases) & \texttt{BENCHMARKS/integrate\_benchmark\_data.py} \\
Table 3.5 (Benchmark summary) & \texttt{BENCHMARKS/compare\_with\_benchmarks.py} \\
Table 3.6 (Dataset summary) & \texttt{TRAINING\_DATA/scripts/combine\_datasets.py} \\
\bottomrule
\end{tabular}
\end{table}

%==============================================================================
\subsection*{Chapter 4: Data-Driven Velocity and Thermal Wall Functions}
%==============================================================================

\noindent\textit{Primary code locations: \texttt{FEATURE\_VARIABLES\_AS\_INPUTS/} and \texttt{TRAINING\_DATA/}}

\begin{table}[H]
\centering
\small
\caption*{\textbf{Chapter 4 -- Figures and Tables}}
\begin{tabular}{p{3.5cm}p{8cm}}
\toprule
\textbf{Item} & \textbf{Code/Data Location} \\
\midrule
Figure 4.1 (Learning curves) & \texttt{FEATURE\_VARIABLES\_AS\_INPUTS/exp1\_primitive\_only/train\_primitive\_vs\_physics.py} \\
Figure 4.2 (Predictions) & \texttt{FEATURE\_VARIABLES\_AS\_INPUTS/exp1\_primitive\_only/quick\_figures.py} \\
Figure 4.3 (WF comparison) & \texttt{FEATURE\_VARIABLES\_AS\_INPUTS/exp1\_primitive\_only/train\_primitive\_vs\_physics.py} \\
\midrule
Table 4.1 (Hyperparameter search) & \texttt{FEATURE\_VARIABLES\_AS\_INPUTS/exp3\_hyperparameters/train\_hyperparam\_search.py} \\
Table 4.2 (Hyperparameter results) & \texttt{FEATURE\_VARIABLES\_AS\_INPUTS/exp3\_hyperparameters/} (results/) \\
Table 4.3 (Stencil study) & \texttt{FEATURE\_VARIABLES\_AS\_INPUTS/exp2\_stencil\_size/train\_stencil\_study.py} \\
Table 4.4 (Baseline performance) & \texttt{FEATURE\_VARIABLES\_AS\_INPUTS/exp1\_primitive\_only/} (metrics.json) \\
Table 4.5 (Traditional comparison) & \texttt{FEATURE\_VARIABLES\_AS\_INPUTS/exp1\_primitive\_only/train\_primitive\_vs\_physics.py} \\
Table 4.6 (Data sources) & \texttt{FEATURE\_VARIABLES\_AS\_INPUTS/exp6\_data\_source\_study/train\_comparison.py} \\
Table 4.7 (Flow regime) & \texttt{TRAINING\_DATA/scripts/features.py} \\
Table 4.8 (Regime performance) & \texttt{FEATURE\_VARIABLES\_AS\_INPUTS/exp5\_final\_model/robustness\_verification.py} \\
\bottomrule
\end{tabular}
\end{table}

%==============================================================================
\subsection*{Chapter 5: Physics-Based Feature Variables as Network Inputs}
%==============================================================================

\noindent\textit{Primary code location: \texttt{FEATURE\_VARIABLES\_AS\_INPUTS/}}

\begin{table}[H]
\centering
\small
\caption*{\textbf{Chapter 5 -- Figures and Tables}}
\begin{tabular}{p{3.5cm}p{8cm}}
\toprule
\textbf{Item} & \textbf{Code/Data Location} \\
\midrule
\multicolumn{2}{l}{\textit{Main figure generation scripts:}} \\
All Chapter 5 figures & \texttt{FEATURE\_VARIABLES\_AS\_INPUTS/exp1\_primitive\_only/chapter5\_comprehensive\_figures.py} \\
& \texttt{FEATURE\_VARIABLES\_AS\_INPUTS/exp1\_primitive\_only/final\_chapter5\_figures.py} \\
& \texttt{FEATURE\_VARIABLES\_AS\_INPUTS/exp1\_primitive\_only/chapter5\_final.py} \\
\midrule
Figure 5.1 (Physics overview) & \texttt{exp1\_primitive\_only/comprehensive\_physics\_analysis.py} \\
Figure 5.2 (Wall function laws) & \texttt{exp1\_primitive\_only/chapter5\_comprehensive\_figures.py} \\
Figure 5.3 (Velocity profiles) & \texttt{exp1\_primitive\_only/chapter5\_comprehensive\_figures.py} \\
Figure 5.4 (Pressure gradient) & \texttt{exp1\_primitive\_only/chapter5\_comprehensive\_figures.py} \\
Figure 5.5 (Flow contours) & \texttt{exp1\_primitive\_only/chapter5\_comprehensive\_figures.py} \\
Figure 5.6 ($\tau_w$ distribution) & \texttt{exp1\_primitive\_only/chapter5\_comprehensive\_figures.py} \\
Figure 5.7--5.17 & \texttt{exp1\_primitive\_only/final\_chapter5\_figures.py} \\
\midrule
Table 5.1 (Feature summary) & \texttt{TRAINING\_DATA/scripts/features.py} \\
Table 5.2 (Dataset composition) & \texttt{TRAINING\_DATA/scripts/combine\_datasets.py} \\
Table 5.3 (Parameter ranges) & \texttt{TRAINING\_DATA/scripts/config.py} \\
Table 5.4 (Feature ranges) & \texttt{exp1\_primitive\_only/train\_primitive\_vs\_physics.py} \\
Table 5.5 (Primitive vs physics) & \texttt{exp1\_primitive\_only/train\_primitive\_vs\_physics.py} \\
Table 5.6 (Separation comparison) & \texttt{exp1\_primitive\_only/comprehensive\_physics\_analysis.py} \\
Table 5.7 (Traditional WF) & \texttt{exp1\_primitive\_only/train\_primitive\_vs\_physics.py} \\
Table 5.8 (Cross-evaluation) & \texttt{exp6\_data\_source\_study/train\_comparison.py} \\
Table 5.9 (Core features) & \texttt{config.py} (feature definitions) \\
\bottomrule
\end{tabular}
\end{table}

%==============================================================================
\subsection*{Chapter 6: Physics-Based Feature Variables as Hidden Layer Neurons}
%==============================================================================

\noindent\textit{Primary code location: \texttt{FEATURE\_VARIABLES\_AS\_NEURONS/}}

\begin{table}[H]
\centering
\small
\caption*{\textbf{Chapter 6 -- Figures and Tables}}
\begin{tabular}{p{3.5cm}p{8cm}}
\toprule
\textbf{Item} & \textbf{Code/Data Location} \\
\midrule
\multicolumn{2}{l}{\textit{Main experiment scripts:}} \\
All experiments & \texttt{FEATURE\_VARIABLES\_AS\_NEURONS/run\_all\_experiments.py} \\
\midrule
Figure 6.1 (Top correlations) & \texttt{experiments/exp2\_neuron\_correlation/compute\_correlations.py} \\
Figure 6.2 (Correlation heatmap) & \texttt{experiments/exp2\_neuron\_correlation/compute\_correlations.py} \\
Figure 6.3 (Architecture invariance) & \texttt{experiments/exp4\_architecture\_comparison/architecture\_invariance.py} \\
Figure 6.4 (Network interpretation) & \texttt{utils/visualization.py} \\
Figure 6.5--6.7 (Hybrid) & \texttt{experiments/exp3\_neuron\_replacement/replace\_neurons.py} \\
Figure 6.8--6.11 & \texttt{experiments/exp5\_data\_source\_study/compare\_sources.py} \\
\midrule
Table 6.1 (Basic inputs) & \texttt{config.py} \\
Table 6.2 (Accuracy) & \texttt{experiments/exp1\_train\_l1pinn/train\_l1pinn.py} \\
Table 6.3 (Top neurons) & \texttt{experiments/exp2\_neuron\_correlation/compute\_correlations.py} \\
Table 6.4 (Invariant features) & \texttt{experiments/exp4\_architecture\_comparison/architecture\_invariance.py} \\
Table 6.5--6.8 (Hybrid) & \texttt{experiments/exp3\_neuron\_replacement/replace\_neurons.py} \\
Table 6.9--6.11 (Data sources) & \texttt{experiments/exp5\_data\_source\_study/compare\_sources.py} \\
Table 6.12--6.14 & \texttt{utils/correlations.py}, \texttt{utils/replacement.py} \\
\bottomrule
\end{tabular}
\end{table}

%==============================================================================
\subsection*{Chapter 7: Physics-Constrained Learning}
%==============================================================================

\noindent\textit{Primary code locations: \texttt{FEATURE\_VARIABLES\_AS\_PINN/} and \texttt{FEATURE\_VARIABLES\_AS\_NEURONS/}}

\vspace{0.5em}
\noindent\textbf{Note:} Chapter 7 uses code from \emph{two} folders due to the L1-PINN architecture developed in Chapter 6.

\begin{table}[H]
\centering
\small
\caption*{\textbf{Chapter 7 -- Figures and Tables}}
\begin{tabular}{p{3.5cm}p{8cm}}
\toprule
\textbf{Item} & \textbf{Code/Data Location} \\
\midrule
\multicolumn{2}{l}{\textit{Main experiment scripts:}} \\
PINN experiments & \texttt{FEATURE\_VARIABLES\_AS\_PINN/run\_pinn\_experiments.py} \\
Chapter 7 results & \texttt{FEATURE\_VARIABLES\_AS\_NEURONS/run\_chapter7\_experiments.py} (cross-ref) \\
\midrule
Figure 7.1 (PINN summary) & \texttt{FEATURE\_VARIABLES\_AS\_PINN/run\_pinn\_experiments.py} \\
Figure 7.2 (Predictions) & \texttt{FEATURE\_VARIABLES\_AS\_PINN/utils/visualization.py} \\
Figure 7.3 (Feature suitability) & \texttt{FEATURE\_VARIABLES\_AS\_PINN/run\_pinn\_experiments.py} \\
Figure 7.4 (Physics ablation) & \texttt{FEATURE\_VARIABLES\_AS\_PINN/run\_pinn\_experiments.py} \\
Figure 7.5 (Loss evolution) & \texttt{FEATURE\_VARIABLES\_AS\_PINN/run\_pinn\_experiments.py} \\
\midrule
Table 7.1 (PINN results) & \texttt{FEATURE\_VARIABLES\_AS\_PINN/run\_pinn\_experiments.py} \\
Table 7.2 (Chapter comparison) & \texttt{FEATURE\_VARIABLES\_AS\_PINN/run\_pinn\_experiments.py} \\
\midrule
\multicolumn{2}{l}{\textit{Supporting code:}} \\
Physics residual computation & \texttt{FEATURE\_VARIABLES\_AS\_PINN/physics/residuals.py} \\
Finite difference methods & \texttt{FEATURE\_VARIABLES\_AS\_PINN/physics/finite\_differences.py} \\
PINN model architecture & \texttt{FEATURE\_VARIABLES\_AS\_PINN/models/pinn\_model.py} \\
L1-PINN architecture & \texttt{FEATURE\_VARIABLES\_AS\_NEURONS/utils/l1pinn.py} \\
\bottomrule
\end{tabular}
\end{table}

%==============================================================================
\subsection*{Chapter 8: Identification of Flow Separation}
%==============================================================================

\noindent\textit{Primary code location: \texttt{IDENTIFY\_FLOW\_SEPARATION/}}

\begin{table}[H]
\centering
\small
\caption*{\textbf{Chapter 8 -- Figures and Tables}}
\begin{tabular}{p{3.5cm}p{8cm}}
\toprule
\textbf{Item} & \textbf{Code/Data Location} \\
\midrule
\multicolumn{2}{l}{\textit{Main scripts:}} \\
All Chapter 8 figures & \texttt{IDENTIFY\_FLOW\_SEPARATION/generate\_chapter8\_figures.py} \\
All experiments & \texttt{IDENTIFY\_FLOW\_SEPARATION/run\_all\_experiments.py} \\
Fast experiments & \texttt{IDENTIFY\_FLOW\_SEPARATION/run\_experiments\_fast.py} \\
\midrule
Figure 8.1 (Separation flow) & \texttt{generate\_chapter8\_figures.py} \\
Figure 8.2 (Classifier comparison) & \texttt{utils/classifiers.py} \\
Figure 8.3 (Generalization) & \texttt{generate\_chapter8\_figures.py} \\
Figure 8.4--8.5 (Hybrid) & \texttt{generate\_chapter8\_figures.py} \\
Figure 8.6 (Spatial detection) & \texttt{generate\_chapter8\_figures.py} \\
\midrule
Table 8.1 (Distribution shift) & \texttt{utils/labeling.py} \\
Table 8.2 (Classifiers) & \texttt{config.py} \\
Table 8.3 (Baseline results) & \texttt{run\_all\_experiments.py} \\
Table 8.4 (Confusion matrix) & \texttt{utils/classifiers.py} \\
Table 8.5 (Minimal features) & \texttt{utils/feature\_selection.py} \\
Table 8.6 (Physics constrained) & \texttt{utils/physics\_losses.py} \\
Table 8.7--8.8 & \texttt{run\_all\_experiments.py} \\
\bottomrule
\end{tabular}
\end{table}

%==============================================================================
\subsection*{Chapter 9: OpenFOAM Integration and Comprehensive Evaluation}
%==============================================================================

\noindent\textit{Primary code locations: \texttt{OPENFOAM\_INTEGRATION/} and \texttt{BENCHMARKS/}}

\vspace{0.5em}
\noindent\textbf{Note:} Chapter 9 validation figures use benchmark data and visualization scripts from the \texttt{BENCHMARKS/} folder.

\begin{table}[H]
\centering
\small
\caption*{\textbf{Chapter 9 -- Figures and Tables}}
\begin{tabular}{p{3.5cm}p{8cm}}
\toprule
\textbf{Item} & \textbf{Code/Data Location} \\
\midrule
\multicolumn{2}{l}{\textit{Main scripts:}} \\
Thesis validation figures & \texttt{BENCHMARKS/generate\_thesis\_validation\_figures.py} \\
Model training & \texttt{OPENFOAM\_INTEGRATION/scripts/train\_models.py} \\
Model export & \texttt{OPENFOAM\_INTEGRATION/scripts/export\_models.py} \\
Experiment runner & \texttt{OPENFOAM\_INTEGRATION/scripts/run\_experiments.py} \\
Results evaluation & \texttt{OPENFOAM\_INTEGRATION/scripts/evaluate\_results.py} \\
\midrule
\multicolumn{2}{l}{\textit{Benchmark validation (from BENCHMARKS/):}} \\
Figure 9.2--9.7 (Validation) & \texttt{BENCHMARKS/turbulent/*/openfoam/validate\_*.py} \\
BFS validation & \texttt{BENCHMARKS/turbulent/backward\_facing\_step/openfoam/validate\_backward\_facing\_step.py} \\
Periodic hills & \texttt{BENCHMARKS/turbulent/periodic\_hills/openfoam/validate\_periodic\_hills.py} \\
Wall hump & \texttt{BENCHMARKS/turbulent/wall\_mounted\_hump/openfoam/validate\_wall\_hump.py} \\
Thermal validation & \texttt{BENCHMARKS/turbulent/*\_thermal/openfoam/validate\_*\_thermal.py} \\
\midrule
\multicolumn{2}{l}{\textit{OpenFOAM integration (from OPENFOAM\_INTEGRATION/):}} \\
Figure 9.1 (Architecture) & \texttt{scripts/export\_models.py} \\
Figure 9.8--9.9 (Timing) & \texttt{scripts/evaluate\_results.py} \\
Figure 9.10--9.15 (3D cases) & \texttt{scripts/run\_experiments.py} \\
Figure 9.16--9.22 & \texttt{scripts/evaluate\_results.py} \\
\midrule
Table 9.1--9.14 & \texttt{OPENFOAM\_INTEGRATION/scripts/evaluate\_results.py} \\
& \texttt{BENCHMARKS/compare\_with\_benchmarks.py} \\
\bottomrule
\end{tabular}
\end{table}

%==============================================================================
\subsection*{Chapter 10: Conclusion and Future Work}
%==============================================================================

\noindent\textit{Primary code location: \texttt{ADDITIONAL\_STUDIES/}}

\begin{table}[H]
\centering
\small
\caption*{\textbf{Chapter 10 -- Figures and Tables}}
\begin{tabular}{p{3.5cm}p{8cm}}
\toprule
\textbf{Item} & \textbf{Code/Data Location} \\
\midrule
Table 10.1 (Method comparison) & \texttt{ADDITIONAL\_STUDIES/run\_all\_experiments.py} \\
& \texttt{ADDITIONAL\_STUDIES/compare\_traditional\_wf.py} \\
Table 10.2 (Method selection) & \texttt{ADDITIONAL\_STUDIES/evaluate\_separation.py} \\
\midrule
\multicolumn{2}{l}{\textit{Supporting experiments in ADDITIONAL\_STUDIES/:}} \\
WF vs no-WF study & \texttt{exp1\_wf\_vs\_nowf/train\_with\_wf.py} \\
Attached vs separated & \texttt{exp2\_attached\_to\_separated/train\_attached\_only.py} \\
Data augmentation & \texttt{exp3\_experimental\_augmentation/train\_with\_augmentation.py} \\
Distribution shift & \texttt{exp4\_distribution\_shift/analyze\_distributions.py} \\
Reynolds analogy & \texttt{exp5\_reynolds\_analogy/analyze\_velocity\_thermal.py} \\
DNS/experimental data & \texttt{exp6\_real\_dns\_experimental/train\_with\_real\_benchmarks.py} \\
\bottomrule
\end{tabular}
\end{table}

%==============================================================================
\section*{Key Scripts and Entry Points}
%==============================================================================

The following scripts serve as main entry points for reproducing the key results:

\begin{table}[H]
\centering
\small
\begin{tabular}{p{5.5cm}p{6.5cm}}
\toprule
\textbf{Script} & \textbf{Purpose} \\
\midrule
\multicolumn{2}{l}{\textit{Data generation:}} \\
\texttt{TRAINING\_DATA/scripts/generate\_cases.py} & Generate CFD simulation cases \\
\texttt{TRAINING\_DATA/scripts/run\_simulations.py} & Run all 244 training simulations \\
\texttt{TRAINING\_DATA/scripts/extract\_and\_pair.py} & Extract training data pairs \\
\midrule
\multicolumn{2}{l}{\textit{Chapter-specific experiments:}} \\
\texttt{FEATURE\_VARIABLES\_AS\_INPUTS/run\_all\_stages.py} & Chapter 4--5 experiments \\
\texttt{FEATURE\_VARIABLES\_AS\_NEURONS/run\_all\_experiments.py} & Chapter 6 experiments \\
\texttt{FEATURE\_VARIABLES\_AS\_PINN/run\_pinn\_experiments.py} & Chapter 7 PINN experiments \\
\texttt{IDENTIFY\_FLOW\_SEPARATION/run\_all\_experiments.py} & Chapter 8 experiments \\
\texttt{OPENFOAM\_INTEGRATION/scripts/run\_experiments.py} & Chapter 9 integration \\
\texttt{ADDITIONAL\_STUDIES/run\_all\_experiments.py} & Chapter 10 supplementary studies \\
\midrule
\multicolumn{2}{l}{\textit{Figure generation:}} \\
\texttt{TRAINING\_DATA/scripts/thesis\_figures/generate\_chapter3\_figures.py} & Chapter 3 figures \\
\texttt{FEATURE\_VARIABLES\_AS\_INPUTS/exp1\_primitive\_only/chapter5\_comprehensive\_figures.py} & Chapter 5 figures \\
\texttt{IDENTIFY\_FLOW\_SEPARATION/generate\_chapter8\_figures.py} & Chapter 8 figures \\
\texttt{BENCHMARKS/generate\_thesis\_validation\_figures.py} & Chapter 9 validation figures \\
\bottomrule
\end{tabular}
\end{table}

%==============================================================================
\section*{Data Availability}
%==============================================================================

\subsection*{Training Data}

The complete training dataset comprising 244 simulation cases is available in:
\begin{itemize}
    \item \texttt{TRAINING\_DATA/data/cases\_WF/} -- Wall-modelled (coarse mesh) simulations
    \item \texttt{TRAINING\_DATA/data/cases\_WR/} -- Wall-resolved (fine mesh) simulations
\end{itemize}

\noindent Each case directory contains:
\begin{itemize}
    \item OpenFOAM case files (\texttt{0/}, \texttt{constant/}, \texttt{system/})
    \item Converged solution fields (\texttt{U}, \texttt{p}, \texttt{T}, \texttt{k}, \texttt{omega}, \texttt{nut})
    \item Post-processed wall quantities (\texttt{wallShearStress}, \texttt{yPlus})
    \item Extracted stencil features (\texttt{stencil\_features.csv})
\end{itemize}

\subsection*{Benchmark Data}

Experimental and DNS benchmark data used for validation:
\begin{itemize}
    \item \texttt{BENCHMARKS/turbulent/backward\_facing\_step/} -- Driver \& Seegmiller (1985)
    \item \texttt{BENCHMARKS/turbulent/periodic\_hills/} -- Breuer et al. (2009)
    \item \texttt{BENCHMARKS/turbulent/wall\_mounted\_hump/} -- NASA wall-mounted hump
    \item \texttt{BENCHMARKS/turbulent/diffuser/} -- Buice \& Eaton diffuser (1997)
    \item \texttt{BENCHMARKS/turbulent/channel\_flow/} -- Moser, Kim \& Mansour DNS (1999)
    \item \texttt{BENCHMARKS/turbulent/flat\_plate/} -- Flat plate boundary layer
    \item \texttt{BENCHMARKS/turbulent/*\_thermal/} -- Thermal benchmark cases
    \item \texttt{BENCHMARKS/laminar/} -- Laminar validation (Blasius, Pohlhausen)
    \item \texttt{BENCHMARKS/transitional/} -- Transitional flow (T3A, T3AM, SK)
\end{itemize}

\noindent Key benchmark processing scripts:
\begin{itemize}
    \item \texttt{BENCHMARKS/visualize\_benchmarks.py} -- Master visualization script
    \item \texttt{BENCHMARKS/compare\_with\_benchmarks.py} -- Comparison analysis
    \item \texttt{BENCHMARKS/integrate\_benchmark\_data.py} -- Data integration
    \item \texttt{BENCHMARKS/train\_with\_benchmarks.py} -- Benchmark-augmented training
\end{itemize}

\subsection*{Trained Models}

Pre-trained neural network models are provided for direct use:
\begin{itemize}
    \item \texttt{FEATURE\_VARIABLES\_AS\_INPUTS/models/} -- Physics-feature models
    \item \texttt{FEATURE\_VARIABLES\_AS\_NEURONS/models/} -- Interpretable models
    \item \texttt{FEATURE\_VARIABLES\_AS\_PINN/models/} -- PINN models
    \item \texttt{IDENTIFY\_FLOW\_SEPARATION/models/} -- Separation classifiers
    \item \texttt{OPENFOAM\_INTEGRATION/models/} -- Production-ready ONNX models
\end{itemize}

%==============================================================================
\section*{Software Dependencies}
%==============================================================================

\begin{table}[H]
\centering
\begin{tabular}{lll}
\toprule
\textbf{Software} & \textbf{Version} & \textbf{Purpose} \\
\midrule
OpenFOAM & v10 & CFD simulations \\
Python & 3.9+ & ML training and analysis \\
PyTorch & 2.0+ & Neural network training \\
scikit-learn & 1.0+ & Traditional ML classifiers \\
NumPy & 1.21+ & Numerical computations \\
Matplotlib & 3.5+ & Visualisation \\
LibTorch & 2.0+ & C++ neural network inference \\
ONNX Runtime & 1.12+ & Cross-platform model deployment \\
\bottomrule
\end{tabular}
\end{table}

\section*{Reproducibility}

All experiments can be reproduced using the provided scripts. Random seeds are fixed for reproducibility. Complete instructions are provided in the repository README files.

\vspace{1em}
\noindent\textit{For questions or issues regarding the code, please open an issue on the GitHub repository or contact the author.}

\end{document}
