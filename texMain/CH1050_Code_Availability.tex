% !TeX root = ../ThesisMain.tex
% !TeX program = XeLaTeX
% !TeX encoding = UTF-8
% !TeX spellcheck = en_GB

\documentclass[../ThesisMain]{subfiles}
\ifSubfilesClassLoaded{}{}%

\begin{document}
\doublespacing%

\chapter{Code and Data Availability}
\label{appendix:code}

\section*{Repository Information}

All source code, training data, trained models, and analysis scripts developed for this thesis are publicly available in a GitHub repository:

\begin{center}
\fbox{\parbox{0.85\textwidth}{
\centering
\textbf{GitHub Repository}\\[6pt]
\texttt{https://github.com/JianouJiang/flow2Dtube}\\[6pt]
\small Licensed under MIT License
}}
\end{center}

\noindent The repository is organised into the following primary directories:

\begin{table}[H]
\centering
\caption*{\textbf{Repository Structure}}
\footnotesize
\begin{tabular}{@{}ll@{}}
\toprule
\textbf{Directory} & \textbf{Contents} \\
\midrule
\texttt{BENCHMARKS/} & Experimental benchmark data processing \\
\texttt{TRAINING\_DATA/} & 244 CFD simulation cases (25,485 samples) \\
\texttt{FEATURE\_VARIABLES\_AS\_INPUTS/} & Chapter 5: Physics-based feature library \\
\texttt{FEATURE\_VARIABLES\_AS\_NEURONS/} & Chapter 6: Neuron correlation analysis \\
\texttt{FEATURE\_VARIABLES\_AS\_PINN/} & Chapter 7: Physics-constrained learning \\
\texttt{IDENTIFY\_FLOW\_SEPARATION/} & Chapter 8: Separation detection classifiers \\
\texttt{OPENFOAM\_INTEGRATION/} & Chapter 9: C++ boundary conditions \\
\texttt{ADDITIONAL\_STUDIES/} & Supplementary experiments \\
\bottomrule
\end{tabular}
\end{table}

\section*{Chapter-by-Chapter Code Mapping}

The following tables provide a comprehensive mapping between thesis figures, tables, and results and their corresponding source code locations. For brevity, folder prefixes are abbreviated as follows:

\begin{itemize}[noitemsep]
\item \textbf{FVAI} = \texttt{FEATURE\_VARIABLES\_AS\_INPUTS/}
\item \textbf{FVAN} = \texttt{FEATURE\_VARIABLES\_AS\_NEURONS/}
\item \textbf{FVAP} = \texttt{FEATURE\_VARIABLES\_AS\_PINN/}
\item \textbf{IFS} = \texttt{IDENTIFY\_FLOW\_SEPARATION/}
\item \textbf{OFI} = \texttt{OPENFOAM\_INTEGRATION/}
\item \textbf{TD} = \texttt{TRAINING\_DATA/}
\item \textbf{BM} = \texttt{BENCHMARKS/}
\end{itemize}

\subsection*{Cross-Folder Dependencies}

\begin{table}[H]
\centering
\footnotesize
\begin{tabular}{@{}lll@{}}
\toprule
\textbf{Chapter} & \textbf{Primary Folder} & \textbf{Additional Folders} \\
\midrule
Chapter 3 & TD & BM (validation) \\
Chapter 4 & FVAI & TD (data analysis) \\
Chapter 5 & FVAI & TD (feature extraction) \\
Chapter 7 & FVAP & FVAN (L1-PINN) \\
Chapter 9 & OFI & BM (validation) \\
Chapter 10 & ADDITIONAL\_STUDIES & All folders \\
\bottomrule
\end{tabular}
\end{table}

%==============================================================================
\subsection*{Chapter 3: Methodology and Structured Data Generation}
%==============================================================================

\begin{table}[H]
\centering
\footnotesize
\caption*{\textbf{Chapter 3 -- Figures and Tables}}
\begin{tabular}{@{}p{2.8cm}p{9cm}@{}}
\toprule
\textbf{Item} & \textbf{Code/Data Location} \\
\midrule
All Ch.\ 3 figures & TD/scripts/thesis\_figures/generate\_chapter3\_figures.py \\
\midrule
Fig.\ 3.1--3.7 & TD/scripts/ (generate\_cases, mesh\_utils, geometry\_sampling, visualize\_geometries, mesh\_sizing, validate\_fine\_mesh, visualize\_results) \\
Fig.\ 3.8 & BM/turbulent/channel\_flow/benchmark/ \\
Fig.\ 3.9 & BM/turbulent/diffuser/benchmark/ \\
Fig.\ 3.10 & BM/turbulent/channel\_flow\_thermal/benchmark/ \\
Fig.\ 3.11--3.12 & BM/turbulent/backward\_facing\_step/ \\
Fig.\ 3.13 & BM/turbulent/wall\_mounted\_hump/benchmark/ \\
Fig.\ 3.14 & BM/turbulent/periodic\_hills/benchmark/ \\
Fig.\ 3.15 & TD/scripts/visualization.py \\
\midrule
Tables 3.1--3.6 & TD/scripts/ and BM/ (see repository) \\
\bottomrule
\end{tabular}
\end{table}

%==============================================================================
\subsection*{Chapter 4: Data-Driven Velocity and Thermal Wall Functions}
%==============================================================================

\begin{table}[H]
\centering
\footnotesize
\caption*{\textbf{Chapter 4 -- Figures and Tables}}
\begin{tabular}{@{}p{2.8cm}p{9cm}@{}}
\toprule
\textbf{Item} & \textbf{Code/Data Location} \\
\midrule
Figures 4.1--4.3 & FVAI/exp1\_primitive\_only/train\_primitive\_vs\_physics.py, quick\_figures.py \\
\midrule
Tables 4.1--4.2 & FVAI/exp3\_hyperparameters/train\_hyperparam\_search.py \\
Table 4.3 & FVAI/exp2\_stencil\_size/train\_stencil\_study.py \\
Tables 4.4--4.5 & FVAI/exp1\_primitive\_only/ \\
Table 4.6 & FVAI/exp6\_data\_source\_study/train\_comparison.py \\
Tables 4.7--4.8 & TD/scripts/features.py, FVAI/exp5\_final\_model/ \\
\bottomrule
\end{tabular}
\end{table}

%==============================================================================
\subsection*{Chapter 5: Physics-Based Feature Variables as Network Inputs}
%==============================================================================

\begin{table}[H]
\centering
\footnotesize
\caption*{\textbf{Chapter 5 -- Figures and Tables}}
\begin{tabular}{@{}p{2.8cm}p{9cm}@{}}
\toprule
\textbf{Item} & \textbf{Code/Data Location} \\
\midrule
All Ch.\ 5 figures & FVAI/exp1\_primitive\_only/chapter5\_comprehensive\_figures.py, final\_chapter5\_figures.py, chapter5\_final.py \\
\midrule
Figures 5.1--5.6 & FVAI/exp1\_primitive\_only/comprehensive\_physics\_analysis.py \\
Figures 5.7--5.17 & FVAI/exp1\_primitive\_only/final\_chapter5\_figures.py \\
\midrule
Tables 5.1--5.9 & TD/scripts/features.py, config.py; FVAI/exp1\_primitive\_only/, exp6\_data\_source\_study/ \\
\bottomrule
\end{tabular}
\end{table}

%==============================================================================
\subsection*{Chapter 6: Physics-Based Feature Variables as Hidden Layer Neurons}
%==============================================================================

\begin{table}[H]
\centering
\footnotesize
\caption*{\textbf{Chapter 6 -- Figures and Tables}}
\begin{tabular}{@{}p{2.8cm}p{9cm}@{}}
\toprule
\textbf{Item} & \textbf{Code/Data Location} \\
\midrule
All experiments & FVAN/run\_all\_experiments.py \\
\midrule
Figures 6.1--6.2 & FVAN/experiments/exp2\_neuron\_correlation/ \\
Figure 6.3 & FVAN/experiments/exp4\_architecture\_comparison/ \\
Figures 6.4--6.7 & FVAN/utils/, experiments/exp3\_neuron\_replacement/ \\
Figures 6.8--6.11 & FVAN/experiments/exp5\_data\_source\_study/ \\
\midrule
Tables 6.1--6.14 & FVAN/config.py, experiments/exp1--exp5/, utils/ \\
\bottomrule
\end{tabular}
\end{table}

%==============================================================================
\subsection*{Chapter 7: Physics-Constrained Learning}
%==============================================================================

\begin{table}[H]
\centering
\footnotesize
\caption*{\textbf{Chapter 7 -- Figures and Tables}}
\begin{tabular}{@{}p{2.8cm}p{9cm}@{}}
\toprule
\textbf{Item} & \textbf{Code/Data Location} \\
\midrule
PINN experiments & FVAP/run\_pinn\_experiments.py \\
Cross-reference & FVAN/run\_chapter7\_experiments.py \\
\midrule
Figures 7.1--7.5 & FVAP/run\_pinn\_experiments.py, utils/visualization.py \\
Tables 7.1--7.2 & FVAP/run\_pinn\_experiments.py \\
\midrule
Supporting code & FVAP/physics/residuals.py, finite\_differences.py; FVAP/models/pinn\_model.py; FVAN/utils/l1pinn.py \\
\bottomrule
\end{tabular}
\end{table}

%==============================================================================
\subsection*{Chapter 8: Identification of Flow Separation}
%==============================================================================

\begin{table}[H]
\centering
\footnotesize
\caption*{\textbf{Chapter 8 -- Figures and Tables}}
\begin{tabular}{@{}p{2.8cm}p{9cm}@{}}
\toprule
\textbf{Item} & \textbf{Code/Data Location} \\
\midrule
All Ch.\ 8 figures & IFS/generate\_chapter8\_figures.py \\
All experiments & IFS/run\_all\_experiments.py, run\_experiments\_fast.py \\
\midrule
Figures 8.1--8.6 & IFS/generate\_chapter8\_figures.py, utils/classifiers.py \\
Tables 8.1--8.8 & IFS/utils/labeling.py, classifiers.py, feature\_selection.py, physics\_losses.py; config.py; run\_all\_experiments.py \\
\bottomrule
\end{tabular}
\end{table}

%==============================================================================
\subsection*{Chapter 9: OpenFOAM Integration and Comprehensive Evaluation}
%==============================================================================

\begin{table}[H]
\centering
\footnotesize
\caption*{\textbf{Chapter 9 -- Figures and Tables}}
\begin{tabular}{@{}p{2.8cm}p{9cm}@{}}
\toprule
\textbf{Item} & \textbf{Code/Data Location} \\
\midrule
Validation figures & BM/generate\_thesis\_validation\_figures.py \\
Model training & OFI/scripts/train\_models.py, export\_models.py \\
Experiments & OFI/scripts/run\_experiments.py, evaluate\_results.py \\
\midrule
Figures 9.2--9.7 & BM/turbulent/*/openfoam/validate\_*.py \\
Figures 9.8--9.22 & OFI/scripts/evaluate\_results.py, run\_experiments.py \\
Tables 9.1--9.14 & OFI/scripts/; BM/compare\_with\_benchmarks.py \\
\bottomrule
\end{tabular}
\end{table}

%==============================================================================
\subsection*{Chapter 10: Conclusion and Future Work}
%==============================================================================

\begin{table}[H]
\centering
\footnotesize
\caption*{\textbf{Chapter 10 -- Figures and Tables}}
\begin{tabular}{@{}p{2.8cm}p{9cm}@{}}
\toprule
\textbf{Item} & \textbf{Code/Data Location} \\
\midrule
Tables 10.1--10.2 & ADDITIONAL\_STUDIES/run\_all\_experiments.py, compare\_traditional\_wf.py, evaluate\_separation.py \\
\midrule
Supporting exp. & ADDITIONAL\_STUDIES/exp1--exp6/ (wf\_vs\_nowf, attached\_to\_separated, experimental\_augmentation, distribution\_shift, reynolds\_analogy, real\_dns\_experimental) \\
\bottomrule
\end{tabular}
\end{table}

%==============================================================================
\section*{Key Scripts and Entry Points}
%==============================================================================

The following scripts serve as main entry points for reproducing the key results:

\begin{table}[H]
\centering
\footnotesize
\begin{tabular}{@{}p{4cm}p{7.5cm}@{}}
\toprule
\textbf{Script} & \textbf{Purpose} \\
\midrule
\multicolumn{2}{@{}l}{\textit{Data generation (TD/scripts/):}} \\
generate\_cases.py & Generate CFD simulation cases \\
run\_simulations.py & Run all 244 training simulations \\
extract\_and\_pair.py & Extract training data pairs \\
\midrule
\multicolumn{2}{@{}l}{\textit{Chapter-specific experiments:}} \\
FVAI/run\_all\_stages.py & Chapter 4--5 experiments \\
FVAN/run\_all\_experiments.py & Chapter 6 experiments \\
FVAP/run\_pinn\_experiments.py & Chapter 7 PINN experiments \\
IFS/run\_all\_experiments.py & Chapter 8 experiments \\
OFI/scripts/run\_experiments.py & Chapter 9 integration \\
\midrule
\multicolumn{2}{@{}l}{\textit{Figure generation:}} \\
TD/scripts/thesis\_figures/ & Chapter 3 figures \\
FVAI/exp1\_primitive\_only/ & Chapter 5 figures \\
IFS/ & Chapter 8 figures \\
BM/ & Chapter 9 validation figures \\
\bottomrule
\end{tabular}
\end{table}

%==============================================================================
\section*{Data Availability}
%==============================================================================

\subsection*{Training Data}

The complete training dataset comprising 244 simulation cases is available in:
\begin{itemize}[noitemsep]
    \item TD/data/cases\_WF/ -- Wall-modelled (coarse mesh) simulations
    \item TD/data/cases\_WR/ -- Wall-resolved (fine mesh) simulations
\end{itemize}

\noindent Each case directory contains OpenFOAM case files (0/, constant/, system/), converged solution fields (U, p, T, k, omega, nut), post-processed wall quantities (wallShearStress, yPlus), and extracted stencil features (stencil\_features.csv).

\subsection*{Benchmark Data}

Experimental and DNS benchmark data used for validation are organised under BM/:
\begin{itemize}[noitemsep]
    \item turbulent/backward\_facing\_step/ -- Driver \& Seegmiller (1985)
    \item turbulent/periodic\_hills/ -- Breuer et al.\ (2009)
    \item turbulent/wall\_mounted\_hump/ -- NASA wall-mounted hump
    \item turbulent/diffuser/ -- Buice \& Eaton diffuser (1997)
    \item turbulent/channel\_flow/ -- Moser, Kim \& Mansour DNS (1999)
    \item turbulent/flat\_plate/ -- Flat plate boundary layer
    \item turbulent/*\_thermal/ -- Thermal benchmark cases
    \item laminar/ -- Laminar validation (Blasius, Pohlhausen)
    \item transitional/ -- Transitional flow (T3A, T3AM, SK)
\end{itemize}

\subsection*{Trained Models}

Pre-trained neural network models are provided for direct use:
\begin{itemize}[noitemsep]
    \item FVAI/models/ -- Physics-feature models
    \item FVAN/models/ -- Interpretable models
    \item FVAP/models/ -- PINN models
    \item IFS/models/ -- Separation classifiers
    \item OFI/models/ -- Production-ready ONNX models
\end{itemize}

%==============================================================================
\section*{Software Dependencies}
%==============================================================================

\begin{table}[H]
\centering
\footnotesize
\begin{tabular}{@{}lll@{}}
\toprule
\textbf{Software} & \textbf{Version} & \textbf{Purpose} \\
\midrule
OpenFOAM & v10 & CFD simulations \\
Python & 3.9+ & ML training and analysis \\
PyTorch & 2.0+ & Neural network training \\
scikit-learn & 1.0+ & Traditional ML classifiers \\
NumPy & 1.21+ & Numerical computations \\
Matplotlib & 3.5+ & Visualisation \\
LibTorch & 2.0+ & C++ neural network inference \\
ONNX Runtime & 1.12+ & Cross-platform model deployment \\
\bottomrule
\end{tabular}
\end{table}

\section*{Reproducibility}

All experiments can be reproduced using the provided scripts. Random seeds are fixed for reproducibility. Complete instructions are provided in the repository README files.

\vspace{1em}
\noindent\textit{For questions or issues regarding the code, please open an issue on the GitHub repository or contact the author.}

\end{document}
