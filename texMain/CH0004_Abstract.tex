% !TeX root = ThesisMain.tex
% !TeX program = XeLaTeX
% !TeX encoding = UTF-8
% !TeX spellcheck = en_GB

\documentclass[../ThesisMain]{subfiles}
\ifSubfilesClassLoaded{}{}%

\begin{document}
\doublespacing%
\begin{abstract}
Classical algebraic wall functions in computational fluid dynamics fail in separated flows and strong pressure gradients---conditions critical for engineering applications. This thesis develops data-driven universal wall functions that achieve accurate predictions across attached, separated, and transitional flow regimes while maintaining computational efficiency. The methodology uses a dual-mesh training approach where coarse meshes ($y^+ \approx 30$--300) provide local stencil inputs and wall-resolved fine meshes ($y^+ < 1$) supply ground truth. A dataset of 244 simulations spanning diffusers, nozzles, and channels generates 25,485 training samples covering Reynolds numbers from 6,000 to 24,000.

Three complementary strategies are developed: physics-encoded inputs that transform 90 primitive variables into 11 core non-dimensional features (8-fold reduction); physics-guided hidden layers where neurons spontaneously learn representations correlated with established turbulence quantities; and physics-constrained learning incorporating conservation residuals into the training objective. The framework achieves 80\% error reduction in separated flows compared to standard wall functions (8.9\% versus 45\% error), with a separation classifier detecting flow regime from local data at 98.8\% accuracy. Integration into OpenFOAM demonstrates production readiness with 2.1\% computational overhead.

\textbf{Keywords:} Wall functions, Machine learning, Reduced-order modelling, Flow separation, Heat transfer, CFD, OpenFOAM
\end{abstract}
\newpage   
\end{document}