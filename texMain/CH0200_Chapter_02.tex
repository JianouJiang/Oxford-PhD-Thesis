% !TeX root = ThesisMain.tex
% !TeX program = XeLaTeX
% !TeX encoding = UTF-8
% !TeX spellcheck = en_GB

\documentclass[../ThesisMain]{subfiles}
\ifSubfilesClassLoaded{}{}%

\begin{document}
\doublespacing%
\chapter{Literature Review}\label{chap:literature}

This chapter traces the evolution of wall function modeling from its classical foundations through the modern era of machine learning, situating the present thesis within this historical trajectory. Rather than cataloging papers in isolation, we follow the intellectual thread that connects Prandtl's boundary layer theory to contemporary physics-informed neural networks, revealing how each generation of researchers built upon---and was constrained by---the work of their predecessors. The chapter begins with the fundamental physics of turbulent boundary layers, develops the mathematical framework underlying all wall function approaches, examines how industry has implemented these ideas in practical computational fluid dynamics codes, and finally surveys the machine learning revolution that motivates this thesis.

\section{The Physics of Turbulent Boundary Layers}
\label{sec:lit_bl_physics}

Understanding wall functions requires deep familiarity with the physics they attempt to model. The near-wall region of turbulent flows exhibits complex multi-scale phenomena that have occupied fluid dynamicists for over a century \cite{2010_12226_v1, 2312_14902_v1}. This section develops the theoretical foundations from Prandtl's original insights through the modern understanding of near-wall turbulence structure, establishing the physical basis upon which all subsequent wall modeling efforts rest.

\subsection{Prandtl's Boundary Layer Concept}

The modern understanding of wall-bounded flows began with Ludwig Prandtl's seminal 1904 paper, which resolved a fundamental paradox that had troubled fluid mechanics for decades \cite{prandtl1904}. D'Alembert's paradox demonstrated that potential flow theory predicted zero drag on bodies moving through inviscid fluids---a prediction manifestly contradicted by everyday experience. Prandtl's profound insight was that viscous effects, while negligible over most of the flow domain, become dominant in a thin layer adjacent to solid surfaces where the no-slip condition must be satisfied. Within this boundary layer, the velocity changes rapidly from zero at the wall to the free-stream value over a distance that may be orders of magnitude smaller than the characteristic length of the body.

This scale separation enables systematic simplification of the Navier-Stokes equations \cite{2511_14497_v1, 2503_17704_v1}. For steady, two-dimensional flow over a flat plate aligned with the streamwise direction, the boundary layer equations reduce to a coupled system where the streamwise momentum balance involves convection by both velocity components, the imposed pressure gradient from the external flow, and viscous diffusion in the wall-normal direction only. The key simplification arises from recognizing that wall-normal gradients are much larger than streamwise gradients when the boundary layer thickness is small compared to the streamwise development length. This enables neglecting the streamwise viscous diffusion term while retaining the dominant wall-normal diffusion, yielding equations that are parabolic in the streamwise direction and can be marched forward from initial conditions without the elliptic complications of the full Navier-Stokes system.

The boundary layer equations take the form
\begin{align}
    u \frac{\partial u}{\partial x} + v \frac{\partial u}{\partial y} &= -\frac{1}{\rho}\frac{\partial p}{\partial x} + \nu \frac{\partial^2 u}{\partial y^2} \label{eq:ch2_bl_momentum} \\
    \frac{\partial u}{\partial x} + \frac{\partial v}{\partial y} &= 0 \label{eq:bl_continuity}
\end{align}
where the pressure gradient is imposed by the external inviscid flow and varies along the surface. For a flat plate at zero incidence with uniform free-stream velocity, this gradient vanishes, yielding the canonical zero-pressure-gradient boundary layer that serves as the foundation for wall function development \cite{2301_00106_v2, 2105_10889_v1}. Non-zero pressure gradients arise when the external flow accelerates or decelerates, profoundly affecting boundary layer behavior in ways that remain challenging to model even today \cite{2509_05886_v1, 2408_08897_v1}.

\subsection{The Blasius Solution and Laminar Boundary Layers}

For the zero-pressure-gradient laminar boundary layer, Blasius obtained a similarity solution in 1908 that remains a cornerstone of boundary layer theory \cite{blasius1908}. By introducing a stream function and recognizing that the velocity profile maintains a self-similar shape when expressed in terms of an appropriately scaled wall-normal coordinate, Blasius reduced the partial differential equations to an ordinary differential equation. The similarity variable combines the wall distance with the local Reynolds number based on streamwise position, and the resulting third-order nonlinear ordinary differential equation, while not admitting closed-form solution, can be solved numerically to arbitrary precision \cite{2301_00106_v2}.

The Blasius solution yields several important results that inform subsequent turbulent boundary layer analysis. The wall shear stress decreases along the plate as the inverse square root of downstream distance, giving a skin friction coefficient that scales as the inverse square root of the local Reynolds number. The boundary layer thickness grows as the square root of streamwise distance, a consequence of the diffusive spreading of vorticity from the wall. These laminar scaling relationships differ fundamentally from turbulent boundary layers, where enhanced mixing by turbulent fluctuations dramatically increases momentum transfer toward the wall \cite{2006_12483_v1, 1905_03634_v1}.

\subsection{Transition to Turbulence}

Laminar boundary layers become unstable at sufficiently high Reynolds numbers through mechanisms first elucidated by Tollmien and Schlichting in the 1930s \cite{2002_01222_v1, 2105_00913_v2}. Linear stability analysis identifies critical conditions beyond which small disturbances amplify exponentially rather than decaying. These Tollmien-Schlichting waves, essentially two-dimensional oscillations of the boundary layer, grow as they convect downstream, interact nonlinearly, and eventually trigger breakdown to fully turbulent flow through a complex cascade involving three-dimensional secondary instabilities, spike formation, and the generation of turbulent spots that spread laterally and merge.

In practical applications, transition occurs over a range of Reynolds numbers depending on the disturbance environment \cite{2404_03542_v1, 2107_07274_v1}. Free-stream turbulence intensity, surface roughness, acoustic disturbances, and pressure gradients all influence the transition process. For smooth flat plates with low free-stream turbulence, transition typically begins around Reynolds numbers of several hundred thousand based on streamwise distance and completes within a factor of ten in Reynolds number. Higher disturbance levels trigger bypass transition at substantially lower Reynolds numbers through mechanisms that circumvent the classical instability route entirely \cite{2010_12226_v1}. The transitional region presents particular challenges for wall function modeling because neither laminar nor fully turbulent assumptions apply, and the flow exhibits intermittent turbulent spots within a laminar background. This thesis focuses on fully turbulent boundary layers, though the machine learning methods developed could potentially extend to transitional flows with appropriate training data.

\subsection{Reynolds Decomposition and the Closure Problem}

The irregular, chaotic nature of turbulent flows motivated Reynolds to propose decomposing instantaneous velocities into mean and fluctuating components \cite{reynolds1895}. Time-averaging the Navier-Stokes equations with this decomposition yields the Reynolds-Averaged Navier-Stokes equations, which govern the evolution of mean quantities \cite{1701_07102_v2, 2312_14902_v1}. However, the averaging process introduces additional terms---the Reynolds stresses---representing momentum flux due to turbulent fluctuations. These six independent stress components appear as unknowns in the mean momentum equations, yet no additional equations emerge from the averaging process. This closure problem means the Reynolds-averaged equations cannot be solved without supplementary information relating Reynolds stresses to mean flow quantities \cite{2005_09023_v2, 1905_07510_v2}.

The most common closure approach invokes the Boussinesq hypothesis, which assumes Reynolds stresses are proportional to mean strain rates through a turbulent eddy viscosity, analogous to the relationship between viscous stress and strain rate in Newtonian fluids \cite{2206_05226_v2, 2010_12226_v1}. This reduces the closure problem to determining a single scalar field, the eddy viscosity, which turbulence models provide through additional transport equations for quantities like turbulent kinetic energy and its dissipation rate. The Boussinesq hypothesis assumes Reynolds stresses respond instantaneously to local strain rates, ignoring history effects and the inherent anisotropy of turbulence \cite{2210_15384_v1, 2307_13144_v1}. These limitations become severe in flows with strong streamline curvature, rapid strain, or separation---precisely the conditions where wall functions also fail, motivating the more sophisticated approaches developed in this thesis.

\subsection{The Structure of Turbulent Boundary Layers}

Turbulent boundary layers exhibit a characteristic layered structure that emerges from the competition between viscous and turbulent stresses at different distances from the wall \cite{2006_12483_v1, 2411_17095_v1}. This structure, revealed through decades of experimental investigation and more recently through direct numerical simulation, provides the physical basis for wall function modeling. Near the wall, viscous stresses dominate because turbulent fluctuations are damped by the no-slip condition. Far from the wall, turbulent stresses dominate as energetic eddies transport momentum efficiently. Between these limiting behaviors lies a transitional region where both mechanisms contribute comparably \cite{1905_03634_v1, 2002_01222_v1}.

The relevant length scale near the wall is the viscous length, defined as the ratio of kinematic viscosity to friction velocity. Normalizing wall distance by this viscous length defines the wall unit, typically denoted $y^+$. Similarly normalizing the mean velocity by friction velocity defines $u^+$. Within the inner layer, comprising roughly the innermost ten to twenty percent of the boundary layer thickness, the velocity profile depends only on these wall-scaled variables and is independent of the outer flow conditions \cite{moser1999, lee2015}. This universality, known as the law of the wall, represents one of the most robust results in turbulence and provides the theoretical foundation for wall function approaches.

The inner layer subdivides into three distinct regions with different physical character \cite{2202_00435_v1, 2308_04501_v2}. Immediately adjacent to the wall, in what is termed the viscous sublayer, turbulent fluctuations are strongly suppressed and viscous stress dominates completely. The velocity profile becomes linear, with $u^+ = y^+$, a result confirmed by countless experiments and simulations to remarkable precision. This linear region extends to roughly $y^+ = 5$, beyond which turbulent stresses begin contributing significantly.

Between the viscous sublayer and the fully turbulent region lies the buffer layer, extending from approximately $y^+ = 5$ to $y^+ = 30$ \cite{2105_00913_v2, 2006_12483_v1}. In this transitional zone, both viscous and turbulent stresses contribute comparably to momentum transfer, and the velocity profile curves smoothly between the linear viscous behavior and the logarithmic profile that emerges at larger wall distances. The buffer layer contains the peak of turbulent kinetic energy production and hosts the most intense turbulent events, including the ejection of low-speed fluid away from the wall and the sweep of high-speed fluid toward it. No simple analytical expression describes this region accurately, though various empirical formulas have been proposed for engineering calculations \cite{spalding1961}.

Beyond the buffer layer, for $y^+ > 30$ but still within roughly twenty percent of the boundary layer thickness, turbulent stresses dominate completely and a logarithmic velocity profile emerges. This logarithmic law of the wall, first proposed by von K\'{a}rm\'{a}n based on dimensional arguments \cite{karman1930}, takes the form $u^+ = (1/\kappa) \ln(y^+) + B$, where $\kappa \approx 0.41$ is the von K\'{a}rm\'{a}n constant and $B \approx 5.0$ is an additive constant determined by matching through the buffer layer. The logarithmic profile can be derived from several independent lines of reasoning, lending confidence to its universality \cite{2301_00106_v2}.

\subsection{Derivations of the Logarithmic Law}

Dimensional analysis provides the most direct route to the logarithmic profile \cite{2511_14497_v1, 2105_10889_v1}. In the fully turbulent region where viscosity no longer directly influences the local dynamics, the velocity gradient can depend only on wall distance, wall shear stress, and fluid density. Dimensional consistency then requires the velocity gradient to scale inversely with wall distance, and integration yields the logarithmic profile with an undetermined multiplicative constant that experiments identify as the inverse of the von K\'{a}rm\'{a}n constant.

Prandtl's mixing length hypothesis offers a physical interpretation of this scaling \cite{2212_08989_v3, 2509_20683_v1}. Turbulent eddies transport momentum over a characteristic mixing length before losing their identity through mixing with surrounding fluid. Near the wall, the largest eddies that can exist are constrained by the wall distance itself, suggesting the mixing length should be proportional to $y$. This assumption, combined with the definition of turbulent stress through the mixing length formulation, yields an eddy viscosity proportional to wall distance times the velocity gradient. When the total stress is constant and equal to the wall shear stress---a good approximation in the inner layer---solving for the velocity gradient and integrating reproduces the logarithmic law.

Millikan's overlap argument provides yet another derivation without assuming anything about the turbulent stress mechanism \cite{millikan1938}. In the inner layer, the velocity profile must depend only on wall-scaled variables. In the outer layer, the velocity defect from the free-stream value must depend only on outer-scaled variables involving the boundary layer thickness. In the overlap region where both descriptions must simultaneously apply, the only functional form consistent with both scalings is logarithmic. This elegant argument demonstrates that the log law is not merely an empirical fit but a mathematical consequence of the two-layer structure of the boundary layer.

\subsection{The Outer Layer and Wake Function}

Beyond the logarithmic region, the velocity continues increasing toward the free-stream value through what is termed the wake region or defect layer \cite{coles1956, 2411_17095_v1}. Here the velocity defect from the free-stream value scales with the friction velocity and the boundary layer thickness rather than with viscous quantities. Coles proposed combining the inner logarithmic profile with an additive wake function to describe the entire boundary layer. The wake function accounts for the departure from the log law in the outer region due to intermittency effects at the boundary layer edge and the influence of the outer boundary conditions.

The wake parameter quantifies the strength of the wake component and varies with pressure gradient \cite{clauser1954, 2408_08897_v1}. For zero-pressure-gradient boundary layers, the wake parameter takes a value around 0.55, corresponding to a modest overshoot above the logarithmic extrapolation in the outer region. Adverse pressure gradients increase the wake parameter, strengthening the deviation from the log law, while favorable pressure gradients decrease it. This pressure-gradient sensitivity of the outer layer contrasts with the relative universality of the inner layer and illustrates why wall functions based solely on the log law struggle under non-equilibrium conditions.

\subsection{Turbulent Stress Distributions}

The Reynolds stress tensor exhibits distinct behavior across the boundary layer structure, behavior that must be captured or circumvented by wall function approaches \cite{moser1999, lee2015}. The landmark direct numerical simulation of Moser, Kim, and Mansour provided definitive data on these distributions in turbulent channel flow, later extended to higher Reynolds numbers by Lee and Moser \cite{2006_12483_v1, 2002_01222_v1}.

The streamwise velocity fluctuations peak in the buffer layer around $y^+ \approx 15$, reaching root-mean-square values roughly 2.5 to 3 times the friction velocity \cite{2509_20683_v1, 2411_17095_v1}. This intense streamwise fluctuation activity reflects the burst-sweep cycle that dominates near-wall turbulence, with low-speed fluid ejected away from the wall and high-speed fluid swept toward it. The wall-normal fluctuations, in contrast, are strongly suppressed near the wall by the kinematic blocking effect of the surface and increase more gradually, peaking in the outer part of the logarithmic layer. The Reynolds shear stress, the only component contributing to mean momentum transfer in parallel shear flows, varies nearly linearly across the inner layer from zero at the wall to a maximum in the outer layer, reflecting the handoff from viscous to turbulent momentum transport.

Understanding these stress distributions is essential for wall function development \cite{2202_00435_v1, 2308_04501_v2}. Standard wall functions assume local equilibrium between production and dissipation of turbulent kinetic energy, which holds approximately in the logarithmic layer where the production rate nearly balances the dissipation rate. In the buffer layer, production exceeds dissipation as turbulent kinetic energy is generated before being transported and dissipated elsewhere. Under non-equilibrium conditions such as adverse pressure gradients or flow separation, the equilibrium assumption fails throughout the boundary layer, invalidating the fundamental premise of classical wall functions \cite{2210_15384_v1, 2409_04143_v1}.

\section{The Mathematical Framework of Wall Functions}
\label{sec:lit_math_wf}

The theoretical understanding developed above must be translated into practical computational tools. This section traces the evolution of wall function formulations from early analytical approaches through modern enhanced treatments implemented in commercial software \cite{launder1974, spalding1961}.

\subsection{The Launder-Spalding Wall Function}

Launder and Spalding formalized the use of the logarithmic law as a boundary condition for Reynolds-averaged simulations in their influential 1974 paper \cite{launder1974}. Their approach places the first computational cell center in the logarithmic region, typically at $y^+$ values between 30 and 300, and relates the cell-center velocity to wall shear stress algebraically through the log law. This eliminates the need to resolve the steep gradients in the viscous sublayer and buffer layer, reducing mesh requirements by an order of magnitude or more while maintaining accuracy for flows where the logarithmic profile applies \cite{2309_02109_v1, 2404_03542_v1}.

The implementation requires care because the wall shear stress appears on both sides of the log law equation---explicitly in the relationship and implicitly through the wall-scaled variables \cite{2202_04233_v3}. Launder and Spalding resolved this by using the cell-center turbulent kinetic energy rather than the friction velocity to define the wall units, yielding an explicit formula for wall shear stress given the cell-center velocity and turbulence quantities. The turbulence boundary conditions similarly employ equilibrium assumptions, specifying that the dissipation rate in the wall-adjacent cell equals the production rate according to local equilibrium.

The thermal wall function follows analogously, using Reynolds analogy to relate the turbulent heat flux to the momentum flux through a turbulent Prandtl number \cite{2201_03200_v2, 2202_00435_v1}. The temperature profile in wall units exhibits a linear conduction-dominated region near the wall analogous to the viscous sublayer, followed by a logarithmic region where turbulent transport dominates. The transition between these regions depends on the molecular Prandtl number, with high-Prandtl-number fluids having extremely thin conduction sublayers and low-Prandtl-number fluids like liquid metals having conduction extending well into the turbulent region.

\subsection{Assumptions and Their Limitations}

The Launder-Spalding wall function rests on several assumptions whose violation causes the approach to fail \cite{2409_04143_v1, 2307_13144_v1}. The assumption that the first cell lies within the logarithmic region means that if the mesh is too coarse the cell extends into the wake region where the log law overpredicts velocity, and if too fine the cell falls into the buffer layer where it underpredicts. The local equilibrium assumption means the approach cannot capture the history effects that characterize boundary layers under pressure gradients or after perturbations. The constant-stress assumption requires the wall-adjacent cell to be thin enough that stress variation across it is negligible.

The most severe failures occur in separated flows where the wall shear stress becomes negative \cite{2309_02109_v1, 2408_08897_v1, 2411_17095_v1}. The friction velocity, defined as the square root of the wall shear stress magnitude divided by density, remains real, but the entire scaling framework loses physical meaning when the near-wall velocity opposes the outer flow direction. The log law, derived for attached flows where velocity increases monotonically from wall to free-stream, has no relevance to the reversed-flow profiles near separation. Most computational codes handle separated regions by ad-hoc fixes such as limiting the wall shear stress magnitude or switching to alternative formulations, but these patches have no physical basis and typically produce poor predictions of separation location, recirculation zone extent, and reattachment \cite{2511_18552_v1, 2509_05886_v1}.

Strong adverse pressure gradients, even without separation, cause the equilibrium assumption to fail \cite{clauser1954, 2206_05226_v2}. As the flow decelerates, the boundary layer thickens, the velocity profile distorts from logarithmic form, and the production of turbulent kinetic energy exceeds its dissipation rate. The classical wall function, seeing only the cell-center conditions and knowing nothing of the upstream history, overpredicts wall shear stress because it assumes an equilibrium profile that the actual flow does not possess.

\subsection{Enhanced Wall Treatments}

Recognition of these limitations motivated development of enhanced wall treatments that attempt to function across a wider range of conditions \cite{wolfshtein1969, 2309_02109_v1}. Two-layer models resolve the viscous sublayer with a fine mesh while using a simplified turbulence model in the near-wall region. These models can handle lower wall-unit mesh spacing but still assume quasi-equilibrium and struggle with separated flows.

Enhanced wall functions blend between viscous sublayer and logarithmic formulations based on local wall-unit spacing, allowing the first cell to be placed anywhere from the sublayer to the log layer without dramatic loss of accuracy \cite{spalding1961, 2202_04233_v3}. Spalding's unified wall law provides a single implicit equation valid across the entire inner layer, though it cannot be solved explicitly for velocity given wall distance. Modern commercial codes implement various blending functions that transition smoothly between the linear and logarithmic regimes.

Scalable wall functions address the problem of meshes refined beyond the buffer layer by defining a virtual wall location at the intersection of the linear and logarithmic profiles \cite{2404_03542_v1}. This prevents the wall function from predicting unphysically low wall shear stress on fine meshes, but it sacrifices the potential accuracy gains that fine mesh resolution near the wall might provide.

Non-equilibrium wall functions attempt to account for pressure gradient effects by modifying the log-law constants or by solving simplified boundary layer equations within the wall function formulation \cite{2409_04143_v1, 2509_05886_v1}. The Launder-Shima approach adds a pressure gradient correction to the log-law intercept, partially compensating for the profile distortion under adverse gradients. More sophisticated analytical wall functions solve the log-layer momentum equation including convection and pressure gradient effects, but even these struggle with the extreme non-equilibrium of separated flows.

\section{Thermal Boundary Layers and Heat Transfer}
\label{sec:lit_thermal}

Wall functions must predict not only momentum transfer through wall shear stress but also heat transfer through wall heat flux for thermal applications \cite{2201_03200_v2, 2202_00435_v1, 1910_03097_v1}. The thermal boundary layer exhibits analogous layered structure to the velocity boundary layer but with important differences that complicate the extension of momentum wall functions to heat transfer.

\subsection{The Reynolds Analogy}

Reynolds observed in 1874 that the mechanisms of momentum and heat transfer in turbulent flow share fundamental similarities, leading to the classical Reynolds analogy relating skin friction coefficient to Stanton number \cite{2201_03200_v2, 2211_00601_v1}. For fluids with Prandtl number equal to unity, where molecular diffusivities of momentum and heat are identical, the analogy predicts exact equality between half the skin friction coefficient and the Stanton number. For fluids with Prandtl numbers differing from unity, modifications account for the different sublayer thicknesses, with the Colburn analogy introducing a Prandtl number correction factor.

These analogies work reasonably well for attached boundary layers with moderate temperature differences but fail when pressure gradients cause the velocity and thermal boundary layers to develop differently \cite{1910_03097_v1, 2202_00435_v1}, when strong temperature differences cause property variations that invalidate the constant-property assumptions, or when buoyancy effects couple the momentum and energy equations. The analogy has no physical basis whatsoever in separated regions where the reversed near-wall flow bears no relationship to the heat transfer at the wall \cite{2502_05577_v2, 2309_15294_v2}.

\subsection{Thermal Law of the Wall}

By analogy with the velocity profile, the temperature field in the inner layer follows a universal law when expressed in appropriate wall units \cite{kader1981, 2201_03200_v2}. Defining a friction temperature through the wall heat flux, analogous to the friction velocity definition, allows expressing the temperature difference from the wall in dimensionless form. In the conduction-dominated sublayer immediately adjacent to the wall, the temperature profile becomes linear with slope equal to the molecular Prandtl number. In the turbulent region, a logarithmic profile emerges with slope related to the turbulent Prandtl number.

Kader developed a widely-used formula that blends smoothly between these regimes while capturing the Prandtl number dependence of the thermal buffer layer \cite{kader1981}. The blending function differs from that for velocity because the thermal and momentum buffer layers have different structures depending on the molecular Prandtl number \cite{2502_05577_v2}. For high-Prandtl-number fluids like oils, the thermal sublayer is extremely thin compared to the viscous sublayer, while for low-Prandtl-number fluids like liquid metals, conduction penetrates far into the turbulent region.

\subsection{Dissimilarity Between Momentum and Heat Transfer}

A fundamental limitation of Reynolds-analogy-based thermal wall functions is the assumption that momentum and heat transfer follow identical mechanisms \cite{2211_00601_v1, 1910_03097_v1}. Several sources of dissimilarity exist even in attached boundary layers. The turbulent Prandtl number, relating turbulent momentum and heat diffusivities, is not constant but varies from roughly unity in the sublayer to 0.85 in the logarithmic layer to perhaps 0.5 in the outer region. Pressure gradients affect velocity and temperature profiles differently, and buoyancy introduces coupling between the fields that the analogy cannot capture \cite{2502_05577_v2}.

These considerations motivate the approach of this thesis, which predicts wall shear stress and wall heat flux simultaneously from unified models trained on data that naturally captures the dissimilarity between momentum and heat transfer \cite{2201_03200_v2, 2202_00435_v1}. Rather than assuming the two are linked by a constant factor, the neural network learns whatever relationship the training data exhibits.

\section{Industrial Computational Fluid Dynamics Practice}
\label{sec:lit_industrial}

The theoretical developments traced above must ultimately serve practical engineering analysis \cite{2209_02051_v1, 2212_08989_v3}. This section examines how wall functions are implemented in commercial codes and applied in industrial workflows, identifying the gap between academic research and production practice that motivates practical deployability as a key objective of this thesis.

\subsection{Commercial Implementations}

Major commercial computational fluid dynamics codes implement wall functions with varying degrees of sophistication \cite{2404_03542_v1, 2309_02109_v1}. ANSYS Fluent offers multiple options including standard wall functions, scalable wall functions, enhanced wall treatment, and non-equilibrium wall functions. The enhanced wall treatment automatically blends between viscous sublayer resolution and log-law formulations based on local mesh resolution, providing flexibility at the cost of potential inconsistency between regions. STAR-CCM+ implements all-$y^+$ wall treatment that switches between low-Reynolds and high-Reynolds formulations automatically, emphasizing robustness for industrial users who may not optimize mesh resolution for each application.

OpenFOAM, the open-source platform used throughout this thesis, provides wall function boundary conditions implementing the Launder-Spalding formulation along with automatic wall treatment options using Spalding's unified law \cite{2409_19851_v1, 2404_03542_v1}. The open-source nature allows direct implementation of new wall function approaches, making it the natural choice for developing and validating the machine learning methods of this thesis.

\subsection{Industrial Practice and Certification}

Industrial computational fluid dynamics practice involves compromises between accuracy, computational cost, and robustness \cite{2212_08989_v3, 2012_10165_v1}. Mesh guidelines typically recommend wall-unit spacing in the range 30 to 300 for standard wall functions and near unity for wall-resolved simulations, but achieving these targets uniformly across complex geometries is often impractical. Real industrial meshes frequently have non-uniform wall spacing that places some regions in the buffer layer where neither wall function formulation is optimal.

High-stakes industries impose stringent validation requirements that often exceed wall function capabilities \cite{2502_05577_v2, 2509_05886_v1}. Aerospace certification may require wall-resolved simulations for flight-critical components because wall function predictions are deemed insufficiently reliable. Nuclear safety analysis requires validated methodologies with quantified uncertainty margins that wall function limitations can expand substantially \cite{2310_11435_v1, 2511_05633_v1, 2202_01560_v2}. These requirements highlight the gap between wall function capabilities and industrial needs that motivates continued research including the machine learning approaches developed here.

\subsection{The Academia-Industry Gap}

A persistent gap exists between academic turbulence modeling research and industrial practice \cite{2010_12226_v1, 2012_10165_v1}. Academic developments often include many adjustable parameters tuned for specific test cases, whereas industrial users need robust defaults that work across diverse applications without case-by-case adjustment. Academic publications report successful cases, but industrial simulations must handle arbitrary geometries and flow conditions without divergence or unphysical results \cite{2206_05226_v2, 2307_13144_v1}. Advanced models that improve accuracy modestly while substantially increasing computational cost are rarely adopted industrially, where throughput considerations often override accuracy concerns.

This thesis addresses several of these gaps. By implementing models in OpenFOAM and providing complete documentation, we lower barriers to industrial adoption \cite{2409_19851_v1, 2404_03542_v1}. By training on diverse configurations and validating on geometries outside the training distribution, we demonstrate the robustness industrial users require. By maintaining computational efficiency comparable to standard wall functions, we avoid the cost barriers that have limited adoption of more sophisticated approaches.

\section{Pressure Gradient Effects}
\label{sec:lit_pressure_gradient}

Pressure gradients profoundly affect turbulent boundary layer behavior and represent a key challenge for wall function modeling \cite{clauser1954, 2408_08897_v1, 2509_05886_v1}. The Clauser parameter, defined as the product of displacement thickness and pressure gradient divided by wall shear stress, quantifies pressure gradient strength relative to wall friction. Zero values correspond to zero-pressure-gradient boundary layers, negative values to favorable (accelerating) gradients, and positive values to adverse (decelerating) gradients. Self-similar equilibrium boundary layers can exist at constant Clauser parameter values, maintaining fixed shape while growing downstream, but most practical flows involve spatially-varying pressure gradients that prevent self-similarity \cite{2409_04143_v1, 2411_17095_v1}.

Under adverse pressure gradients, the boundary layer thickens as retarded near-wall fluid decelerates further, the velocity profile distorts from logarithmic form with increased wake component, turbulence intensifies due to enhanced production, and wall shear stress decreases toward zero \cite{2509_05886_v1, 2511_18552_v1}. If the adverse gradient is sufficiently strong, flow separation occurs when wall shear stress vanishes and the near-wall fluid reverses direction. The approach to separation involves highly non-equilibrium dynamics that classical wall functions cannot capture.

Stratford developed a criterion for predicting turbulent boundary layer separation based on the pressure rise and streamwise development length \cite{stratford1959}. While useful for preliminary design, such criteria cannot predict the detailed behavior through and after separation \cite{2411_17095_v1, 2312_03295_v2}. The machine learning approach of this thesis provides predictions throughout the separation and reattachment process by learning from training data that includes these regimes.

\section{The Machine Learning Revolution}
\label{sec:lit_ml_revolution}

Having established the classical physics and computational frameworks, we turn to the modern machine learning approaches that motivate this thesis. The emergence of data-driven turbulence modeling represents not merely an incremental improvement but a fundamental shift in how the field approaches the closure problem and the limitations of classical models \cite{2010_12226_v1, 2312_14902_v1, 1701_07102_v2}.

\subsection{Data-Driven Turbulence Modeling}

The modern era of machine learning in computational fluid dynamics emerged from advances in deep learning for image recognition and natural language processing that demonstrated neural networks could learn complex nonlinear mappings from large datasets \cite{ling2016, 1701_07102_v2, 2005_09023_v2}. The seminal work of Ling, Kurzawski, and Templeton marked a watershed moment by demonstrating that deep neural networks could predict Reynolds stress anisotropy from mean flow features, capturing physics beyond the Boussinesq hypothesis that had constrained turbulence modeling for decades.

Ling's approach incorporated physics awareness through its input construction \cite{1905_07510_v2, 2210_15384_v1}. Rather than providing raw flow quantities, she used invariants of the strain rate and rotation tensors as inputs, encoding Galilean invariance---the requirement that turbulence statistics cannot depend on observer frame of reference---directly into the network architecture. The outputs similarly respected tensorial structure through a basis expansion. This physics-informed approach yielded better generalization than purely black-box models, establishing a paradigm that subsequent work has extended \cite{2307_13144_v1, 2312_14902_v1}.

Wang, Wu, and Xiao developed comprehensive frameworks combining machine learning with uncertainty quantification, emphasizing that neural networks trained on high-fidelity data encode implicit knowledge of turbulent dynamics but that this knowledge may not transfer reliably to flows outside the training distribution \cite{1701_07102_v2, 2012_10165_v1, 2206_05226_v2}. The generalization problem---ensuring models trained on canonical configurations perform well in practical applications---emerged as the central challenge for the field and remains largely unsolved \cite{2307_13144_v1, 2210_15384_v1, 2409_04143_v1}.

Subsequent research has explored numerous variations on data-driven turbulence modeling \cite{2312_14902_v1, 2010_12226_v1}. Sparse symbolic regression discovers algebraic Reynolds stress models automatically from data \cite{1905_07510_v2, 2312_03295_v2}. Multi-agent reinforcement learning optimizes turbulence model parameters through automated experimentation \cite{2005_09023_v2}. Ensemble methods quantify uncertainty in learned closures \cite{2306_13370_v2, 2310_11435_v1, 2511_05633_v1}. Each approach offers unique advantages, but the fundamental challenge of generalizing from training data to novel flows persists across methodologies \cite{2206_05226_v2, 2409_04143_v1}.

\subsection{Physics-Informed Neural Networks}

Raissi, Perdikaris, and Karniadakis introduced Physics-Informed Neural Networks (PINNs), addressing the fundamental limitation that traditional supervised learning requires abundant labeled training data \cite{raissi2019physics, 2511_14497_v1, 2503_17704_v1}. Their approach incorporates governing equations as soft constraints during training by adding physics residuals to the loss function. If the network learns to satisfy conservation laws throughout the domain, its predictions should remain physically consistent even in regions without training data \cite{2409_19851_v1, 2408_08897_v1, 2301_00106_v2}.

The physics loss evaluates partial differential equation residuals at collocation points using automatic differentiation through the network, penalizing departures from the governing equations \cite{2511_14497_v1, 2105_10889_v1, 2404_03542_v1}. For incompressible flows, this includes continuity and momentum residuals throughout the domain. The balance between data fitting and physics constraint satisfaction, governed by a hyperparameter weighting the physics loss, requires careful tuning that varies across problems \cite{2505_00343_v1, 2406_00471_v1, 2309_15294_v2}.

Physics-informed networks have been applied to numerous fluid dynamics problems \cite{2205_08663_v2, 2309_15294_v2, 2404_03542_v1, 2501_00014_v1}. Wang and colleagues applied PINNs to turbulence modeling with tensor basis constraints \cite{2312_13005_v3, 1701_07102_v2}. Applications span biomedical flows \cite{2309_15294_v2, 2501_00014_v1}, combustion \cite{2111_00328_v2, 2505_00343_v1}, and free shear flows \cite{2404_03542_v1}. OpenFOAM integration has been demonstrated for practical CFD workflows \cite{2409_19851_v1}.

Physics-informed networks face substantial challenges for turbulence applications \cite{2206_05226_v2, 2505_00343_v1, 2307_13784_v1}. The Reynolds-averaged equations are not closed, requiring models for Reynolds stresses that cannot be derived from first principles. Neural networks exhibit spectral bias toward learning low-frequency features before high-frequency ones, making multi-scale turbulence difficult to capture \cite{2406_00471_v1, 2509_05886_v1}. The physics loss can create ill-conditioned optimization landscapes near boundaries where multiple constraints interact \cite{2505_00343_v1, 2503_17704_v1}.

Chapter~\ref{chap:pinn} of this thesis develops a local stencil-based variant that addresses these limitations for wall function applications. Rather than enforcing global conservation over entire domains, we evaluate physics residuals on the same local stencil used for input features, maintaining computational efficiency while focusing physical constraints where they matter for wall shear and heat flux prediction.

\subsection{Deep Learning Architectures for Turbulence}

Beyond physics-informed approaches, various deep learning architectures have been applied to turbulence problems \cite{2006_12483_v1, 2002_01222_v1, 1905_03634_v1, 1711_09846_v2, 2103_09389_v2}. Convolutional neural networks exploit spatial structure in flow fields \cite{2006_12483_v1, 2211_16845_v2, 2509_20683_v1, 2107_01750_v2, 2108_11985_v1}, while recurrent networks capture temporal dependencies in unsteady flows \cite{2002_01222_v1, 2502_05577_v2, 1906_04029_v3, 2109_09363_v2}. Attention mechanisms allow networks to focus on relevant regions \cite{2105_00913_v2, 2502_05577_v2, 2110_07510_v3, 2203_02498_v1}, and transformer architectures originally developed for natural language processing show promise for sequence modeling in turbulence \cite{2509_20683_v1, 2411_17095_v1, 2204_03911_v2, 2304_10717_v2}.

Graph neural networks offer particular promise for CFD applications due to their ability to handle unstructured meshes naturally \cite{2406_10534_v3, 2206_04979_v4, 2202_04233_v3, 2303_00836_v2, 2311_05128_v2}. By representing mesh cells as nodes and connectivity as edges, graph networks can process arbitrary mesh topologies without the structured grid requirements of convolutional networks \cite{2403_06418_v2, 2405_09256_v1}. This flexibility enables application to complex geometries without the mesh interpolation that convolutional approaches require \cite{2406_10534_v3, 2501_00014_v1, 2407_20801_v1, 2408_06486_v1}.

Super-resolution techniques reconstruct fine-scale turbulent structures from coarse representations \cite{2509_20683_v1, 2211_16845_v2, 2411_17095_v1, 2302_08780_v3, 2311_14464_v1}. Neural operators learn mappings between function spaces \cite{2108_11985_v1, 2501_00014_v1, 2301_09048_v1, 2304_11247_v3}, potentially enabling generalization across geometries and operating conditions \cite{2503_17704_v1, 2402_18236_v1, 2403_11746_v1}. Diffusion models generate realistic turbulent fields through iterative denoising \cite{2407_02519_v1, 2411_17095_v1, 2308_07358_v1, 2309_06010_v1}. Each architecture class offers distinct advantages for specific aspects of the turbulence modeling challenge \cite{2401_09932_v1, 2402_03153_v1}.

\subsection{Machine Learning Wall Functions}

The specific application of machine learning to wall function modeling represents the confluence of classical wall-bounded turbulence theory, data-driven Reynolds stress modeling, and physics-informed learning \cite{2409_04143_v1, 2308_04501_v2, 2006_12483_v1, 2103_03115_v3, 2105_08633_v6}. Milano and Koumoutsakos pioneered this direction in 2002, training neural networks to predict wall shear stress from outer flow quantities for large eddy simulation \cite{milano2002, 2010_10491_v2, 2011_04157_v1}. Limited by contemporary computational resources, their work demonstrated proof of concept but could not address generalization challenges \cite{2101_02535_v2, 2102_03767_v4}.

Recent work has demonstrated that neural networks can handle separated flows over periodic hills where standard wall functions fail entirely \cite{2409_04143_v1, 2312_03295_v2, 2411_17095_v1, 2104_08249_v2, 2201_06628_v4}, that training on canonical building-block flows can construct composite models for more complex geometries \cite{2206_05226_v2, 2307_13144_v1, 2209_02977_v2, 2210_04193_v2}, and that physics-informed variants improve consistency at the cost of some fitting accuracy \cite{2505_00343_v1, 2406_00471_v1, 2212_05023_v2, 2305_10043_v2}. Deep neural networks have been applied specifically to near-wall turbulence with consideration of wall-distance effects \cite{2409_04143_v1, 1905_03634_v1, 2006_12483_v1, 2106_09512_v1, 2107_07274_v1}. Data-driven approaches for separated flows have shown particular promise \cite{2312_03295_v2, 2511_18552_v1, 2509_05886_v1, 2208_14301_v1, 2210_07094_v2}.

Yet critical questions remain \cite{2206_05226_v2, 2307_13144_v1}. The choice between primitive variables and physics-based features as network inputs varies across studies without systematic justification---a gap Chapter~\ref{chap:physics_features} addresses. Thermal wall functions receive disproportionately little attention despite engineering importance \cite{2201_03200_v2, 2202_00435_v1, 1910_03097_v1}---a gap this thesis addresses by predicting both momentum and thermal quantities jointly. The generalization from training geometries to practical applications remains the central unsolved challenge \cite{2206_05226_v2, 2409_04143_v1}.

\subsection{Uncertainty Quantification in Machine Learning CFD}

Reliable deployment of machine learning models requires quantifying prediction uncertainty \cite{2310_11435_v1, 2511_05633_v1, 2202_01560_v2, 2306_13370_v2, 2112_02751_v2, 2203_16394_v1}. Bayesian approaches provide principled uncertainty estimates through posterior distributions over network weights \cite{1804_01065_v2, 2012_10165_v1, 2006_02979_v2, 2104_06217_v1}. Deep ensembles use disagreement among independently trained networks as an uncertainty proxy \cite{2306_13370_v2, 2310_11435_v1, 2205_04739_v2, 2311_18027_v1}. Dropout during inference approximates Bayesian uncertainty without the computational cost of full posterior inference \cite{2206_05226_v2, 1908_00294_v1, 2003_01968_v2}.

Physics-constrained random forests provide uncertainty quantification with interpretable decision boundaries \cite{2306_13370_v2, 2101_10528_v1, 2201_01287_v2}. Mondrian forests offer calibrated uncertainty estimates for turbulence modeling \cite{2007_03898_v3, 2004_03783_v1, 2005_02599_v1}. Machine learning uncertainty has been applied to airfoil predictions \cite{2211_03665_v1, 2402_08037_v1, 2406_17446_v2}, turbulence model selection \cite{2210_16358_v2, 2212_00332_v2, 2302_01802_v2}, and Reynolds stress modeling \cite{2012_10165_v1, 2202_01560_v2, 2303_17178_v2, 2403_20295_v2}.

For wall functions specifically, uncertainty quantification enables appropriate selection between ML and traditional approaches based on prediction confidence \cite{2310_11435_v1, 2511_05633_v1}. Regions where the ML model is uncertain can fall back to validated classical methods, combining the accuracy of data-driven approaches where they are confident with the reliability of physics-based models elsewhere. This thesis does not develop uncertainty quantification methods but identifies this as an important direction for future work.

\subsection{Transfer Learning and Domain Adaptation}

A critical challenge for data-driven wall functions is transferring knowledge from training configurations to novel applications \cite{2206_05226_v2, 2307_13144_v1, 2409_04143_v1, 2409_03395_v1, 2410_01657_v1}. Transfer learning leverages features learned on one task to improve performance on related tasks \cite{1711_09846_v2, 2407_02519_v1, 2501_08738_v3, 2502_16245_v1}. Domain adaptation addresses distribution shift between training and deployment conditions \cite{2409_04143_v1, 2509_05886_v1, 2503_02482_v2, 2504_07837_v2}.

For turbulence modeling, transfer learning has been applied to adapt models trained on canonical flows to complex geometries \cite{2206_05226_v2, 2409_04143_v1, 2504_04982_v2, 2505_07964_v1}. The physics-based features developed in Chapter~\ref{chap:physics_features} inherently provide transfer learning benefits by encoding scale-invariant relationships that apply across Reynolds numbers and geometries \cite{1905_07510_v2, 2210_15384_v1, 2506_08516_v1, 2507_08986_v2}. The architecture-invariant features identified in Chapter~\ref{chap:neurons} suggest which representations transfer most reliably \cite{2508_01537_v1, 2509_01963_v1}.

\section{Benchmark Cases and Validation}
\label{sec:lit_benchmarks}

The development and validation of wall functions requires high-fidelity reference data spanning conditions from simple equilibrium to complex separation \cite{moser1999, lee2015, driver1985, breuer2009, 2509_06041_v1, 2510_01091_v1}. The turbulence modeling community has established canonical benchmark cases that serve this purpose \cite{2510_07106_v1, 2511_01766_v1}.

Fully-developed channel flow between parallel plates represents the simplest configuration, with zero pressure gradient, statistical stationarity, and clean validation of the law of the wall \cite{moser1999, lee2015, 2006_12483_v1}. The direct numerical simulation databases of Moser, Kim, and Mansour at friction Reynolds numbers up to 590 remain definitive references, later extended by Lee and Moser to friction Reynolds numbers exceeding 5000. These databases provide mean velocity profiles, Reynolds stress components, and turbulent kinetic energy budgets against which wall function predictions can be quantitatively assessed.

The zero-pressure-gradient flat plate boundary layer adds spatial development to the problem \cite{schlatter2010, 2002_01222_v1, 2301_00106_v2}. Direct numerical simulation data from Schlatter and \"{O}rl\"{u} provide validation through the transition and early turbulent regimes. Standard wall functions perform well for this canonical configuration, establishing baseline performance that more complex flows challenge.

Diffuser flows introduce adverse pressure gradients of controllable severity \cite{2409_04143_v1, 2312_03295_v2, 2509_05886_v1}. By varying expansion angle and area ratio, diffusers span conditions from mild deceleration with attached flow through incipient separation to massive recirculation. The asymmetric diffuser configuration used throughout this thesis provides systematic variation of pressure gradient while maintaining a flat wall for clean data extraction.

The backward-facing step represents an extreme test case where sudden expansion creates massive separation with reattachment several step heights downstream \cite{driver1985, 2411_17095_v1, 2511_18552_v1}. The experiments of Driver and Seegmiller provide detailed validation data. Standard wall functions fail catastrophically in the recirculation zone, making the backward-facing step a stringent test for any improved approach.

Periodic hills add surface curvature to the separation challenge \cite{breuer2009, 2312_03295_v2, 2409_04143_v1}. The configuration of Breuer and colleagues provides cyclic separation and reattachment over curved surfaces with direct numerical simulation validation. The periodic boundary conditions eliminate inlet specification issues while creating a challenging test of wall function performance over curved separating surfaces.

Chapter~\ref{chap:methodology} describes how training data from diffuser, nozzle, and channel configurations---244 cases spanning Reynolds numbers 8,000 to 24,000 and expansion ratios from 0.5 to 5.5---provides the foundation for the machine learning models developed in subsequent chapters. Chapter~\ref{chap:openfoam} returns to these benchmark cases for validation, testing model generalization from training geometries to backward-facing steps and periodic hills.

\section{Related Applications and Extensions}
\label{sec:lit_applications}

Machine learning approaches to near-wall modeling connect to several related application domains that inform and are informed by wall function development \cite{2209_02051_v1, 2212_08989_v3, 2511_09847_v2, 2512_04452_v1}.

\subsection{Large Eddy Simulation Subgrid Modeling}

Large eddy simulation requires modeling the effect of unresolved subgrid scales on the resolved flow \cite{2010_10491_v2, 2204_03911_v2, 2411_17095_v1, 2601_03613_v1, 2601_06506_v1}. Machine learning approaches to subgrid modeling share many challenges with wall function development: both must represent unresolved physics through modeled terms that depend on resolved quantities \cite{2404_09074_v1, 2307_13784_v1, 2601_11946_v1}. Deep learning subgrid models have demonstrated improved accuracy over classical approaches \cite{2010_10491_v2, 2411_17095_v1, 2412_08460_v2}, though stability in coupled simulations remains challenging \cite{2404_09074_v1, 2409_11899_v1}.

The wall model for LES (WMLES) approach combines features of wall functions and subgrid models \cite{2409_04143_v1, 2006_12483_v1}. Machine learning wall models for LES must capture the effect of unresolved near-wall turbulence on the resolved outer flow, a more demanding task than RANS wall functions because the instantaneous fluctuations carry physical information that time-averaged quantities lack \cite{2006_12483_v1, 1905_03634_v1}.

\subsection{Reduced-Order Modeling}

Reduced-order models compress high-dimensional flow fields into low-dimensional representations suitable for rapid prediction \cite{2509_20683_v1, 2511_04567_v1, 2407_02519_v1, 2408_07110_v1, 2409_01626_v1}. Proper orthogonal decomposition identifies optimal bases for linear dimensionality reduction \cite{2509_20683_v1, 2510_22469_v2, 2510_22839_v2}, while autoencoders learn nonlinear manifolds \cite{2211_16845_v2, 2411_17095_v1, 2511_10910_v1, 2511_16511_v1}. These approaches complement wall functions by providing efficient full-field predictions that the wall function can then refine near boundaries \cite{2512_13336_v2}.

Neural operators learn mappings between function spaces, enabling prediction of flow fields for new boundary conditions or parameters without retraining \cite{2108_11985_v1, 2501_00014_v1, 2503_17704_v1}. When combined with wall functions, neural operators could provide rapid predictions across families of geometries or operating conditions, with the wall function ensuring accurate near-wall behavior.

\subsection{Multi-Fidelity and Multi-Scale Approaches}

Many applications require predictions at multiple fidelity levels or across multiple scales \cite{2502_05577_v2, 2211_00601_v1, 2504_13750_v1, 2504_14473_v1}. Multi-fidelity machine learning combines cheap low-fidelity data with expensive high-fidelity data to achieve accuracy at reduced cost \cite{2211_00601_v1, 2504_14485_v1, 2504_15952_v1}. For wall functions, this could mean training on abundant RANS data augmented by limited DNS or experimental data \cite{2502_05577_v2, 2505_00522_v2, 2505_01681_v1}.

Multi-scale approaches explicitly model the interaction between resolved and unresolved scales \cite{2411_17095_v1, 2509_20683_v1}. Super-resolution techniques reconstruct fine-scale structure from coarse representations \cite{2509_20683_v1, 2211_16845_v2}. These ideas connect to wall function development by providing principled frameworks for coupling different resolution levels near boundaries.

\section{Research Gaps and Thesis Contributions}
\label{sec:lit_gaps}

This comprehensive literature review reveals specific gaps that the present thesis addresses through its five major contributions.

The first gap concerns systematic comparison of input representations \cite{2206_05226_v2, 2307_13144_v1}. Despite extensive work on machine learning wall functions, no study has systematically compared primitive variables versus physics-based features under controlled conditions. Chapter~\ref{chap:physics_features} fills this gap through experiments comparing 11 core physics features against 58 comprehensive features and 90 primitive stencil variables, demonstrating that a compact physics-based representation achieves accuracy comparable to much larger primitive inputs while providing interpretability advantages.

The second gap concerns thermal predictions \cite{2201_03200_v2, 2202_00435_v1, 1910_03097_v1}. The literature focuses overwhelmingly on velocity wall functions, with thermal predictions receiving disproportionately little attention despite their engineering importance in applications from gas turbine cooling to building ventilation. All experimental chapters of this thesis predict both skin friction coefficient and Stanton number jointly, learning whatever dissimilarity exists between momentum and heat transfer from the training data rather than assuming Reynolds analogy.

The third gap concerns training data diversity \cite{2206_05226_v2, 2409_04143_v1}. Many studies train on simple geometries and hope for generalization to more complex configurations. Chapter~\ref{chap:methodology} presents training data specifically designed to span attached to separated conditions through systematic variation of pressure gradients and geometric parameters across 244 configurations, providing comprehensive coverage of the conditions machine learning wall functions must handle.

The fourth gap concerns physics encoding in network architecture \cite{1701_07102_v2, 1905_07510_v2}. Beyond input features and loss functions, the network architecture itself can encode physical knowledge. Chapter~\ref{chap:neurons} explores whether neural networks trained on primitive inputs discover physics-based features in their hidden layers, finding architecture-invariant features that validate the physics-based input design and suggest certain relationships emerge naturally from the learning problem.

The fifth gap concerns practical deployment \cite{2409_19851_v1, 2404_03542_v1}. Academic machine learning research rarely addresses implementation in production codes, limiting impact on engineering practice. Chapter~\ref{chap:openfoam} presents complete integration with OpenFOAM including boundary condition implementation, model export utilities, and validation on benchmark geometries outside the training distribution.

\section{Chapter Summary}
\label{sec:lit_summary}

The trajectory from Prandtl's boundary layer theory through modern physics-informed neural networks spans over a century of scientific development. Each generation built upon predecessors: von K\'{a}rm\'{a}n's logarithmic profile upon Prandtl's thin-layer analysis, Launder and Spalding's wall functions upon von K\'{a}rm\'{a}n's scaling laws, and contemporary machine learning approaches upon both classical physics and the data-driven paradigm shift enabled by modern computing \cite{prandtl1904, karman1930, launder1974, raissi2019physics}.

This chapter has developed the theoretical foundations essential for understanding wall function modeling: the physics of turbulent boundary layers from viscous sublayer through logarithmic region to outer wake \cite{moser1999, lee2015, coles1956}, the mathematical framework underlying classical wall functions and their enhanced variants \cite{launder1974, spalding1961, wolfshtein1969}, the additional complexity of thermal boundary layers and their dissimilarity from momentum transport \cite{kader1981, 2201_03200_v2}, the implementation of wall functions in industrial practice and the gaps between capabilities and requirements \cite{2212_08989_v3, 2012_10165_v1}, and the recent machine learning revolution that offers fundamentally new approaches to these classical challenges \cite{1701_07102_v2, raissi2019physics, 2206_05226_v2, 2409_04143_v1}.

The present thesis continues this progression, combining physics-based feature engineering \cite{1905_07510_v2, 2210_15384_v1}, interpretable neural architectures \cite{2105_00913_v2, 2006_12483_v1}, physics-constrained training \cite{2505_00343_v1, 2406_00471_v1}, and practical deployment \cite{2409_19851_v1} into an integrated framework for machine learning wall functions. By addressing the gaps identified in this review, this work aims to advance the state of the art while honoring the intellectual heritage that makes such advances possible.

\end{document}
