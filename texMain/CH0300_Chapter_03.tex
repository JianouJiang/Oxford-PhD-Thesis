% !TeX root = ThesisMain.tex
% !TeX program = XeLaTeX
% !TeX encoding = UTF-8
% !TeX spellcheck = en_GB

\documentclass[../ThesisMain]{subfiles}
\ifSubfilesClassLoaded{}{}%

\begin{document}
\doublespacing%
\chapter{Methodology and Structured Data Generation}\label{chap:methodology}

This chapter presents the comprehensive methodology for generating structured training data for machine learning wall functions \cite{2409_04143_v1, 2206_05226_v2, 1701_07102_v2}. The dual-mesh approach pairs coarse mesh inputs with fine mesh ground truth, enabling supervised learning of wall quantities across diverse flow conditions \cite{2312_14902_v1, 2005_09023_v2}. The methodology is designed to produce high-quality, validated data suitable for training neural networks that can generalize across Reynolds numbers, pressure gradients, and geometric configurations \cite{2307_13144_v1, 2210_15384_v1}.

\section{Overview of the Data Generation Framework}
\label{sec:ch3_overview}

The central challenge in developing data-driven wall functions is obtaining paired training data that relates coarse mesh flow fields to accurate wall quantities. Traditional wall-resolved simulations provide ground truth but are computationally expensive \cite{moser1999, lee2015}, while coarse mesh simulations with standard wall functions may not accurately predict wall quantities in non-equilibrium flows \cite{launder1974, 2309_02109_v1}. Our approach leverages a dual-mesh methodology that addresses this challenge through three complementary components.

Fine mesh simulations employ wall-resolved meshes with $y^+ < 2$ at the first cell to provide ground truth for wall shear stress $\tau_w$ and wall heat flux $q_w$ \cite{2006_12483_v1, 1905_03634_v1}. These simulations integrate the turbulence equations directly to the wall without wall functions, ensuring that the computed wall gradients are not contaminated by wall function approximations. Coarse mesh simulations use practical meshes with $y^+ \approx 5$--$10$ at the first cell, providing input features that represent what a CFD solver would have access to during inference \cite{2404_03542_v1, 2202_04233_v3}. These are the meshes that would typically require wall functions in production simulations, making the learned mapping directly applicable to real-world applications. Stencil extraction completes the methodology by gathering local neighborhoods of cells from the coarse mesh to capture spatial context around each wall location, providing the neural network with information about the local flow structure that extends beyond the immediate wall-adjacent cell.

Figure~\ref{fig:data_pipeline} illustrates the complete data generation pipeline, showing how geometry parameters flow through mesh generation, CFD simulation, and feature extraction to produce the final training dataset.

\begin{figure}[H]
    \centering
    \includegraphics[width=0.98\textwidth]{figures/fig_3_1_mesh_comparison.pdf}
    \caption{Data generation methodology for ML wall function training. \textbf{Top row}: Schematic views of stencil extraction from coarse mesh (left, input) and wall gradient computation from fine mesh (right, target). \textbf{Middle row}: Full diffuser geometry meshes showing the coarse mesh ($y^+ \approx 30$) and fine mesh ($y^+ < 2$) with zoom regions indicated. \textbf{Bottom}: Data generation pipeline flowchart showing how geometry parameters flow through dual-mesh simulations to produce training pairs $(\mathbf{X}_{\text{stencil}}, \tau_w, q_w)$ for neural network training.}
    \label{fig:data_pipeline}
\end{figure}

The key insight of this approach is that the fine mesh simulation provides the ``correct'' wall quantities that the coarse mesh simulation should predict if it had access to a perfect wall function. By training a neural network on this paired data, we learn a mapping from coarse mesh flow features to fine mesh wall quantities---effectively learning an improved wall function from data.

\section{Geometry Design and Parameterization}
\label{sec:ch3_geometries}

The training dataset spans three geometry families designed to cover a comprehensive range of pressure gradient conditions. As discussed in the literature review (Chapter~\ref{chap:literature}), these geometries are motivated by classical benchmark cases for wall function validation, spanning from equilibrium channel flow to separated diffuser flows.

Figure~\ref{fig:geometry_types} shows all geometry variations used for training data generation, illustrating how parameters are systematically varied to create diverse training conditions. All geometries feature an asymmetric configuration with one flat wall (top) where training data is collected, and one inclined wall (bottom) that creates the pressure gradient.

\begin{figure}[H]
    \centering
    \includegraphics[width=0.98\textwidth]{figures/fig_geometry_family.png}
    \caption{Complete training geometry family showing 15 unique geometry shapes (10 diffusers, 1 channel, 4 nozzles). \textbf{Top}: Overview with all bottom wall profiles superimposed, showing how the flat top wall (training data source) remains constant while the bottom wall varies to create different pressure gradient conditions. \textbf{Bottom}: Grid of individual geometries sorted by expansion ratio. Diffusers (green/yellow) create adverse pressure gradients, channels (gray) provide zero pressure gradient baseline, and nozzles (blue) create favorable pressure gradients. Each geometry is simulated at two Reynolds numbers (12,000 and 18,000) with two mesh resolutions, yielding 60 total training simulations.}
    \label{fig:geometry_types}
\end{figure}

\subsection{Diffuser Geometries}
\label{sec:ch3_diffuser}

Diffuser configurations feature expanding channels that create adverse pressure gradients (APG), representing challenging conditions where traditional wall functions often exhibit significant errors \cite{driver1985, breuer2009, 2509_05886_v1}. The geometry is parameterized by three independent variables. The expansion ratio, defined as the ratio of outlet to inlet height $\text{ER} = H_{out}/H_{in}$, ranges from 1.5 to 5.5, covering mild to severe expansions. The divergence angle $\theta$, measured as the half-angle of the expanding section from the horizontal, ranges from $2°$ to $20°$, with larger angles producing stronger adverse pressure gradients. The Reynolds number, based on inlet conditions as $Re = U_{in} H_{in} / \nu$, ranges from 8,000 to 24,000, spanning the turbulent regime relevant to industrial applications.

The diffuser geometry consists of three sections: an inlet development region where the channel height remains constant, an expansion region where the height increases linearly, and an outlet region that allows the flow to recover. This configuration ensures that the flow is fully developed before encountering the pressure gradient, isolating the APG effects from inlet effects.

\subsection{Nozzle Geometries}
\label{sec:ch3_nozzle}

Nozzle configurations feature contracting channels that create favorable pressure gradients (FPG), representing accelerating flows \cite{clauser1954, 2408_08897_v1}. Under FPG conditions, the boundary layer thins and the flow remains attached even at high acceleration rates. The nozzle geometries use the same parameterization as diffusers but with contraction ratios. The contraction ratio $\text{CR} = H_{in}/H_{out} > 1$ ranges from 1.25 to 5.0, while the convergence angle (the half-angle of the contracting section) varies from $2°$ to $15°$. The Reynolds number spans the same range as diffusers, from 8,000 to 24,000.

Including both diffuser and nozzle geometries ensures that the neural network learns to distinguish between APG and FPG conditions and can predict wall quantities accurately in both regimes.

\subsection{Channel Flow}
\label{sec:ch3_channel}

Fully-developed channel flow cases serve as baseline configurations with zero pressure gradient (ZPG) \cite{moser1999, lee2015}. These cases have expansion ratio $\text{ER} = 1$ and provide reference conditions for model validation. The channel flow configuration allows direct comparison with DNS data and classical analytical solutions \cite{karman1930, spalding1961}, providing confidence in the simulation methodology.

Figure~\ref{fig:parameter_space} shows the coverage of the parameter space by the training dataset, demonstrating comprehensive sampling across the three geometry types.

\begin{figure}[H]
    \centering
    \includegraphics[width=0.98\textwidth]{chapter3/parameter_space.png}
    \caption{Parameter space coverage of the training dataset. (a) Geometry parameters showing expansion/contraction ratio versus divergence angle for all 244 configurations. (b) Reynolds number distribution across geometry types. (c) Dataset composition by geometry category, with sample counts indicated.}
    \label{fig:parameter_space}
\end{figure}

\section{Mesh Generation Methodology}
\label{sec:ch3_mesh}

Mesh quality is critical for obtaining accurate wall quantities from CFD simulations. Both fine and coarse meshes are generated using OpenFOAM's blockMesh utility, which creates structured hexahedral meshes with precise control over cell distribution near walls.

\subsection{Fine Mesh Specifications}

The fine mesh is designed to resolve the viscous sublayer and buffer region of the turbulent boundary layer, enabling direct integration of the turbulence equations to the wall. The first cell height is sized to achieve $y^+ < 2$ at the first cell center under all flow conditions, ensuring that the viscous sublayer ($y^+ < 5$) is resolved with multiple cells. A geometric progression with expansion ratio 1.1 gradually increases cell height away from the wall, providing smooth transitions while maintaining adequate resolution in the buffer layer ($5 < y^+ < 30$) and log-law region ($y^+ > 30$). The streamwise direction employs 200--400 cells with clustering near the inlet and in regions of strong pressure gradient. The resulting meshes typically contain 40,000--80,000 cells for 2D simulations, depending on geometry.

\subsection{Coarse Mesh Specifications}

The coarse mesh represents a practical mesh density that would typically be used with wall functions in production simulations. The first cell height is sized to achieve $y^+ \approx 5$--$10$ at the first cell center, placing the first cell in the buffer layer or lower log-law region where wall functions are designed to operate. A geometric progression with expansion ratio 1.2 provides faster growth away from the wall compared to the fine mesh. The streamwise direction employs 50--100 cells, and the resulting meshes typically contain 5,000--15,000 cells, representing a 5--8 times reduction from the fine mesh that reflects the computational savings achievable with wall functions.

Figure~\ref{fig:mesh_comparison} compares the near-wall mesh structure for fine and coarse configurations, highlighting the difference in resolution.

\begin{figure}[H]
    \centering
    \includegraphics[width=0.95\textwidth]{chapter3/mesh_comparison.png}
    \caption{Comparison of fine and coarse mesh near-wall resolution. (a) Fine mesh with $y^+ < 1$ first cell, showing dense clustering near the wall. (b) Coarse mesh with $y^+ \approx 30$--$300$ first cell, typical of production meshes requiring wall functions.}
    \label{fig:mesh_comparison}
\end{figure}

\subsection{Mesh Quality Metrics}

All generated meshes are verified against quality metrics to ensure reliable CFD solutions:

\begin{table}[H]
\centering
\caption{Mesh quality requirements for fine and coarse meshes.}
\label{tab:mesh_quality}
\begin{tabular}{|l|c|c|c|}
\hline
\textbf{Quality Metric} & \textbf{Fine Mesh} & \textbf{Coarse Mesh} & \textbf{Criterion} \\
\hline
Maximum skewness & $< 0.3$ & $< 0.4$ & $< 0.85$ (OpenFOAM) \\
Maximum aspect ratio & $< 50$ & $< 100$ & $< 1000$ \\
Non-orthogonality & $< 40°$ & $< 50°$ & $< 70°$ \\
Cell volume ratio & $< 3$ & $< 5$ & Adjacent cells \\
\hline
\end{tabular}
\end{table}

\subsection{Grid Independence Study}

To ensure that the fine mesh provides grid-independent results suitable as ground truth, a systematic grid independence study was conducted. Figure~\ref{fig:grid_independence} shows the convergence of skin friction coefficient and Nusselt number with increasing mesh density.

\begin{figure}[H]
    \centering
    \includegraphics[width=0.95\textwidth]{chapter3/grid_independence.png}
    \caption{Grid independence study for a representative diffuser case. (a) Skin friction coefficient convergence with mesh refinement. (b) Nusselt number convergence. Richardson extrapolation values and $\pm 1\%$ bands are shown. The finest mesh (used for ground truth) lies within 1\% of the extrapolated value for both quantities.}
    \label{fig:grid_independence}
\end{figure}

The grid independence study confirms that the fine mesh resolution is sufficient to provide accurate ground truth values. The discretization error is estimated to be less than 1\% for wall quantities, which is acceptable given other sources of uncertainty in RANS simulations.

\section{OpenFOAM Simulation Setup}
\label{sec:ch3_openfoam}

All simulations are performed using OpenFOAM v10, an open-source CFD platform widely used in both academia and industry \cite{2409_19851_v1, 2404_03542_v1}. The solver configuration is designed to produce accurate, converged solutions for turbulent heat transfer in internal flows \cite{2201_03200_v2, 2202_00435_v1}.

\subsection{Governing Equations}

The incompressible Reynolds-Averaged Navier-Stokes (RANS) equations are solved in steady-state form:

\textbf{Continuity equation:}
\begin{equation}
    \nabla \cdot \mathbf{U} = 0
    \label{eq:ch3_continuity}
\end{equation}

\textbf{Momentum equation:}
\begin{equation}
    \nabla \cdot (\mathbf{U} \mathbf{U}) = -\frac{1}{\rho}\nabla p + \nabla \cdot \left[(\nu + \nu_t)\left(\nabla \mathbf{U} + (\nabla \mathbf{U})^T\right)\right]
    \label{eq:ch3_momentum}
\end{equation}

\textbf{Energy equation:}
\begin{equation}
    \nabla \cdot (\mathbf{U} T) = \nabla \cdot \left[\left(\alpha + \alpha_t\right)\nabla T\right]
    \label{eq:ch3_energy}
\end{equation}

where $\mathbf{U}$ is the velocity vector, $p$ is the kinematic pressure, $\nu$ is the kinematic viscosity, $\nu_t$ is the turbulent viscosity, $T$ is temperature, $\alpha$ is thermal diffusivity, and $\alpha_t$ is turbulent thermal diffusivity.

\subsection{Turbulence Modeling}

The $k$-$\omega$ SST (Shear Stress Transport) turbulence model is employed for its demonstrated accuracy in wall-bounded flows with adverse pressure gradients \cite{2005_09023_v2, 2206_05226_v2}. The model blends the $k$-$\omega$ formulation near walls with the $k$-$\epsilon$ formulation in the outer region \cite{launder1974}, combining the strengths of both approaches.

The transport equations for turbulent kinetic energy $k$ and specific dissipation rate $\omega$ are:

\begin{equation}
    \nabla \cdot (\mathbf{U} k) = \nabla \cdot \left[(\nu + \sigma_k \nu_t) \nabla k\right] + P_k - \beta^* \omega k
    \label{eq:k_transport}
\end{equation}

\begin{equation}
    \nabla \cdot (\mathbf{U} \omega) = \nabla \cdot \left[(\nu + \sigma_\omega \nu_t) \nabla \omega\right] + \frac{\gamma}{\nu_t} P_k - \beta \omega^2 + CD_{k\omega}
    \label{eq:omega_transport}
\end{equation}

where $P_k$ is the production of turbulent kinetic energy and $CD_{k\omega}$ is the cross-diffusion term that enables blending between the two formulations.

For fine mesh simulations, no wall functions are used---the equations are integrated directly to the wall with appropriate low-Reynolds-number damping. For coarse mesh simulations, standard OpenFOAM wall functions are applied.

\subsection{Boundary Conditions}

Figure~\ref{fig:boundary_conditions} illustrates the boundary conditions applied to the diffuser geometry. The same structure is used for nozzle and channel configurations with appropriate modifications.

\begin{figure}[H]
    \centering
    \includegraphics[width=0.95\textwidth]{chapter3/boundary_conditions.png}
    \caption{Boundary conditions for the diffuser configuration. Inlet conditions include specified velocity profile and turbulence quantities. Walls are no-slip with fixed temperature. Outlet uses a zero-gradient pressure condition.}
    \label{fig:boundary_conditions}
\end{figure}

At the inlet, a uniform velocity profile $U = U_{in}$ is prescribed along with turbulence quantities derived from standard correlations. The turbulent kinetic energy is set as $k = \frac{3}{2}(U_{in} \cdot TI)^2$ using a turbulence intensity of $TI = 0.05$ (5\%), while the specific dissipation rate is computed as $\omega = k^{0.5}/(C_\mu^{0.25} \cdot l_t)$ where the turbulent length scale $l_t = 0.07 H_{in}$ is based on the inlet height. The inlet temperature is fixed at $T_{in} = 300$~K.

At the outlet, a fixed gauge pressure $p = 0$ is specified with zero gradient conditions for all other quantities, allowing the flow to exit freely without artificial constraints on the velocity or turbulence profiles.

At solid walls, the no-slip condition $\mathbf{U} = 0$ is enforced for velocity, and an isothermal boundary condition with $T_w = 330$~K is applied for temperature to drive heat transfer from the heated wall into the fluid. For turbulent quantities, wall functions are applied on the coarse mesh while low-Reynolds-number treatment (direct integration to the wall) is used on the fine mesh.

\subsection{Numerical Schemes and Solver Settings}

The following discretization schemes are used:

\begin{table}[H]
\centering
\caption{Numerical discretization schemes used in OpenFOAM simulations.}
\label{tab:schemes}
\begin{tabular}{|l|l|l|}
\hline
\textbf{Term} & \textbf{Scheme} & \textbf{Justification} \\
\hline
Time derivative & steadyState & Steady-state solution \\
Gradient & Gauss linear & Second-order accurate \\
Divergence (U) & Gauss linearUpwind & Bounded, low diffusion \\
Divergence (k, $\omega$, T) & Gauss upwind & Stability for turbulence \\
Laplacian & Gauss linear corrected & Second-order, non-orthogonal \\
Interpolation & linear & Second-order \\
\hline
\end{tabular}
\end{table}

The SIMPLE algorithm is used for pressure-velocity coupling with the following relaxation factors:

\begin{table}[H]
\centering
\caption{Under-relaxation factors for the SIMPLE algorithm.}
\label{tab:relaxation}
\begin{tabular}{|l|c|}
\hline
\textbf{Variable} & \textbf{Relaxation Factor} \\
\hline
Pressure ($p$) & 0.3 \\
Velocity ($\mathbf{U}$) & 0.7 \\
Turbulent kinetic energy ($k$) & 0.5 \\
Specific dissipation rate ($\omega$) & 0.5 \\
Temperature ($T$) & 0.7 \\
\hline
\end{tabular}
\end{table}

\subsection{Convergence Criteria}

Simulations are considered converged when all residuals fall below $10^{-6}$ and monitored quantities (wall shear stress, heat flux) show less than 0.1\% variation over 100 iterations. Figure~\ref{fig:residual_convergence} shows typical residual convergence behavior for a diffuser case.

\begin{figure}[H]
    \centering
    \includegraphics[width=0.85\textwidth]{chapter3/residual_convergence.png}
    \caption{Residual convergence history for a representative diffuser case. All residuals reach the convergence criterion of $10^{-6}$ within 5000 iterations. The velocity and pressure equations converge fastest, followed by temperature and turbulence quantities.}
    \label{fig:residual_convergence}
\end{figure}

\section{Validation Against Benchmark Data}
\label{sec:ch3_validation}

Before using the simulation results as training data, it is essential to verify that our OpenFOAM simulations produce accurate predictions in regions where the underlying physics is well understood. This validation serves two critical purposes. First, it establishes confidence that the training data generated from our fine mesh simulations can be trusted as ground truth for neural network training---if the simulations do not match established benchmarks in attached flow regions where RANS is expected to perform well, the resulting training data would teach the network incorrect relationships. Second, by using OpenFOAM consistently throughout this work, from training data generation to eventual deployment of the trained ML wall functions in Chapter~\ref{chap:pinn}, we ensure that any performance comparisons between traditional wall functions and our learned models are conducted on a level playing field, with identical numerical schemes, boundary condition implementations, and solver settings.

\subsection{Channel Flow Validation}

Fully-developed channel flow is validated against the DNS data of Moser, Kim, and Mansour at $Re_\tau = 180$ \cite{moser1999, 2006_12483_v1}. Figure~\ref{fig:velocity_profiles} compares the mean velocity profiles in wall units.

\begin{figure}[H]
    \centering
    \includegraphics[width=0.95\textwidth]{chapter3/velocity_profiles.png}
    \caption{Mean velocity profiles in wall units for three flow configurations. (a) Channel flow compared to DNS data and analytical laws. (b) Diffuser with adverse pressure gradient showing deviation from log-law. (c) Nozzle with favorable pressure gradient. Fine mesh results capture the physics accurately, while coarse mesh results show the behavior that wall functions must correct.}
    \label{fig:velocity_profiles}
\end{figure}

The fine mesh simulation captures the viscous sublayer ($u^+ = y^+$), buffer layer, and log-law region accurately, with deviations from DNS less than 2\% in the log-law region. This close agreement with high-fidelity DNS data confirms that our OpenFOAM simulations produce reliable wall shear stress and velocity gradient values in equilibrium channel flow, validating the channel flow portion of our training dataset.

\subsection{Diffuser Validation}

Diffuser simulations are validated against the experimental data of Buice and Eaton (1997) for a planar diffuser with 10° total divergence angle. Figure~\ref{fig:diffuser_validation} shows the skin friction coefficient distribution along the inclined wall, comparing our fine mesh RANS results with the experimental measurements.

\begin{figure}[H]
    \centering
    \includegraphics[width=0.95\textwidth]{chapter3/diffuser_validation.png}
    \caption{Diffuser validation against Buice \& Eaton (1997) experimental data. (a) Asymmetric diffuser geometry with 10° divergence and 4.7:1 expansion ratio. (b) Skin friction coefficient distribution showing good agreement between fine mesh k-$\omega$ SST simulation and experimental measurements. The expansion region is highlighted in yellow.}
    \label{fig:diffuser_validation}
\end{figure}

The fine mesh simulation captures the gradual reduction in $C_f$ through the expansion region, with agreement within 15\% of experimental values in the attached flow portions of the domain. This level of accuracy is consistent with the expected performance of the $k$-$\omega$ SST model in moderate adverse pressure gradient flows and provides confidence that our diffuser training data represents the true wall shear stress behavior. The slight under-prediction near the outlet reflects the well-documented limitations of two-equation RANS models as flows approach separation, a challenge that ultimately motivates the development of improved wall functions through machine learning.

\subsection{Heat Transfer Validation}

Thermal simulations are validated by comparing temperature profiles and Nusselt number distributions with DNS data and established correlations. Figure~\ref{fig:heat_transfer_validation} presents the validation results for channel flow heat transfer.

\begin{figure}[H]
    \centering
    \includegraphics[width=0.95\textwidth]{chapter3/heat_transfer_validation.png}
    \caption{Heat transfer validation. (a) Temperature profile in wall units ($T^+$ vs $y^+$) for channel flow at $Pr = 0.71$, comparing fine mesh results with JAXA DNS data and analytical laws. (b) Nusselt number development along the channel, showing entrance region effects and comparison with the Dittus-Boelter correlation ($Nu = 55.3$ for fully developed flow).}
    \label{fig:heat_transfer_validation}
\end{figure}

The Dittus-Boelter correlation provides a reference for developed flow regions:

\begin{equation}
    Nu = 0.023 \, Re^{0.8} \, Pr^{0.4}
    \label{eq:dittus_boelter}
\end{equation}

The fine mesh simulations agree with this correlation within 5\% for channel flow cases in the fully developed region, providing confidence that our thermal training data accurately represents the true wall heat flux values. The entrance region shows expected enhancement that decays toward the correlation value. Together with the velocity validation results, these comparisons establish that our OpenFOAM simulations produce reliable ground truth for both momentum and thermal wall quantities in attached flow conditions, forming a sound foundation for the subsequent machine learning experiments.

\section{Experimental Benchmark Data for Separated Flows}
\label{sec:ch3_benchmarks}

The preceding section demonstrated that our OpenFOAM simulations produce accurate results in attached flow regions, matching DNS and experimental benchmarks within a few percent. However, when flows separate due to strong adverse pressure gradients or geometric discontinuities, the situation changes fundamentally. Despite considerable effort to tune mesh resolution, solver settings, and turbulence model parameters, RANS simulations consistently struggle to match experimental skin friction distributions in separated regions. The classical wall function assumptions break down entirely when flow reverses near the wall, and even the underlying $k$-$\omega$ SST turbulence model---widely regarded as one of the most capable two-equation closures for adverse pressure gradient flows---cannot capture the complex physics of separation and reattachment with the fidelity achieved in attached regions. This is not a limitation of our particular implementation, but rather reflects fundamental shortcomings in RANS modeling of separated flows that persist across the CFD community \cite{driver1985, breuer2009}. It is precisely this gap between RANS capability and physical reality that motivates the development of machine learning wall functions capable of learning the true wall behavior from high-fidelity data.

To address this limitation and provide reliable training data for separated flow conditions, we incorporate experimental and DNS benchmark data from canonical separated flow configurations. While we cannot generate separated flow training data from our own RANS simulations with the same confidence as attached flow data, these established benchmarks provide the ground truth needed to train models that can recognize and correctly predict wall quantities in separated regions.

\subsection{Backward-Facing Step}
\label{sec:ch3_bfs}

The backward-facing step represents the most fundamental separated flow configuration, where a sudden geometric expansion triggers immediate flow separation at the step corner. This geometry has been extensively studied since the landmark experiments of Driver and Seegmiller (1985) \cite{driver1985}, making it an essential validation case for any turbulence modeling approach. Figure~\ref{fig:bfs_geometry} illustrates the geometry and characteristic flow structure.

\begin{figure}[H]
    \centering
    \includegraphics[width=0.95\textwidth]{chapter3/bfs_geometry.png}
    \caption{Backward-facing step geometry and flow structure. (a) Schematic showing the step height $H$, inlet boundary layer, separation at the step corner, recirculation zone, and reattachment downstream. (b) Characteristic flow regions: 1--upstream attached flow, 2--separation shear layer, 3--recirculation zone with $\tau_w < 0$, 4--reattachment region, 5--recovery zone.}
    \label{fig:bfs_geometry}
\end{figure}

The Driver-Seegmiller experiment was conducted at $Re_H = 37{,}500$ based on step height, with a modest expansion ratio of 1.125 and a turbulent inlet boundary layer of thickness $\delta/H \approx 1.5$. The flow separates immediately at the step corner and reattaches at $x_R/H = 6.26 \pm 0.10$ downstream, creating a well-defined recirculation zone where the wall shear stress becomes negative. This dataset, available through the NASA Turbulence Modeling Resource, includes detailed measurements of skin friction, surface pressure, and velocity profiles at multiple streamwise stations.

\begin{figure}[H]
    \centering
    \includegraphics[width=0.85\textwidth]{chapter3/bfs_cf_profile.png}
    \caption{Skin friction coefficient distribution for the backward-facing step from Driver \& Seegmiller (1985). The flow separates at the step corner ($x/H = 0$), exhibits negative $C_f$ in the recirculation zone (minimum at $x/H \approx 3$), reattaches at $x/H \approx 6.26$, and recovers downstream. The shaded region indicates where classical wall functions fail.}
    \label{fig:bfs_cf_profile}
\end{figure}

Figure~\ref{fig:bfs_cf_profile} shows the characteristic skin friction distribution, which transitions through three distinct regions. In the recirculation zone extending from the step to $x/H \approx 6.26$, the reversed flow produces negative $C_f$ with a minimum of approximately $-0.002$ near $x/H = 3$. At reattachment, the skin friction passes through zero with steep gradients in both $C_f$ and $C_p$. Beyond reattachment, the flow gradually recovers toward equilibrium boundary layer conditions. DNS data from Le and Moin (1997) at lower Reynolds number ($Re_H = 5{,}100$) complements these measurements with complete turbulence statistics including Reynolds stress tensor components. When we attempted to reproduce these results using our OpenFOAM setup with the $k$-$\omega$ SST model, the predicted reattachment length differed from the experimental value by approximately 20\%, and the skin friction magnitude in the recirculation zone showed significant discrepancies. This outcome, while consistent with published RANS studies of backward-facing steps, underscores why experimental benchmark data is essential for training ML wall functions that must work correctly in separated regions.

\subsection{Wall-Mounted Hump}
\label{sec:ch3_wall_hump}

The NASA wall-mounted hump, based on a modified Glauert-Goldschmied body, provides a benchmark for pressure-induced separation over a smoothly curved surface. Unlike the abrupt separation at a backward-facing step, this geometry features gradual separation driven purely by the adverse pressure gradient on the hump's leeward side, making it particularly relevant for validating wall function behavior under smooth-wall separation conditions. Figure~\ref{fig:wall_hump_geometry} shows the geometry and flow characteristics.

\begin{figure}[H]
    \centering
    \includegraphics[width=0.95\textwidth]{chapter3/wall_hump_geometry.png}
    \caption{Wall-mounted hump geometry based on NASA experiments. (a) The hump profile with chord length $c = 420$ mm and maximum height $h/c = 0.128$ mounted on a flat plate. Flow accelerates over the windward face and separates on the leeward side. (b) Skin friction distribution showing the favorable pressure gradient (FPG) region, separation point, recirculation zone, and recovery.}
    \label{fig:wall_hump_geometry}
\end{figure}

The experiments were conducted at chord Reynolds number $Re_c = 936{,}000$, significantly higher than other separated flow benchmarks, which tests the ability of wall models to capture high-Reynolds-number separation physics. The hump's maximum height of $h/c = 0.128$ creates strong acceleration over the windward face followed by rapid deceleration on the leeward side. Separation occurs near $x/c = 0.67$ and the flow reattaches at approximately $x/c = 1.1$, creating a separation bubble that extends over roughly 40\% of the chord length. The dataset includes detailed surface pressure and skin friction measurements, providing ground truth for validation across the entire separation and recovery process. RANS models, including the $k$-$\omega$ SST formulation used in this work, typically predict separation onset too late and a separation bubble that is too short compared to experimental measurements, again highlighting the need for data-driven corrections in these flow regimes.

\subsection{Periodic Hills}
\label{sec:ch3_periodic_hills}

The periodic hills configuration, established as ERCOFTAC benchmark case UFR 3-30, extends separated flow validation to geometries with cyclic separation and reattachment \cite{breuer2009}. The domain consists of a channel with smoothly-contoured hills on the lower wall, where the flow separates on each hill's lee side and reattaches in the valley before the next hill. This periodic arrangement enables collection of statistically converged data from a compact computational domain while testing model behavior under repeated pressure gradient reversals.

\begin{figure}[H]
    \centering
    \includegraphics[width=0.95\textwidth]{chapter3/periodic_hills_geometry.png}
    \caption{Periodic hills geometry showing two periods of the domain. (a) Hill profile with height $H$, period $L_x = 9H$, and channel height $L_y = 3.035H$. The polynomial hill shape creates smooth pressure gradient transitions. (b) Flow structure showing separation on each lee side and reattachment in the valleys, with recirculation indicated by reversed streamlines.}
    \label{fig:periodic_hills_geometry}
\end{figure}

The comprehensive study by Breuer et al. (2009) provides both LDA measurements and high-fidelity LES data at $Re_H = 10{,}595$, with additional DNS data available at $Re_H = 5{,}600$. The separation point occurs consistently at $x/H \approx 0.5$ on the lee side of each hill crest, while the reattachment location varies with Reynolds number, typically falling near $x/H \approx 4.5$ for the reference case. Figure~\ref{fig:periodic_hills_cf} presents the wall shear stress distribution over one period, illustrating the rapid transition from favorable pressure gradient on the windward side to separation and eventual recovery.

\begin{figure}[H]
    \centering
    \includegraphics[width=0.85\textwidth]{chapter3/periodic_hills_cf.png}
    \caption{Wall shear stress distribution over one period of the periodic hills geometry at $Re_H = 10{,}595$ from Breuer et al. (2009). The flow experiences favorable pressure gradient (FPG) on the windward face with high $C_f$, transitions to adverse pressure gradient (APG) past the crest, separates near $x/H = 0.5$, and recovers after reattachment near $x/H = 4.5$.}
    \label{fig:periodic_hills_cf}
\end{figure}

The periodic nature of this geometry makes it particularly demanding for RANS models, as errors in predicting separation or reattachment accumulate over successive hill periods. Published RANS studies consistently under-predict the separation bubble extent, and our own OpenFOAM simulations exhibit similar deficiencies. This periodic hills benchmark therefore provides a stringent test for any ML wall function that claims to improve upon classical RANS predictions in complex separated flows.

\subsection{DNS Databases}
\label{sec:ch3_dns_databases}

In addition to experimental data, Direct Numerical Simulation databases provide complete flow field information at high fidelity, enabling extraction of any quantity of interest including higher-order statistics. We have downloaded and processed data from several authoritative DNS databases, summarized in Table~\ref{tab:dns_databases}.

\begin{table}[H]
\centering
\caption{DNS databases used for validation and training data augmentation.}
\label{tab:dns_databases}
\small
\begin{tabular}{llll}
\toprule
\textbf{Database} & \textbf{Flow Type} & \textbf{$Re_\tau$ / Conditions} & \textbf{Data Available} \\
\midrule
MKM \cite{moser1999} & Channel & 180, 395, 590 & $u^+$, $\overline{u'v'}$, $k$ \\
Lee \& Moser \cite{lee2015} & Channel & 1000, 2000, 5200 & Mean profiles, stresses \\
JAXA Thermal & Channel (heated) & 180--1020 & $T^+$, $Pr_t$ \\
KTH Stockholm & Flat plate TBL & $Re_\theta = 670$--$4{,}300$ & $u^+$, $C_f$, $\delta^*$ \\
Wu et al. & Bypass transition & 50+ stations & Heat transfer, $\overline{u'T'}$ \\
JHTDB & Channel & 1000, 5200 & Full 3D fields \\
\bottomrule
\end{tabular}
\end{table}

The MKM database from Moser, Kim, and Mansour (1999) provides the primary validation data for our channel flow cases, with complete mean velocity profiles, Reynolds stress components, and turbulent kinetic energy budgets at three Reynolds numbers spanning the range typical of industrial applications. The JAXA thermal DNS database extends this to heated channel flows, providing temperature profiles and turbulent Prandtl number distributions at multiple Prandtl numbers, which is essential for validating our thermal wall function predictions.

For boundary layer flows, the KTH Stockholm database from Schlatter and \"{O}rl\"{u} provides high-fidelity flat plate turbulent boundary layer data across a range of momentum thickness Reynolds numbers. The transitional DNS data from Wu et al. captures bypass transition with detailed heat transfer statistics at over 50 streamwise stations, enabling validation of models in the challenging transitional regime.

The Johns Hopkins Turbulence Database (JHTDB) provides access to exceptionally large DNS datasets with full three-dimensional, time-resolved flow fields. While the complete datasets are too large for local storage (exceeding 100 TB), we access specific quantities through the database's web API for targeted validation studies at higher Reynolds numbers than the locally stored MKM data.

\subsection{Role of Benchmark Data in Training}
\label{sec:ch3_benchmark_role}

The experimental and DNS benchmark data described above serve a critical role in our machine learning wall function development. Most fundamentally, they fill the gap where our RANS simulations cannot be trusted: as demonstrated in the preceding subsections, even well-tuned RANS models with fine mesh resolution consistently fail to match experimental skin friction distributions in separated regions. By incorporating benchmark data as training targets, we enable the neural network to learn the correct wall behavior in flow regimes where RANS-generated ground truth would be misleading.

Beyond training data augmentation, the benchmarks provide independent validation of model predictions that is particularly valuable for separated flows outside the training distribution. The detailed measurements also inform our feature engineering decisions, as they reveal which flow quantities remain predictable in separated regions and which lose their correlation with wall quantities. Finally, comparing the feature distributions between attached flow data from RANS and separated flow data from experiments helps quantify the domain shift that limits generalization and guides strategies to bridge this gap. The integration of benchmark data into the training pipeline is investigated systematically in Chapters~\ref{chap:physics_features}--\ref{chap:pinn}, where we compare model performance with and without benchmark data augmentation.

\subsubsection{Simulate-and-Replace Integration Method}
\label{sec:ch3_simulate_replace}

A key challenge is creating compatible training pairs from benchmark data, since experimental measurements typically provide only wall quantities ($C_f$ versus $x/H$) without the full flow field needed for stencil feature extraction. We address this through a \emph{simulate-and-replace} approach. First, a coarse-mesh RANS simulation of the benchmark geometry is performed using the same mesh specifications ($y^+ \approx 5$--$10$) as the main training data. Stencil-based input features $\mathbf{X}$ are then extracted from this coarse mesh following the procedure described in Section~\ref{sec:ch3_stencil}. The experimental or DNS $C_f$ values are interpolated to the wall mesh points, and training pairs $(\mathbf{X}_\text{coarse}, C_{f,\text{exp}})$ are created by replacing the RANS-predicted $C_f$ with the experimental target.

This approach is illustrated conceptually:
\begin{equation}
    \underbrace{\mathbf{X}_\text{coarse}}_{\text{From RANS}} \xrightarrow{\text{NN}} \hat{C}_f \approx \underbrace{C_{f,\text{exp}}}_{\text{From benchmark}}
\end{equation}

The key insight is that by pairing RANS features with experimental targets, the neural network learns to correct the systematic bias in RANS predictions for separated flows. When the coarse mesh shows features indicative of separation (decelerating flow, adverse pressure gradient, flow reversal near wall), the model learns that the true $C_f$ differs from what RANS would predict.

\subsubsection{Target Variable Consistency: Converting $C_f$ to $\tau_w$}
\label{sec:ch3_cf_conversion}

Our existing training pipeline predicts dimensional wall quantities---wall shear stress $\tau_w$ and wall heat flux $q_w$---rather than non-dimensional coefficients. Since benchmark data typically reports skin friction coefficient $C_f$, we must convert to dimensional form for consistency. The conversion is straightforward:

\begin{equation}
    \tau_w = C_f \times \frac{1}{2} \rho U_{ref}^2
    \label{eq:cf_to_tau}
\end{equation}

where $\rho$ is the fluid density and $U_{ref}$ is the reference velocity (typically bulk velocity or freestream velocity, as defined by each benchmark). This conversion is mathematically exact---$C_f$ is simply the non-dimensional form of $\tau_w$---and introduces no approximation. Each benchmark configuration specifies the reference conditions ($Re$, characteristic length $H$, and thereby $U_{ref}$) needed for this conversion.

For example, for the backward-facing step at $Re_H = 37{,}500$ with step height $H = 12.7$~mm and air at standard conditions ($\nu = 1.5 \times 10^{-5}$~m$^2$/s, $\rho = 1.2$~kg/m$^3$):
\begin{equation}
    U_{ref} = \frac{Re_H \cdot \nu}{H} = \frac{37{,}500 \times 1.5 \times 10^{-5}}{0.0127} \approx 44.3~\text{m/s}
\end{equation}

A measured $C_f = -0.003$ in the recirculation zone then converts to:
\begin{equation}
    \tau_w = -0.003 \times \frac{1}{2} \times 1.2 \times (44.3)^2 \approx -3.5~\text{Pa}
\end{equation}

This dimensional consistency ensures that benchmark-derived training samples integrate seamlessly with RANS-generated data where $\tau_w$ is computed directly from the wall velocity gradient.

\subsubsection{Thermal Data Limitations in Benchmark Experiments}
\label{sec:ch3_thermal_limitation}

A significant limitation of available separated flow benchmarks is the scarcity of heat transfer measurements. While momentum quantities such as skin friction coefficient, pressure coefficient, and velocity profiles are routinely measured using pressure taps, hot-wire anemometry, or laser Doppler anemometry, thermal measurements demand considerably more complex experimental apparatus. Accurate wall heat flux measurements require heated or cooled test section walls with precisely controlled boundary conditions, embedded thermocouples or infrared thermography for surface temperature measurement, and careful thermal insulation to minimize heat losses to the surroundings. These additional requirements explain why most canonical separated flow benchmarks---including the backward-facing step of Driver and Seegmiller \cite{driver1985}, the periodic hills of Breuer et al. \cite{breuer2009}, and the asymmetric diffuser of Buice and Eaton---provide only momentum data. Notable exceptions include DNS databases that can simulate passive scalar transport and dedicated heated experiments such as the heated backward-facing step of Vogel and Eaton \cite{vogel1985}. This asymmetry in available benchmark data means that our training methodology can augment $\tau_w$ targets in separated regions with high-fidelity experimental values, while $q_w$ predictions must rely primarily on RANS-generated training data. The implications of this limitation are examined in Chapter~\ref{chap:pinn}, where we investigate whether shared feature representations learned from momentum benchmarks can indirectly improve thermal predictions.

Table~\ref{tab:benchmark_summary} summarizes the key characteristics of each benchmark case.

\begin{table}[H]
\centering
\small
\caption{Summary of experimental and DNS benchmark cases for separated flows.}
\label{tab:benchmark_summary}
\begin{tabular}{llll}
\toprule
\textbf{Benchmark} & \textbf{$Re$} & \textbf{Separation} & \textbf{Data} \\
\midrule
Backward-facing step & 37,500 & Geometry-induced & $C_f$, $C_p$, $U(y)$ \\
Periodic hills & 10,595 & Pressure-induced & $\tau_w$, $U(y)$ \\
Asymmetric diffuser & 20,000 & Smooth APG & $C_f$, $C_p$ \\
Channel flow DNS & $Re_\tau = 1000$ & Attached & Full field \\
\bottomrule
\end{tabular}
\end{table}

The integration of these benchmark datasets with our RANS-generated training data addresses the fundamental challenge identified in Chapter~\ref{chap:literature}: developing wall functions that remain accurate across the full spectrum of flow conditions, from equilibrium attached flows to complex separated regions.

\section{Stencil-Based Input Representation}
\label{sec:ch3_stencil}

For each wall-adjacent cell in the coarse mesh, a structured $3 \times 5$ stencil of neighboring cells is extracted, capturing local flow context in both streamwise and wall-normal directions. The stencil consists of 15 cells arranged in a 3 (streamwise) $\times$ 5 (wall-normal) grid centered on the wall-adjacent cell of interest. For curved or inclined wall surfaces, stencils are extracted using local wall-aligned coordinate transformations, where the tangent and normal vectors adapt to the local wall orientation. This enables the trained model to be deployed on arbitrary geometries.

For each of the 15 cells in the stencil, six primitive variables are extracted: streamwise and wall-normal coordinates $(x, y)$, pressure $p$, velocity components $(U_x, U_y)$, and temperature $T$. This yields an input vector of dimension $3 \times 5 \times 6 = 90$.

\section{Dataset Summary and Statistics}
\label{sec:ch3_dataset}

The complete training dataset consists of paired input-output samples extracted from the 244 geometry configurations. The neural network is trained to predict two wall quantities from the stencil inputs. The \textbf{wall shear stress} $\tau_w = \mu (\partial U_x / \partial y)|_{y=0}$ is computed from the fine mesh velocity gradient, with the non-dimensional skin friction coefficient $C_f = \tau_w / (\frac{1}{2} \rho U_{ref}^2)$ used as the training target. The \textbf{wall heat flux} $q_w = -k (\partial T / \partial y)|_{y=0}$ is similarly computed from the fine mesh temperature gradient, with the Stanton number $\mathrm{St} = q_w / (\rho C_p U_{ref} \Delta T)$ as the thermal target. In separated flow regions, the wall shear stress becomes negative ($\tau_w < 0$), and this sign is preserved in the training data to enable the network to learn separation prediction.

\begin{table}[H]
\centering
\caption{Summary of the generated training dataset.}
\label{tab:dataset_summary}
\small
\begin{tabular}{lc}
\toprule
\textbf{Parameter} & \textbf{Value} \\
\midrule
Total geometry configurations & 244 \\
\quad Diffuser cases & 180 \\
\quad Nozzle cases & 60 \\
\quad Channel cases & 4 \\
\midrule
Total training samples & 25,485 \\
Input features per sample & 90 (primitive) \\
Output targets per sample & 2 ($C_f$, St) \\
\midrule
Expansion ratio range & 0.5--5.5 \\
Divergence angle range & $0°$--$20°$ \\
Reynolds number range & 8,000--24,000 \\
\bottomrule
\end{tabular}
\end{table}

\subsection{Dataset Statistics}

Figure~\ref{fig:dataset_statistics} presents the statistical distributions of the output variables and their correlations. The $C_f$ distribution is positively skewed, exhibiting a long tail toward higher values that corresponds to accelerating flow regions in nozzle geometries where the boundary layer thins and wall shear stress increases. The Stanton number distribution shows similar characteristics, reflecting the Reynolds analogy between momentum and heat transfer that couples these two quantities. The correlation coefficient between $C_f$ and St is approximately 0.85, indicating a strong but imperfect relationship---the deviation from perfect correlation arises from variations in the turbulent Prandtl number and from thermal boundary layer effects that differ from momentum boundary layer behavior, particularly in regions of strong pressure gradient or near separation.

\begin{figure}[H]
    \centering
    \includegraphics[width=0.95\textwidth]{chapter3/dataset_statistics.png}
    \caption{Statistical analysis of the training dataset. (a) Distribution of skin friction coefficient $C_f$ values. (b) Distribution of Stanton number St values. (c) Correlation between $C_f$ and St, showing the expected positive relationship. (d) Summary statistics table including mean, standard deviation, and percentiles.}
    \label{fig:dataset_statistics}
\end{figure}

\section{Chapter Summary}
\label{sec:ch3_summary}

This chapter has presented the methodology for generating structured training data for machine learning wall functions. The dual-mesh approach pairs fine mesh simulations ($y^+ < 2$) that provide ground truth wall shear stress and heat flux with coarse mesh simulations ($y^+ \approx 5$--$10$) that provide the input features available during inference. The dataset spans 244 geometry configurations---including diffusers with adverse pressure gradients, nozzles with favorable pressure gradients, and baseline channel flows---covering Reynolds numbers from 8,000 to 24,000.

Grid independence studies and validation against DNS and experimental benchmarks confirm that the fine mesh simulations provide reliable ground truth in attached flow regions, with less than 2\% deviation from DNS in channel flow and within 5\% of heat transfer correlations. However, RANS simulations consistently fail to match experimental skin friction distributions in separated regions, motivating the incorporation of high-fidelity benchmark data from backward-facing step, wall-mounted hump, and periodic hills experiments.

The final dataset contains 25,485 paired samples, with each sample comprising a $3 \times 5$ stencil of primitive variables (90 inputs) and corresponding wall quantities ($C_f$ and St). The stencil extraction uses wall-aligned coordinates to enable deployment on curved geometries. The dataset includes both attached flow regions where $\tau_w > 0$ and separated regions where $\tau_w < 0$, providing the diversity needed for training models that generalize across flow conditions.

In the following chapters, we develop machine learning models that learn from this data. Chapter~\ref{chap:baseline} establishes baseline performance using primitive variables as network inputs, while subsequent chapters investigate physics-informed approaches to improve generalization and accuracy.

\end{document}
