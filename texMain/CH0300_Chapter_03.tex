% !TeX root = ThesisMain.tex
% !TeX program = XeLaTeX
% !TeX encoding = UTF-8
% !TeX spellcheck = en_GB

\documentclass[../ThesisMain]{subfiles}
\ifSubfilesClassLoaded{}{}%

\begin{document}
\doublespacing%
\chapter{Methodology and Structured Data Generation}\label{chap:methodology}

This chapter presents the comprehensive methodology for generating structured training data for machine learning wall functions \cite{2409_04143_v1, 2206_05226_v2, 1701_07102_v2}. The dual-mesh approach pairs coarse mesh inputs with fine mesh ground truth, enabling supervised learning of wall quantities across diverse flow conditions \cite{2312_14902_v1, 2005_09023_v2}. The methodology is designed to produce high-quality, validated data suitable for training neural networks that can generalize across Reynolds numbers, pressure gradients, and geometric configurations \cite{2307_13144_v1, 2210_15384_v1}.

\section{Overview of the Data Generation Framework}
\label{sec:ch3_overview}

The central challenge in developing data-driven wall functions is obtaining paired training data that relates coarse mesh flow fields to accurate wall quantities. Traditional wall-resolved simulations provide ground truth but are computationally expensive \cite{moser1999, lee2015}, while coarse mesh simulations with standard wall functions may not accurately predict wall quantities in non-equilibrium flows \cite{launder1974, 2309_02109_v1}. Our approach leverages a dual-mesh methodology that addresses this challenge:

\begin{enumerate}
    \item \textbf{Fine mesh simulations}: Wall-resolved meshes with $y^+ < 2$ at the first cell provide ground truth for wall shear stress $\tau_w$ and wall heat flux $q_w$ \cite{2006_12483_v1, 1905_03634_v1}. These simulations integrate the turbulence equations directly to the wall without wall functions.

    \item \textbf{Coarse mesh simulations}: Practical meshes with $y^+ \approx 5$--$10$ at the first cell provide input features representing what a CFD solver would have access to during inference \cite{2404_03542_v1, 2202_04233_v3}. These are the meshes that would typically require wall functions in production simulations.

    \item \textbf{Stencil extraction}: Local neighborhoods of cells are extracted from the coarse mesh to capture spatial context around each wall location, providing the neural network with information about the local flow structure.
\end{enumerate}

Figure~\ref{fig:data_pipeline} illustrates the complete data generation pipeline, showing how geometry parameters flow through mesh generation, CFD simulation, and feature extraction to produce the final training dataset.

\begin{figure}[H]
    \centering
    \includegraphics[width=0.95\textwidth]{chapter3/data_pipeline.png}
    \caption{Data generation pipeline for the ML wall function training dataset. The process begins with geometry parameterization, proceeds through parallel fine and coarse mesh simulations, and concludes with feature extraction to create input-output pairs for supervised learning.}
    \label{fig:data_pipeline}
\end{figure}

The key insight of this approach is that the fine mesh simulation provides the ``correct'' wall quantities that the coarse mesh simulation should predict if it had access to a perfect wall function. By training a neural network on this paired data, we learn a mapping from coarse mesh flow features to fine mesh wall quantities---effectively learning an improved wall function from data.

\section{Geometry Design and Parameterization}
\label{sec:ch3_geometries}

The training dataset spans three geometry families designed to cover a comprehensive range of pressure gradient conditions. As discussed in the literature review (Chapter~\ref{chap:literature}), these geometries are motivated by classical benchmark cases for wall function validation, spanning from equilibrium channel flow to separated diffuser flows.

Figure~\ref{fig:geometry_types} shows the three geometry families used for training data generation. All geometries feature an asymmetric configuration with one flat wall (top) where training data is collected, and one inclined wall (bottom) that creates the pressure gradient.

\begin{figure}[H]
    \centering
    \includegraphics[width=0.95\textwidth]{chapter3/geometry_types.png}
    \caption{The three geometry families used for training data generation. (a) Asymmetric diffuser with adverse pressure gradient (APG). (b) Asymmetric nozzle with favorable pressure gradient (FPG). (c) Channel flow with zero pressure gradient (ZPG). Training data is collected from the flat top wall in all cases.}
    \label{fig:geometry_types}
\end{figure}

\subsection{Diffuser Geometries}
\label{sec:ch3_diffuser}

Diffuser configurations feature expanding channels that create adverse pressure gradients (APG), representing challenging conditions where traditional wall functions often exhibit significant errors \cite{driver1985, breuer2009, 2509_05886_v1}. The geometry is parameterized by three independent variables:

\begin{itemize}
    \item \textbf{Expansion ratio (ER)}: Defined as the ratio of outlet to inlet height, $\text{ER} = H_{out}/H_{in}$. Values range from 1.5 to 5.5, covering mild to severe expansions.

    \item \textbf{Divergence angle ($\theta$)}: The half-angle of the expanding section measured from the horizontal. Values range from $2°$ to $20°$, with larger angles producing stronger adverse pressure gradients.

    \item \textbf{Reynolds number ($Re$)}: Based on inlet conditions, $Re = U_{in} H_{in} / \nu$. Values range from 8,000 to 24,000, spanning the turbulent regime.
\end{itemize}

The diffuser geometry consists of three sections: an inlet development region where the channel height remains constant, an expansion region where the height increases linearly, and an outlet region that allows the flow to recover. This configuration ensures that the flow is fully developed before encountering the pressure gradient, isolating the APG effects from inlet effects.

\subsection{Nozzle Geometries}
\label{sec:ch3_nozzle}

Nozzle configurations feature contracting channels that create favorable pressure gradients (FPG), representing accelerating flows \cite{clauser1954, 2408_08897_v1}. Under FPG conditions, the boundary layer thins and the flow remains attached even at high acceleration rates. The nozzle geometries use the same parameterization as diffusers but with contraction ratios:

\begin{itemize}
    \item \textbf{Contraction ratio}: $\text{CR} = H_{in}/H_{out} > 1$, ranging from 1.25 to 5.0
    \item \textbf{Convergence angle}: Half-angle of the contracting section, $2°$ to $15°$
    \item \textbf{Reynolds number}: Same range as diffusers, 8,000 to 24,000
\end{itemize}

Including both diffuser and nozzle geometries ensures that the neural network learns to distinguish between APG and FPG conditions and can predict wall quantities accurately in both regimes.

\subsection{Channel Flow}
\label{sec:ch3_channel}

Fully-developed channel flow cases serve as baseline configurations with zero pressure gradient (ZPG) \cite{moser1999, lee2015}. These cases have expansion ratio $\text{ER} = 1$ and provide reference conditions for model validation. The channel flow configuration allows direct comparison with DNS data and classical analytical solutions \cite{karman1930, spalding1961}, providing confidence in the simulation methodology.

Figure~\ref{fig:parameter_space} shows the coverage of the parameter space by the training dataset, demonstrating comprehensive sampling across the three geometry types.

\begin{figure}[H]
    \centering
    \includegraphics[width=0.98\textwidth]{chapter3/parameter_space.png}
    \caption{Parameter space coverage of the training dataset. (a) Geometry parameters showing expansion/contraction ratio versus divergence angle for all 244 configurations. (b) Reynolds number distribution across geometry types. (c) Dataset composition by geometry category, with sample counts indicated.}
    \label{fig:parameter_space}
\end{figure}

\section{Mesh Generation Methodology}
\label{sec:ch3_mesh}

Mesh quality is critical for obtaining accurate wall quantities from CFD simulations. Both fine and coarse meshes are generated using OpenFOAM's blockMesh utility, which creates structured hexahedral meshes with precise control over cell distribution near walls.

\subsection{Fine Mesh Specifications}

The fine mesh is designed to resolve the viscous sublayer and buffer region of the turbulent boundary layer, enabling direct integration of the turbulence equations to the wall. Key specifications include:

\begin{itemize}
    \item \textbf{First cell height}: Sized to achieve $y^+ < 2$ at the first cell center under all flow conditions. This ensures that the viscous sublayer ($y^+ < 5$) is resolved with multiple cells.

    \item \textbf{Wall-normal expansion ratio}: A geometric progression with ratio 1.1 is used to gradually increase cell height away from the wall. This provides smooth transitions while maintaining adequate resolution in the buffer layer ($5 < y^+ < 30$) and log-law region ($y^+ > 30$).

    \item \textbf{Streamwise resolution}: 200--400 cells in the streamwise direction, with clustering near the inlet and in regions of strong pressure gradient.

    \item \textbf{Total cell count}: Typically 40,000--80,000 cells for 2D simulations, depending on geometry.
\end{itemize}

\subsection{Coarse Mesh Specifications}

The coarse mesh represents a practical mesh density that would typically be used with wall functions in production simulations:

\begin{itemize}
    \item \textbf{First cell height}: Sized to achieve $y^+ \approx 5$--$10$ at the first cell center. This places the first cell in the buffer layer or lower log-law region.

    \item \textbf{Wall-normal expansion ratio}: A geometric progression with ratio 1.2 provides faster growth away from the wall.

    \item \textbf{Streamwise resolution}: 50--100 cells in the streamwise direction.

    \item \textbf{Total cell count}: Typically 5,000--15,000 cells, representing a 5--8× reduction from the fine mesh.
\end{itemize}

Figure~\ref{fig:mesh_comparison} compares the near-wall mesh structure for fine and coarse configurations, highlighting the difference in resolution.

\begin{figure}[H]
    \centering
    \includegraphics[width=0.95\textwidth]{chapter3/mesh_comparison.png}
    \caption{Comparison of fine and coarse mesh near-wall resolution. (a) Fine mesh with $y^+ < 2$ first cell, showing dense clustering near the wall. (b) Coarse mesh with $y^+ \approx 5$--$10$ first cell, typical of production meshes requiring wall functions.}
    \label{fig:mesh_comparison}
\end{figure}

\subsection{Mesh Quality Metrics}

All generated meshes are verified against quality metrics to ensure reliable CFD solutions:

\begin{table}[H]
\centering
\caption{Mesh quality requirements for fine and coarse meshes.}
\label{tab:mesh_quality}
\begin{tabular}{|l|c|c|c|}
\hline
\textbf{Quality Metric} & \textbf{Fine Mesh} & \textbf{Coarse Mesh} & \textbf{Criterion} \\
\hline
Maximum skewness & $< 0.3$ & $< 0.4$ & $< 0.85$ (OpenFOAM) \\
Maximum aspect ratio & $< 50$ & $< 100$ & $< 1000$ \\
Non-orthogonality & $< 40°$ & $< 50°$ & $< 70°$ \\
Cell volume ratio & $< 3$ & $< 5$ & Adjacent cells \\
\hline
\end{tabular}
\end{table}

\subsection{Grid Independence Study}

To ensure that the fine mesh provides grid-independent results suitable as ground truth, a systematic grid independence study was conducted. Figure~\ref{fig:grid_independence} shows the convergence of skin friction coefficient and Nusselt number with increasing mesh density.

\begin{figure}[H]
    \centering
    \includegraphics[width=0.95\textwidth]{chapter3/grid_independence.png}
    \caption{Grid independence study for a representative diffuser case. (a) Skin friction coefficient convergence with mesh refinement. (b) Nusselt number convergence. Richardson extrapolation values and $\pm 1\%$ bands are shown. The finest mesh (used for ground truth) lies within 1\% of the extrapolated value for both quantities.}
    \label{fig:grid_independence}
\end{figure}

The grid independence study confirms that the fine mesh resolution is sufficient to provide accurate ground truth values. The discretization error is estimated to be less than 1\% for wall quantities, which is acceptable given other sources of uncertainty in RANS simulations.

\section{OpenFOAM Simulation Setup}
\label{sec:ch3_openfoam}

All simulations are performed using OpenFOAM v10, an open-source CFD platform widely used in both academia and industry \cite{2409_19851_v1, 2404_03542_v1}. The solver configuration is designed to produce accurate, converged solutions for turbulent heat transfer in internal flows \cite{2201_03200_v2, 2202_00435_v1}.

\subsection{Governing Equations}

The incompressible Reynolds-Averaged Navier-Stokes (RANS) equations are solved in steady-state form:

\textbf{Continuity equation:}
\begin{equation}
    \nabla \cdot \mathbf{U} = 0
    \label{eq:continuity}
\end{equation}

\textbf{Momentum equation:}
\begin{equation}
    \nabla \cdot (\mathbf{U} \mathbf{U}) = -\frac{1}{\rho}\nabla p + \nabla \cdot \left[(\nu + \nu_t)\left(\nabla \mathbf{U} + (\nabla \mathbf{U})^T\right)\right]
    \label{eq:momentum}
\end{equation}

\textbf{Energy equation:}
\begin{equation}
    \nabla \cdot (\mathbf{U} T) = \nabla \cdot \left[\left(\alpha + \alpha_t\right)\nabla T\right]
    \label{eq:energy}
\end{equation}

where $\mathbf{U}$ is the velocity vector, $p$ is the kinematic pressure, $\nu$ is the kinematic viscosity, $\nu_t$ is the turbulent viscosity, $T$ is temperature, $\alpha$ is thermal diffusivity, and $\alpha_t$ is turbulent thermal diffusivity.

\subsection{Turbulence Modeling}

The $k$-$\omega$ SST (Shear Stress Transport) turbulence model is employed for its demonstrated accuracy in wall-bounded flows with adverse pressure gradients \cite{2005_09023_v2, 2206_05226_v2}. The model blends the $k$-$\omega$ formulation near walls with the $k$-$\epsilon$ formulation in the outer region \cite{launder1974}, combining the strengths of both approaches.

The transport equations for turbulent kinetic energy $k$ and specific dissipation rate $\omega$ are:

\begin{equation}
    \nabla \cdot (\mathbf{U} k) = \nabla \cdot \left[(\nu + \sigma_k \nu_t) \nabla k\right] + P_k - \beta^* \omega k
    \label{eq:k_transport}
\end{equation}

\begin{equation}
    \nabla \cdot (\mathbf{U} \omega) = \nabla \cdot \left[(\nu + \sigma_\omega \nu_t) \nabla \omega\right] + \frac{\gamma}{\nu_t} P_k - \beta \omega^2 + CD_{k\omega}
    \label{eq:omega_transport}
\end{equation}

where $P_k$ is the production of turbulent kinetic energy and $CD_{k\omega}$ is the cross-diffusion term that enables blending between the two formulations.

For fine mesh simulations, no wall functions are used---the equations are integrated directly to the wall with appropriate low-Reynolds-number damping. For coarse mesh simulations, standard OpenFOAM wall functions are applied.

\subsection{Boundary Conditions}

Figure~\ref{fig:boundary_conditions} illustrates the boundary conditions applied to the diffuser geometry. The same structure is used for nozzle and channel configurations with appropriate modifications.

\begin{figure}[H]
    \centering
    \includegraphics[width=0.95\textwidth]{chapter3/boundary_conditions.png}
    \caption{Boundary conditions for the diffuser configuration. Inlet conditions include specified velocity profile and turbulence quantities. Walls are no-slip with fixed temperature. Outlet uses a zero-gradient pressure condition.}
    \label{fig:boundary_conditions}
\end{figure}

\textbf{Inlet conditions:}
\begin{itemize}
    \item Velocity: Uniform profile, $U = U_{in}$
    \item Turbulent kinetic energy: $k = \frac{3}{2}(U_{in} \cdot TI)^2$, where $TI = 0.05$ (5\% turbulence intensity)
    \item Specific dissipation rate: $\omega = \frac{k^{0.5}}{C_\mu^{0.25} \cdot l_t}$, where $l_t = 0.07 H_{in}$ is the turbulent length scale
    \item Temperature: $T = T_{in} = 300$ K
\end{itemize}

\textbf{Outlet conditions:}
\begin{itemize}
    \item Pressure: Fixed value, $p = 0$ (gauge)
    \item All other quantities: Zero gradient
\end{itemize}

\textbf{Wall conditions:}
\begin{itemize}
    \item Velocity: No-slip, $\mathbf{U} = 0$
    \item Temperature: Fixed value, $T_w = 330$ K (isothermal)
    \item Turbulent quantities: Wall functions (coarse) or low-Re treatment (fine)
\end{itemize}

\subsection{Numerical Schemes and Solver Settings}

The following discretization schemes are used:

\begin{table}[H]
\centering
\caption{Numerical discretization schemes used in OpenFOAM simulations.}
\label{tab:schemes}
\begin{tabular}{|l|l|l|}
\hline
\textbf{Term} & \textbf{Scheme} & \textbf{Justification} \\
\hline
Time derivative & steadyState & Steady-state solution \\
Gradient & Gauss linear & Second-order accurate \\
Divergence (U) & Gauss linearUpwind & Bounded, low diffusion \\
Divergence (k, $\omega$, T) & Gauss upwind & Stability for turbulence \\
Laplacian & Gauss linear corrected & Second-order, non-orthogonal \\
Interpolation & linear & Second-order \\
\hline
\end{tabular}
\end{table}

The SIMPLE algorithm is used for pressure-velocity coupling with the following relaxation factors:

\begin{table}[H]
\centering
\caption{Under-relaxation factors for the SIMPLE algorithm.}
\label{tab:relaxation}
\begin{tabular}{|l|c|}
\hline
\textbf{Variable} & \textbf{Relaxation Factor} \\
\hline
Pressure ($p$) & 0.3 \\
Velocity ($\mathbf{U}$) & 0.7 \\
Turbulent kinetic energy ($k$) & 0.5 \\
Specific dissipation rate ($\omega$) & 0.5 \\
Temperature ($T$) & 0.7 \\
\hline
\end{tabular}
\end{table}

\subsection{Convergence Criteria}

Simulations are considered converged when all residuals fall below $10^{-6}$ and monitored quantities (wall shear stress, heat flux) show less than 0.1\% variation over 100 iterations. Figure~\ref{fig:residual_convergence} shows typical residual convergence behavior for a diffuser case.

\begin{figure}[H]
    \centering
    \includegraphics[width=0.85\textwidth]{chapter3/residual_convergence.png}
    \caption{Residual convergence history for a representative diffuser case. All residuals reach the convergence criterion of $10^{-6}$ within 5000 iterations. The velocity and pressure equations converge fastest, followed by temperature and turbulence quantities.}
    \label{fig:residual_convergence}
\end{figure}

\section{Validation Against Benchmark Data}
\label{sec:ch3_validation}

The simulation methodology is validated against established benchmark data to ensure that the fine mesh simulations provide reliable ground truth for neural network training.

\subsection{Channel Flow Validation}

Fully-developed channel flow is validated against the DNS data of Moser, Kim, and Mansour at $Re_\tau = 180$ \cite{moser1999, 2006_12483_v1}. Figure~\ref{fig:velocity_profiles} compares the mean velocity profiles in wall units.

\begin{figure}[H]
    \centering
    \includegraphics[width=0.95\textwidth]{chapter3/velocity_profiles.png}
    \caption{Mean velocity profiles in wall units for three flow configurations. (a) Channel flow compared to DNS data and analytical laws. (b) Diffuser with adverse pressure gradient showing deviation from log-law. (c) Nozzle with favorable pressure gradient. Fine mesh results capture the physics accurately, while coarse mesh results show the behavior that wall functions must correct.}
    \label{fig:velocity_profiles}
\end{figure}

The fine mesh simulation captures the viscous sublayer ($u^+ = y^+$), buffer layer, and log-law region accurately, with deviations from DNS less than 2\% in the log-law region.

\subsection{Diffuser Validation}

Diffuser simulations are validated against the experimental data of Buice and Eaton (1997) for a planar diffuser with 10° total divergence angle. The skin friction coefficient distribution along the wall shows good agreement with experimental measurements, including the prediction of the separation point location.

\subsection{Heat Transfer Validation}

Thermal simulations are validated by comparing the Nusselt number distribution with correlations for turbulent pipe flow. The Dittus-Boelter correlation provides a reference for developed flow regions:

\begin{equation}
    Nu = 0.023 \, Re^{0.8} \, Pr^{0.4}
    \label{eq:dittus_boelter}
\end{equation}

The fine mesh simulations agree with this correlation within 5\% for channel flow cases, providing confidence in the thermal ground truth.

\section{Experimental Benchmark Data for Separated Flows}
\label{sec:ch3_benchmarks}

While our RANS-generated training data covers a comprehensive range of pressure gradient conditions, accurately predicting wall quantities in separated flow regions remains a fundamental challenge. In separated flows, the classical wall function assumptions break down, and models trained only on attached flow data may fail to generalize \cite{driver1985, breuer2009}. To address this limitation, we incorporate high-fidelity experimental and DNS benchmark data from canonical separated flow configurations. This section describes the key benchmark cases and their role in augmenting the training dataset.

\subsection{Backward-Facing Step}
\label{sec:ch3_bfs}

The backward-facing step (BFS) is a canonical benchmark for separated flow studies, featuring sudden expansion that triggers immediate separation at the step corner followed by gradual reattachment downstream \cite{driver1985, 2309_02109_v1}. The geometry and flow structure are illustrated in Figure~\ref{fig:bfs_geometry}.

\begin{figure}[H]
    \centering
    \includegraphics[width=0.95\textwidth]{chapter3/bfs_geometry.png}
    \caption{Backward-facing step geometry and flow structure. (a) Schematic showing the step height $H$, inlet boundary layer, separation at the step corner, recirculation zone, and reattachment downstream. (b) Characteristic flow regions: 1--upstream attached flow, 2--separation shear layer, 3--recirculation zone with $\tau_w < 0$, 4--reattachment region, 5--recovery zone.}
    \label{fig:bfs_geometry}
\end{figure}

The primary experimental reference is Driver and Seegmiller (1985) \cite{driver1985}, conducted at Reynolds number $Re_H = 37{,}500$ based on step height and freestream velocity. This dataset is available through the NASA Turbulence Modeling Resource and has become a standard benchmark for turbulence model validation. Key features include:

\begin{itemize}
    \item \textbf{Expansion ratio}: 1.125 (step height $H$ = 12.7 mm)
    \item \textbf{Inlet boundary layer}: Turbulent, $\delta/H \approx 1.5$
    \item \textbf{Reattachment length}: $x_R/H = 6.26 \pm 0.10$
    \item \textbf{Available measurements}: Skin friction $C_f(x)$, surface pressure $C_p(x)$, velocity profiles $U(y)$ at multiple stations
\end{itemize}

Figure~\ref{fig:bfs_cf_profile} shows the characteristic skin friction distribution along the bottom wall. The $C_f$ distribution exhibits three distinct regions that present different challenges for wall function modeling:

\begin{figure}[H]
    \centering
    \includegraphics[width=0.85\textwidth]{chapter3/bfs_cf_profile.png}
    \caption{Skin friction coefficient distribution for the backward-facing step from Driver \& Seegmiller (1985). The flow separates at the step corner ($x/H = 0$), exhibits negative $C_f$ in the recirculation zone (minimum at $x/H \approx 3$), reattaches at $x/H \approx 6.26$, and recovers downstream. The shaded region indicates where classical wall functions fail.}
    \label{fig:bfs_cf_profile}
\end{figure}

\begin{enumerate}
    \item \textbf{Recirculation zone} ($0 < x/H < 6.26$): Negative $C_f$ indicates reversed flow. The minimum $C_f \approx -0.003$ occurs at approximately $x/H = 3$.
    \item \textbf{Reattachment region} ($x/H \approx 6.26$): $C_f = 0$ at the reattachment point, with high gradients in both $C_f$ and $C_p$.
    \item \textbf{Recovery zone} ($x/H > 6.26$): Gradual recovery toward equilibrium boundary layer values.
\end{enumerate}

DNS data from Le and Moin (1997) \cite{le1997} at $Re_H = 5{,}100$ provides additional high-fidelity reference data with complete turbulence statistics, confirming the experimental measurements and extending them to the full Reynolds stress tensor.

\subsection{Flow Over Periodic Hills}
\label{sec:ch3_periodic_hills}

The periodic hill geometry, established as an ERCOFTAC benchmark case (UFR 3-30), features cyclic separation and reattachment that tests model behavior under repeated flow reversals \cite{breuer2009, 2206_05226_v2}. The geometry is illustrated in Figure~\ref{fig:periodic_hills_geometry}.

\begin{figure}[H]
    \centering
    \includegraphics[width=0.95\textwidth]{chapter3/periodic_hills_geometry.png}
    \caption{Periodic hills geometry. (a) Domain showing the hill profile, with period $L_x = 9H$ and channel height $L_y = 3.035H$. The hill shape is defined by polynomial curves. (b) Flow visualization showing separation on the lee side of each hill and reattachment before the windward face of the next hill.}
    \label{fig:periodic_hills_geometry}
\end{figure}

The comprehensive study by Breuer et al. (2009) \cite{breuer2009} provides both experimental (LDA) and numerical (DNS/LES) reference data across multiple Reynolds numbers:

\begin{itemize}
    \item \textbf{Reynolds numbers studied}: $Re_H = 5{,}600$ (DNS), $10{,}595$ (LES/experiments), $19{,}000$, $37{,}000$
    \item \textbf{Domain size}: $L_x \times L_y \times L_z = 9H \times 3.035H \times 4.5H$
    \item \textbf{Separation point}: $x/H \approx 0.5$ (lee side of hill crest)
    \item \textbf{Reattachment point}: $x/H \approx 4.5$ (dependent on Reynolds number)
    \item \textbf{Available data}: Wall shear stress $\tau_w(x)$, mean velocity profiles, Reynolds stresses
\end{itemize}

Figure~\ref{fig:periodic_hills_cf} shows the wall shear stress distribution over one period, highlighting the regions of favorable pressure gradient (windward side), adverse pressure gradient (lee side), separation, and recovery.

\begin{figure}[H]
    \centering
    \includegraphics[width=0.85\textwidth]{chapter3/periodic_hills_cf.png}
    \caption{Wall shear stress distribution over one period of the periodic hills geometry at $Re_H = 10{,}595$ from Breuer et al. (2009). FPG = favorable pressure gradient (windward side), APG = adverse pressure gradient (lee side). The separation bubble extends from $x/H \approx 0.5$ to $x/H \approx 4.5$.}
    \label{fig:periodic_hills_cf}
\end{figure}

The periodic nature of this geometry makes it particularly valuable for training because each period provides consistent separation/reattachment behavior, enabling the collection of multiple statistically independent samples from a single simulation.

\subsection{Asymmetric Plane Diffuser}
\label{sec:ch3_asymmetric_diffuser}

The asymmetric plane diffuser of Buice and Eaton (1997) \cite{buice1997} represents a more gradual separation scenario compared to the sudden expansion of the backward-facing step. This geometry tests model performance under smooth-wall separation driven by adverse pressure gradient alone, without geometric singularities.

\begin{figure}[H]
    \centering
    \includegraphics[width=0.95\textwidth]{chapter3/diffuser_geometry.png}
    \caption{Asymmetric plane diffuser geometry from Buice \& Eaton (1997). The lower wall diverges at $\theta = 10°$ over a length of $21H$, expanding the channel from height $H$ to $4.7H$. The upper wall remains flat. Fully developed channel flow enters at $x/H = 0$.}
    \label{fig:diffuser_geometry}
\end{figure}

Key characteristics of the Buice-Eaton diffuser:

\begin{itemize}
    \item \textbf{Divergence angle}: $\theta = 10°$ (total included angle)
    \item \textbf{Expansion ratio}: $H_{out}/H_{in} = 4.7$
    \item \textbf{Reynolds number}: $Re_H = 20{,}000$ based on inlet channel height and bulk velocity
    \item \textbf{Inlet condition}: Fully developed turbulent channel flow ($C_f = 0.0061$)
    \item \textbf{Separation location}: Approximately 2/3 of the way along the inclined wall
    \item \textbf{Available measurements}: Wall shear stress (pulsed wall-wire probes), mean and fluctuating velocities, surface pressure
\end{itemize}

Figure~\ref{fig:diffuser_cf} presents the skin friction distribution along the inclined wall, showing the gradual transition from attached flow to separation.

\begin{figure}[H]
    \centering
    \includegraphics[width=0.85\textwidth]{chapter3/diffuser_cf.png}
    \caption{Skin friction coefficient along the inclined wall of the asymmetric diffuser from Buice \& Eaton (1997). The flow gradually decelerates under the adverse pressure gradient, transitioning through incipient separation ($C_f \to 0$) to intermittent reversed flow in the downstream region.}
    \label{fig:diffuser_cf}
\end{figure}

This benchmark is particularly relevant for our diffuser training cases, as it represents a more challenging version of the adverse pressure gradient conditions present in our geometry parameterization.

\subsection{DNS Databases}
\label{sec:ch3_dns_databases}

In addition to experimental data, Direct Numerical Simulation databases provide complete flow field information at high fidelity, enabling extraction of any quantity of interest including higher-order statistics.

\subsubsection{Johns Hopkins Turbulence Database (JHTDB)}

The Johns Hopkins Turbulence Database provides open access to DNS data for various canonical flows:

\begin{itemize}
    \item \textbf{Channel flow at $Re_\tau = 1{,}000$}: 130 TB dataset with full temporal evolution, domain size $8\pi \times 2 \times 3\pi$, resolution $2048 \times 512 \times 1536$
    \item \textbf{Channel flow at $Re_\tau = 5{,}200$}: Snapshots at higher Reynolds number (20 TB)
    \item \textbf{Transitional boundary layer}: 105 TB dataset covering bypass transition
\end{itemize}

These datasets enable validation of our RANS-based training data against fully-resolved turbulence and provide reference wall shear stress distributions under equilibrium conditions.

\subsubsection{Other DNS Resources}

Additional DNS resources include:
\begin{itemize}
    \item Moser, Kim, and Mansour (1999) \cite{moser1999}: Channel flow at $Re_\tau = 180$, $395$, $590$ --- used for validating our channel flow cases
    \item Lee and Moser (2015) \cite{lee2015}: Extended channel flow DNS up to $Re_\tau = 5{,}200$
    \item TU Darmstadt DNS Database: Various wall-bounded turbulent flows with detailed near-wall statistics
\end{itemize}

\subsection{Role of Benchmark Data in Training}
\label{sec:ch3_benchmark_role}

The experimental and DNS benchmark data serve multiple purposes in our machine learning wall function development:

\begin{enumerate}
    \item \textbf{Training data augmentation}: High-fidelity data from separated flow benchmarks fills the gap where RANS simulations are unreliable. This is investigated in the Additional Studies (Chapter~\ref{chap:additional_studies}).

    \item \textbf{Validation}: Benchmark data provides independent validation of model predictions, especially for separated flows outside the training distribution.

    \item \textbf{Physics insight}: The detailed measurements inform feature engineering decisions, identifying which flow quantities are predictable in separated regions.

    \item \textbf{Distribution shift analysis}: Comparing feature distributions between attached (RANS) and separated (experimental) data quantifies the domain shift that limits generalization.
\end{enumerate}

\subsubsection{Simulate-and-Replace Integration Method}
\label{sec:ch3_simulate_replace}

A key challenge is creating compatible training pairs from benchmark data, since experimental measurements typically provide only wall quantities ($C_f$ versus $x/H$) without the full flow field needed for stencil feature extraction. We address this through a \emph{simulate-and-replace} approach:

\begin{enumerate}
    \item Run coarse-mesh RANS simulation of the benchmark geometry with the same mesh specifications ($y^+ \approx 5$--$10$) used for the main training data.
    \item Extract stencil-based input features $\mathbf{X}$ from the coarse mesh, following the same procedure as Section~\ref{sec:ch3_stencil}.
    \item Interpolate experimental/DNS $C_f$ values to the wall mesh points.
    \item Create training pairs $(\mathbf{X}_\text{coarse}, C_{f,\text{exp}})$ by replacing the RANS-predicted $C_f$ with the experimental target.
\end{enumerate}

This approach is illustrated conceptually:
\begin{equation}
    \underbrace{\mathbf{X}_\text{coarse}}_{\text{From RANS}} \xrightarrow{\text{NN}} \hat{C}_f \approx \underbrace{C_{f,\text{exp}}}_{\text{From benchmark}}
\end{equation}

The key insight is that by pairing RANS features with experimental targets, the neural network learns to correct the systematic bias in RANS predictions for separated flows. When the coarse mesh shows features indicative of separation (decelerating flow, adverse pressure gradient, flow reversal near wall), the model learns that the true $C_f$ differs from what RANS would predict.

\subsubsection{Separation-Aware Loss Weighting}

To further emphasize accuracy in separated regions, we employ a weighted loss function:
\begin{equation}
    \mathcal{L} = \frac{1}{N} \sum_{i=1}^{N} w_i \left( \hat{C}_{f,i} - C_{f,i}^\text{true} \right)^2
\end{equation}
where the weight $w_i$ is elevated for samples from separated flow regions:
\begin{equation}
    w_i = 1 + (\alpha_\text{sep} - 1) \cdot \mathbb{1}[C_{f,i} < \epsilon] + (\alpha_\text{bench} - 1) \cdot \mathbb{1}[\text{source}_i = \text{benchmark}]
\end{equation}

Here $\alpha_\text{sep}$ and $\alpha_\text{bench}$ are hyperparameters controlling the emphasis on separated flows and benchmark data respectively. Typical values are $\alpha_\text{sep} = 3$ and $\alpha_\text{bench} = 2$, giving separated benchmark samples up to 4$\times$ the weight of attached RANS samples. This encourages the model to prioritize accuracy in the most challenging flow regimes where traditional wall functions fail.

\subsubsection{Target Variable Consistency: Converting $C_f$ to $\tau_w$}
\label{sec:ch3_cf_conversion}

Our existing training pipeline predicts dimensional wall quantities---wall shear stress $\tau_w$ and wall heat flux $q_w$---rather than non-dimensional coefficients. Since benchmark data typically reports skin friction coefficient $C_f$, we must convert to dimensional form for consistency. The conversion is straightforward:

\begin{equation}
    \tau_w = C_f \times \frac{1}{2} \rho U_{ref}^2
    \label{eq:cf_to_tau}
\end{equation}

where $\rho$ is the fluid density and $U_{ref}$ is the reference velocity (typically bulk velocity or freestream velocity, as defined by each benchmark). This conversion is mathematically exact---$C_f$ is simply the non-dimensional form of $\tau_w$---and introduces no approximation. Each benchmark configuration specifies the reference conditions ($Re$, characteristic length $H$, and thereby $U_{ref}$) needed for this conversion.

For example, for the backward-facing step at $Re_H = 37{,}500$ with step height $H = 12.7$~mm and air at standard conditions ($\nu = 1.5 \times 10^{-5}$~m$^2$/s, $\rho = 1.2$~kg/m$^3$):
\begin{equation}
    U_{ref} = \frac{Re_H \cdot \nu}{H} = \frac{37{,}500 \times 1.5 \times 10^{-5}}{0.0127} \approx 44.3~\text{m/s}
\end{equation}

A measured $C_f = -0.003$ in the recirculation zone then converts to:
\begin{equation}
    \tau_w = -0.003 \times \frac{1}{2} \times 1.2 \times (44.3)^2 \approx -3.5~\text{Pa}
\end{equation}

This dimensional consistency ensures that benchmark-derived training samples integrate seamlessly with RANS-generated data where $\tau_w$ is computed directly from the wall velocity gradient.

\subsubsection{Thermal Data Limitations in Benchmark Experiments}
\label{sec:ch3_thermal_limitation}

A significant limitation of available separated flow benchmarks is the \textbf{scarcity of heat transfer measurements}. While momentum quantities ($C_f$, $C_p$, velocity profiles) are routinely measured using pressure taps, hot-wire anemometry, or LDA, thermal measurements require:

\begin{itemize}
    \item Heated or cooled test section walls
    \item Embedded thermocouples or infrared thermography
    \item Careful thermal insulation to ensure defined boundary conditions
    \item Significantly more complex experimental apparatus
\end{itemize}

Consequently, most canonical separated flow benchmarks provide only momentum data:

\begin{table}[H]
\centering
\caption{Availability of momentum and thermal data in separated flow benchmarks.}
\label{tab:thermal_availability}
\begin{tabular}{lccc}
\toprule
\textbf{Benchmark} & \textbf{$C_f$ / $\tau_w$} & \textbf{$q_w$ / Nu / St} & \textbf{Reference} \\
\midrule
Backward-facing step (Driver \& Seegmiller) & \checkmark & --- & \cite{driver1985} \\
Periodic hills (Breuer et al.) & \checkmark & --- & \cite{breuer2009} \\
Asymmetric diffuser (Buice \& Eaton) & \checkmark & --- & \cite{buice1997} \\
Channel flow DNS (Moser et al.) & \checkmark & \checkmark$^*$ & \cite{moser1999} \\
Heated backward step (Vogel \& Eaton) & \checkmark & \checkmark & \cite{vogel1985} \\
\bottomrule
\multicolumn{4}{l}{\small $^*$DNS databases can include passive scalar transport for thermal analysis.}
\end{tabular}
\end{table}

This creates an asymmetry in our training data: benchmark data constrains $\tau_w$ predictions in separated regions, but $q_w$ predictions remain informed only by RANS-generated data.

\subsubsection{Multi-Task Learning with Partial Targets}
\label{sec:ch3_partial_targets}

To address the thermal data gap, we employ \textbf{multi-task learning with partial targets}. The neural network predicts both outputs simultaneously, but the loss function handles missing targets gracefully:

\begin{equation}
    \mathcal{L} = \frac{1}{N} \sum_{i=1}^{N} \left[ w_i^{(\tau)} \cdot \mathcal{L}_{\tau_w,i} + \mathbb{1}[q_{w,i} \neq \text{NaN}] \cdot w_i^{(q)} \cdot \mathcal{L}_{q_w,i} \right]
    \label{eq:partial_target_loss}
\end{equation}

where $\mathbb{1}[\cdot]$ is the indicator function that excludes $q_w$ loss terms when thermal data is unavailable. In practice:

\begin{itemize}
    \item \textbf{RANS samples}: Both $\tau_w$ and $q_w$ targets are available; both loss terms contribute.
    \item \textbf{Benchmark samples}: Only $\tau_w$ target is available; $q_w$ loss term is masked out.
\end{itemize}

This approach offers several advantages:

\begin{enumerate}
    \item \textbf{Shared feature learning}: The hidden layers learn representations useful for both $\tau_w$ and $q_w$ prediction. Improved $\tau_w$ features from benchmark data may indirectly benefit $q_w$ prediction through transfer within the network.

    \item \textbf{Physical coupling}: Momentum and thermal boundary layers share underlying physics. Features that help distinguish attached from separated flow (e.g., velocity deceleration, flow reversal indicators) inform both outputs.

    \item \textbf{Graceful degradation}: The model learns $\tau_w$ from the combined RANS+benchmark dataset and $q_w$ from RANS data alone. Where benchmark data improves feature representations, $q_w$ benefits; where it does not, $q_w$ predictions default to RANS-trained behavior.
\end{enumerate}

\subsubsection{Expected Outcomes and Future Work}
\label{sec:ch3_expected_outcomes}

Based on the above methodology, we anticipate the following outcomes from integrating benchmark data:

\begin{enumerate}
    \item \textbf{Improved $\tau_w$ prediction in separated regions}: The simulate-and-replace approach, combined with separation-aware loss weighting, should reduce prediction errors where RANS-trained models currently fail. Specifically:
    \begin{itemize}
        \item Correct sign prediction ($\tau_w < 0$) in recirculation zones
        \item Accurate reattachment point location ($\tau_w = 0$)
        \item Improved magnitude prediction in the recovery region
    \end{itemize}

    \item \textbf{Maintained $q_w$ accuracy in attached regions}: Since thermal targets come from RANS data (which is reliable in attached flows), $q_w$ predictions should remain accurate where the training methodology is unchanged.

    \item \textbf{Uncertain $q_w$ behavior in separated regions}: Without benchmark thermal data, $q_w$ predictions in separated flows will rely on:
    \begin{itemize}
        \item Generalization from RANS thermal data
        \item Transfer learning through shared hidden representations
        \item Physical coupling between momentum and thermal boundary layers
    \end{itemize}
    The accuracy of these predictions remains an open question that future work should address through:
    \begin{itemize}
        \item Acquisition of thermal benchmark data (e.g., heated backward step experiments)
        \item DNS with passive scalar transport for benchmark geometries
        \item Physics-informed constraints based on the Reynolds analogy (though this analogy weakens in separated flows where $Pr_t$ varies significantly)
    \end{itemize}
\end{enumerate}

The experiments in Chapter~\ref{chap:additional_studies} will quantitatively evaluate these expected outcomes, comparing models trained with and without benchmark data augmentation across the full range of flow conditions.

Table~\ref{tab:benchmark_summary} summarizes the key characteristics of each benchmark case.

\begin{table}[H]
\centering
\caption{Summary of experimental and DNS benchmark cases for separated flows.}
\label{tab:benchmark_summary}
\begin{tabular}{|l|c|c|c|c|}
\hline
\textbf{Benchmark} & \textbf{$Re$} & \textbf{Separation Type} & \textbf{Data Available} & \textbf{Reference} \\
\hline
Backward-facing step & 37,500 & Geometry-induced & $C_f$, $C_p$, $U(y)$ & Driver \& Seegmiller \\
Periodic hills & 10,595 & Pressure-induced & $\tau_w$, $U(y)$, $\langle u'v'\rangle$ & Breuer et al. \\
Asymmetric diffuser & 20,000 & Smooth APG & $C_f$, $C_p$, $U(y)$ & Buice \& Eaton \\
Channel flow DNS & $Re_\tau$ = 1000 & None (attached) & Full field data & JHTDB \\
\hline
\end{tabular}
\end{table}

The integration of these benchmark datasets with our RANS-generated training data addresses the fundamental challenge identified in Chapter~\ref{chap:literature}: developing wall functions that remain accurate across the full spectrum of flow conditions, from equilibrium attached flows to complex separated regions.

\section{Flow Field Analysis}
\label{sec:ch3_flow_analysis}

Understanding the flow physics is essential for interpreting the training data and ensuring that the neural network learns physically meaningful relationships.

\subsection{Velocity and Pressure Fields}

Figure~\ref{fig:flow_contours} presents contour plots of the flow field variables for a representative diffuser case.

\begin{figure}[H]
    \centering
    \includegraphics[width=0.95\textwidth]{chapter3/flow_contours.png}
    \caption{Flow field visualization for a diffuser with ER = 4.7 and $\theta = 10°$. (a) Velocity magnitude showing deceleration in the expansion region. (b) Pressure field showing pressure recovery. (c) Temperature field showing thermal boundary layer development. (d) Wall quantities ($\tau_w$ and $q_w$) along the bottom wall, with the expansion region shaded.}
    \label{fig:flow_contours}
\end{figure}

The velocity field shows the expected deceleration in the diffuser expansion region due to area increase. The pressure field shows pressure recovery as kinetic energy is converted to pressure energy. The temperature field shows thermal boundary layer growth along the heated wall.

\subsection{Effect of Pressure Gradient}

The pressure gradient has a profound effect on the boundary layer structure:

\begin{itemize}
    \item \textbf{Adverse pressure gradient (APG)}: The boundary layer thickens, the velocity profile becomes less full, and the wall shear stress decreases. In severe cases, flow separation occurs where $\tau_w < 0$.

    \item \textbf{Favorable pressure gradient (FPG)}: The boundary layer thins, the velocity profile becomes fuller, and the wall shear stress increases.

    \item \textbf{Zero pressure gradient (ZPG)}: The boundary layer grows gradually according to classical flat plate theory.
\end{itemize}

These effects are reflected in the training data, and the neural network must learn to distinguish between these regimes based on local flow features.

\section{Stencil-Based Feature Extraction}
\label{sec:ch3_stencil}

For each wall-adjacent cell in the coarse mesh, a structured $3 \times 5$ stencil of neighboring cells is extracted. This stencil captures local flow context in both streamwise and wall-normal directions, providing the neural network with information beyond the immediate wall-adjacent cell.

\subsection{Stencil Structure}

The stencil consists of 15 cells arranged in a 3 (streamwise) $\times$ 5 (wall-normal) grid centered on the wall-adjacent cell of interest. Figure~\ref{fig:stencil_detailed} illustrates the stencil structure and data organization.

\begin{figure}[H]
    \centering
    \includegraphics[width=0.95\textwidth]{chapter3/stencil_detailed.png}
    \caption{Stencil extraction methodology. (a) The $3 \times 5$ stencil on the computational mesh, with the center cell (target location) highlighted in red and stencil cells in blue. (b) Data structure showing the indexing convention and variables extracted per cell. Each stencil provides 90 input values (15 cells $\times$ 6 variables).}
    \label{fig:stencil_detailed}
\end{figure}

\subsection{Stencil Extraction on Curved Surfaces}

A key capability of our methodology is the ability to extract stencils on curved or inclined wall surfaces. This is achieved through local wall-aligned coordinate transformations. At each wall location:

\begin{enumerate}
    \item The local \textbf{wall tangent} direction is computed from the wall geometry.
    \item The local \textbf{wall normal} direction is perpendicular to the tangent.
    \item Cell neighbors are selected based on their position in this local coordinate system.
\end{enumerate}

Figure~\ref{fig:stencil_curved} demonstrates stencil extraction on two curved geometries: a diffuser expansion region and a wall-mounted hump. The local coordinate system adapts to the wall orientation, ensuring consistent feature extraction regardless of wall shape.

\begin{figure}[H]
    \centering
    \begin{subfigure}[b]{0.48\textwidth}
        \includegraphics[width=\textwidth]{chapter3/stencil_curved_diffuser.png}
        \caption{Diffuser expansion region}
    \end{subfigure}
    \hfill
    \begin{subfigure}[b]{0.48\textwidth}
        \includegraphics[width=\textwidth]{chapter3/stencil_curved_hump.png}
        \caption{Wall-mounted hump}
    \end{subfigure}
    \caption{Stencil extraction on curved surfaces using local wall-aligned coordinates. The tangent (green) and normal (orange) vectors adapt to the local wall orientation. Cell indices $(i,j)$ are defined in this local coordinate system, ensuring consistent feature extraction regardless of wall curvature.}
    \label{fig:stencil_curved}
\end{figure}

This wall-aligned extraction enables the trained model to be deployed on arbitrary geometries by applying the same local coordinate transformation at runtime.

\subsection{Stencil Extraction Algorithm}

The extraction algorithm proceeds as follows for each wall-adjacent cell:

\begin{enumerate}
    \item Identify the wall-adjacent cell at streamwise position $i$.
    \item Compute the local wall tangent and normal vectors.
    \item Extract cells at positions $(i-1, i, i+1)$ in the tangential (streamwise) direction.
    \item For each tangential position, extract cells at wall-normal layers $j = 0, 1, 2, 3, 4$.
    \item Collect the primitive variables from each cell.
    \item Store the stencil data as a tensor of shape $(3, 5, 6)$ or flattened to dimension 90.
\end{enumerate}

Special handling is required at domain boundaries:
\begin{itemize}
    \item At the inlet boundary, the upstream cell is extrapolated from interior values.
    \item At the outlet boundary, the downstream cell is extrapolated similarly.
    \item Cells beyond the wall-normal extent are not included (the stencil is truncated if necessary).
\end{itemize}

\subsection{Input Variables}

For each of the 15 cells in the stencil, six primitive variables are extracted:

\begin{table}[H]
\centering
\caption{Primitive variables extracted for each stencil cell.}
\label{tab:input_variables}
\begin{tabular}{|l|l|l|}
\hline
\textbf{Variable} & \textbf{Symbol} & \textbf{Units} \\
\hline
Streamwise coordinate & $x$ & m \\
Wall-normal coordinate & $y$ & m \\
Pressure & $p$ & Pa \\
Streamwise velocity & $U_x$ & m/s \\
Wall-normal velocity & $U_y$ & m/s \\
Temperature & $T$ & K \\
\hline
\end{tabular}
\end{table}

This yields an input tensor of shape $(3, 5, 6)$ or a flattened vector of dimension $3 \times 5 \times 6 = 90$.

\section{Ground Truth Computation}
\label{sec:ch3_ground_truth}

The neural network is trained to predict two wall quantities: wall shear stress and wall heat flux. These are computed from the fine mesh solution.

\subsection{Wall Shear Stress}

The wall shear stress is computed from the velocity gradient at the wall:

\begin{equation}
    \tau_w = \mu \left.\frac{\partial U_x}{\partial y}\right|_{y=0}
    \label{eq:tau_w}
\end{equation}

where $\mu$ is the dynamic viscosity. In OpenFOAM, this is computed using the \texttt{wallShearStress} function object, which evaluates the viscous stress at wall boundaries.

For non-dimensional analysis, the skin friction coefficient is defined as:

\begin{equation}
    C_f = \frac{\tau_w}{\frac{1}{2} \rho U_{ref}^2}
    \label{eq:cf}
\end{equation}

where $U_{ref}$ is the inlet velocity.

\subsection{Wall Heat Flux}

The wall heat flux is computed from the temperature gradient at the wall:

\begin{equation}
    q_w = -k \left.\frac{\partial T}{\partial y}\right|_{y=0}
    \label{eq:q_w}
\end{equation}

where $k$ is the thermal conductivity. The negative sign indicates that positive $q_w$ corresponds to heat transfer into the fluid (when $T_w > T_{fluid}$).

For non-dimensional analysis, the Stanton number is defined as:

\begin{equation}
    \mathrm{St} = \frac{q_w}{\rho C_p U_{ref} (T_w - T_{in})}
    \label{eq:st}
\end{equation}

\subsection{Handling of Separated Flow}

In regions of flow separation, the wall shear stress becomes negative ($\tau_w < 0$), indicating reversed flow near the wall. The neural network must be capable of predicting this behavior. The training data includes cases with:

\begin{itemize}
    \item Attached flow: $\tau_w > 0$ throughout
    \item Incipient separation: $\tau_w \approx 0$ locally
    \item Separated flow: $\tau_w < 0$ in the separation bubble
\end{itemize}

The sign of $\tau_w$ is preserved in the training data to enable the network to learn separation prediction.

\section{Dataset Summary and Statistics}
\label{sec:ch3_dataset}

The complete training dataset consists of paired input-output samples extracted from the 244 geometry configurations.

\begin{table}[H]
\centering
\caption{Summary of the generated training dataset.}
\label{tab:dataset_summary}
\begin{tabular}{|l|c|}
\hline
\textbf{Parameter} & \textbf{Value} \\
\hline
Total geometry configurations & 244 \\
\quad Diffuser cases & 180 \\
\quad Nozzle cases & 60 \\
\quad Channel cases & 4 \\
\hline
Total training samples & 25,485 \\
Input features per sample & 90 (primitive) \\
Output targets per sample & 2 ($C_f$, St) \\
\hline
Expansion ratio range & 0.5--5.5 \\
Divergence angle range & $0°$--$20°$ \\
Reynolds number range & 8,000--24,000 \\
\hline
\end{tabular}
\end{table}

\subsection{Dataset Statistics}

Figure~\ref{fig:dataset_statistics} presents the statistical distributions of the output variables and their correlations.

\begin{figure}[H]
    \centering
    \includegraphics[width=0.95\textwidth]{chapter3/dataset_statistics.png}
    \caption{Statistical analysis of the training dataset. (a) Distribution of skin friction coefficient $C_f$ values. (b) Distribution of Stanton number St values. (c) Correlation between $C_f$ and St, showing the expected positive relationship. (d) Summary statistics table including mean, standard deviation, and percentiles.}
    \label{fig:dataset_statistics}
\end{figure}

Key observations from the dataset statistics:

\begin{itemize}
    \item The $C_f$ distribution is positively skewed, with a long tail toward higher values corresponding to accelerating flow regions.
    \item The St distribution shows similar characteristics, reflecting the Reynolds analogy between momentum and heat transfer.
    \item The correlation coefficient between $C_f$ and St is approximately 0.85, indicating a strong but not perfect relationship that the neural network can exploit.
\end{itemize}

\section{Data Splitting for Machine Learning}
\label{sec:ch3_preprocessing}

Prior to neural network training, the dataset is split into training (70\%) and test (30\%) sets. Importantly, the split is performed at the \textbf{geometry level}, not the sample level:

\begin{itemize}
    \item All samples from a given geometry configuration belong entirely to either the training or test set.
    \item This ensures the test set evaluates true generalization to unseen geometries, not interpolation between similar flow conditions.
    \item All geometry types (diffuser, nozzle, channel) are represented in both sets.
\end{itemize}

Feature normalization and scaling strategies are discussed in Chapter~\ref{chap:physics_features}, where we compare different input representations including primitive variables and physics-based features.

\section{Training Data Sources Strategy}
\label{sec:ch3_data_sources_strategy}

The preceding sections have presented two complementary sources of training data: RANS-generated data from our dual-mesh methodology and high-fidelity benchmark data from DNS and experimental studies. This section synthesizes these approaches into a unified training strategy illustrated in Figure~\ref{fig:data_sources_strategy}.

\begin{figure}[H]
    \centering
    \includegraphics[width=0.98\textwidth]{chapter3/data_sources_strategy.png}
    \caption{Training data sources strategy for ML wall functions. (a) Flow regions and corresponding data sources: attached flow regions use RANS-generated data, while separated flow regions require DNS or experimental benchmarks. (b) Summary of available benchmark cases with their characteristics. (c) Decision flowchart for data source selection. (d) Comparison of strengths and limitations of each data source, motivating the combined strategy.}
    \label{fig:data_sources_strategy}
\end{figure}

\subsection{Region-Based Data Source Selection}
\label{sec:ch3_region_selection}

The fundamental insight driving our training strategy is that different flow regions have different data requirements:

\begin{itemize}
    \item \textbf{Attached flow regions} ($\tau_w > 0$, mild pressure gradients): The classical wall function assumptions \cite{launder1974, spalding1961} remain approximately valid, and RANS simulations with well-resolved meshes provide reliable ground truth. Our dual-mesh approach generates abundant training data for these conditions efficiently.

    \item \textbf{Separated flow regions} ($\tau_w \leq 0$, strong adverse pressure gradients): The assumptions underlying both traditional wall functions and RANS closures become questionable. DNS or carefully validated experimental data provide the only reliable ground truth for these challenging conditions.

    \item \textbf{Transitional regions} (near separation and reattachment): These present the greatest modeling challenge, as flow physics are neither fully attached nor fully separated. Benchmark data from the backward-facing step, periodic hills, and diffuser cases specifically capture these critical transitional behaviors.
\end{itemize}

\subsection{Practical Data Source Comparison}
\label{sec:ch3_source_comparison}

Table~\ref{tab:data_sources_comparison} summarizes the practical trade-offs between data sources:

\begin{table}[H]
\centering
\caption{Comparison of training data sources for ML wall functions.}
\label{tab:data_sources_comparison}
\begin{tabular}{p{3.5cm}p{5cm}p{5cm}}
\toprule
\textbf{Aspect} & \textbf{RANS (This Work)} & \textbf{DNS/Experimental} \\
\midrule
Computational cost & Low (minutes per case) & High (weeks for DNS) \\
Parameter coverage & Extensive (244 configurations) & Limited (fixed conditions) \\
Geometry flexibility & Any parameterized geometry & Fixed benchmark geometries \\
Fidelity in attached flow & Good (validated against DNS) & Excellent (ground truth) \\
Fidelity in separated flow & Questionable (RANS limitations) & Excellent (ground truth) \\
Wall function assumptions & Embedded in fine mesh solution & None (wall-resolved) \\
Boundary condition control & Full control & Fixed by experiment \\
\bottomrule
\end{tabular}
\end{table}

\subsection{Combined Training Strategy}
\label{sec:ch3_combined_strategy}

Based on the above analysis, we employ a combined training strategy:

\begin{enumerate}
    \item \textbf{Primary training data}: The RANS-generated dataset (25,485 samples) provides comprehensive coverage of attached and mildly separated flows across diverse geometries and Reynolds numbers.

    \item \textbf{Separation augmentation}: Benchmark data from backward-facing step, periodic hills, and asymmetric diffuser cases augment training in separated flow regions where RANS accuracy is limited.

    \item \textbf{Transfer learning}: Models pre-trained on RANS data can be fine-tuned on benchmark data to improve separation prediction while retaining attached flow accuracy.

    \item \textbf{Physics-informed constraints}: The Reynolds analogy relationship between momentum ($C_f$) and heat transfer (St) provides additional constraints that hold across data sources, enabling consistent multi-task learning.
\end{enumerate}

This hybrid approach leverages the strengths of each data source: RANS provides breadth of parameter coverage, while DNS and experimental benchmarks provide depth in challenging flow regimes. The subsequent chapters will demonstrate how this combined dataset enables neural networks to generalize across the full range of wall-bounded turbulent flows.

\section{Chapter Summary}
\label{sec:ch3_summary}

This chapter has presented the comprehensive methodology for generating structured training data for machine learning wall functions. The key contributions and findings are:

\begin{enumerate}
    \item \textbf{Dual-mesh methodology}: A systematic approach pairing fine mesh ($y^+ < 2$) ground truth with coarse mesh ($y^+ \approx 5$--$10$) input features enables supervised learning of wall quantities without requiring expensive DNS data.

    \item \textbf{Comprehensive geometry coverage}: The dataset spans 244 configurations including diffusers (adverse pressure gradient), nozzles (favorable pressure gradient), and channels (zero pressure gradient), covering Reynolds numbers from 8,000 to 24,000.

    \item \textbf{Validated simulation methodology}: Grid independence studies and comparison with benchmark data confirm that the fine mesh simulations provide reliable ground truth with less than 2\% discretization error.

    \item \textbf{Structured stencil extraction}: The $3 \times 5$ stencil provides 90 primitive variable inputs per sample, capturing local flow context including pressure gradient effects and boundary layer structure.

    \item \textbf{Complete dataset}: The final dataset contains 25,485 paired samples suitable for training neural networks to predict both velocity (through $C_f$) and thermal (through St) wall functions.

    \item \textbf{Hybrid data sources strategy}: RANS-generated data provides comprehensive coverage of attached flows, while DNS and experimental benchmarks (backward-facing step, periodic hills, Buice-Eaton diffuser) augment training data for separated flow regions where RANS accuracy is limited. This region-based approach leverages the strengths of each data source.

    \item \textbf{Target consistency and thermal limitations}: Benchmark skin friction data ($C_f$) is converted to wall shear stress ($\tau_w$) via the straightforward relation $\tau_w = C_f \times \frac{1}{2}\rho U_{ref}^2$, ensuring dimensional consistency with RANS-generated targets. However, most separated flow benchmarks lack heat transfer measurements, creating an asymmetry where $\tau_w$ benefits from high-fidelity augmentation while $q_w$ relies primarily on RANS data. This limitation is addressed through multi-task learning with partial targets, where shared feature representations may transfer improvements across outputs.
\end{enumerate}

The methodology presented in this chapter provides the foundation for the machine learning experiments in subsequent chapters. In Chapter~\ref{chap:baseline}, we establish baseline performance using primitive variables as network inputs. Chapter~\ref{chap:physics_features} investigates whether physics-based feature engineering can improve upon this baseline.

\end{document}
